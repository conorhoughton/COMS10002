
%worksheet3_background.tex 
%a note on hash tables as background for the worksheet 3
%programming problem for the course PandA1 COMS10002
%taught at the University of Bristol 2015-10-29 Conor Houghton
%conor.houghton@bristol.ac.uk

%To the extent possible under law, the author has dedicated all copyright 
%and related and neighboring rights to these notes to the public domain 
%worldwide. These notes are distributed without any warranty. 


\documentclass[11pt,a4paper]{scrartcl}
\typearea{12}
\usepackage{graphicx}
\usepackage{pstricks}
\usepackage{listings}
\usepackage{color}
\lstset{language=C}
\pagestyle{headings}
\markright{COMS10002 - PandA1 algorithms worksheet 3 - Conor}
\begin{document}

\begin{verbatim}

Algorithms Worksheet 3

This assignment requires you to compare two ways of storing and
retrieving data records. In order to do the comparison, you must write
a program which stores data records in a linked list and a hash table
with direct chaining as collision resolution strategy. The data to be
used for this assignment is the story "A Portrait of the Artist as a
Young Man" by James Joyce. You can download it and save it as a text
file. Your program must take the file name as an argument to main() so
if your program is called foo.c and is compiled to foo and the text
file is called a_portrait.txt, you would call your program using

foo a_portrait.txt

When your program has read the input text file, it is expected to
answer queries via a user interface. For a query, the user types in a
keyword. To answer a query, your program prints the number of
occurrences of the given keyword in the input text file and possibly
some other statistics. Hence

>./foo example.txt

Reading into a list took 1.05s for 44 words.
Reading into a table took 0.54s for 44 words.

Enter word for retrieval: time

'time' occurs 3 times.
Retrieval from list took 9 comparisons.
Retrieval from table took 3 comparisons.

Another query? [y/n] y
Enter word for retrieval: your

'your' occurs 2 times.
Retrieval from list took 19 comparisons.
Retrieval from table took 2 comparisons.

Another query? [y/n] n
>

[The information about the number of comparisons is an optional extra]

Program Design

Choose a systematic approach when designing your program: Define the
data structures that you need. Define the functions that you need to
operate on these data structures. Re-use types and functions wherever
possible. If a problem is too difficult to solve in one go, divide and
conquer. A good starting point is to first develop a program that uses
the linked list approach - a sample version of this has been
supplied. Several functions of this code can be re-used when extending
the program to use a hash table. (Note: You must hand in ONE program
only; not one for the linked list and another one for the hash table!)
A good solution will have at least 6 functions in addition to the main
program: A function that reads the data.  A function that inserts data
into the linked list.  A function that retrieves data from the linked
list.  A function that inserts data into the hash table.  A function
that retrieves data from the hash table.  A function that handles the
queries.  Choose the hash function and the size of the hash table
carefully. You will have to set some conventions e.g. for dealing with
capitalised words, or phrases like "I'll" and "we've". There is not a
single right solution. Choose something which seems sensible to you.

Assessment

You will have to mark your assignment yourself, using the marking
scheme provided on the marksheet. These marks will be checked.

Submission

Your submission should contain:

The source code of your program.

Sample output showing:

The time taken to organise the data in a) the linked list and b) the
hash table; and the total number of words read.

It might give the size of the hash table, the number of empty indexes,
what percentage of the hash table is used (i.e. not empty) and the
maximum length of an overflow list.

It should list the number of occurances of the words

looking, me, epiphany, seraphim, ardent, windless, esthetic, computer.

in the text. It might also give the number of comparisons necessary to retrieve the words.

The sample output should be submitted in a file called OUTPUT.txt.


You may submit a short report explaining anything you might think
would help with the marking of an unfinished project.

A marksheet where you have filled in the blanks. Download the
marksheet from the github and edit it.

Some suggest code for loading the file has been included in the folder.
\end{verbatim}
\end{document}
