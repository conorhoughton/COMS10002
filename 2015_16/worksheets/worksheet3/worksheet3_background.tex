%worksheet3_background.tex 
%a note on hash tables as background for the worksheet 3
%programming problem for the course PandA1 COMS10002
%taught at the University of Bristol 2015-10-29 Conor Houghton
%conor.houghton@bristol.ac.uk

%To the extent possible under law, the author has dedicated all copyright 
%and related and neighboring rights to these notes to the public domain 
%worldwide. These notes are distributed without any warranty. 


\documentclass[11pt,a4paper]{scrartcl}
\typearea{12}
\usepackage{graphicx}
\usepackage{pstricks}
\usepackage{listings}
\usepackage{color}
\lstset{language=C}
\pagestyle{headings}
\markright{COMS10002 - PandA1 algorithms worksheet 3 - Conor}
\begin{document}

\subsection*{Algorithms Worksheet 3 background - hash tables}
The aim of this problem sheet is to implement a hash table and compare
its performance to a linked list. A hash table is a quick way to
access stored data; it is a frequently used and invaluable way to
store date. It can be used in a very general way but as a definite
example we will consider a hash table which counts how many times
words are used in a book. This example was also used in the Haskell
course where a trie was used. The clever thing about a trie is that
the word itself was used as a kind of instruction for how to get to
the location storing how many times the word was used. This is very
elegant, but not as generic as a hash table. A hash table does share
some part of the trick though, it uses the word as an instruction for
finding the storage location.

In our problem we have a \textbf{key}, a word in our case, and a
\textbf{bucket} which stores the information we want to associate with
the key. In our case the bucket will hold the number of uses of the
word. The idea behind a hash table is to store the buckets in an
array, we will have to change this slightly, but imagine each bucket
has an index, so, if the array was called \texttt{hash\_array}, there
would be buckets located at \texttt{hash\_array[0]},
\texttt{hash\_array[1]} and so on. The challenge is to go from the key
to the correct index. The simplest strategy is to look in each bucket
in turn and stopping when you find the correct one. This is clearly
very slow, we will experiment with this in the exercise. Another
strategy would be to make a big lookup table of all the indices and
using the alphabetic structure of the keys to seach it using a binary
search algorithm, this would be smart but it does rely on the keys
having an ordering, they do here when the keys are words, but we want
to solve the problem in a more general way.

The answer is to find some way of working out the index from the key
itself, in other words, a hash table starts with a map, called the
\textbf{hashing function} which maps each key to an index:
\begin{eqnarray}
h:\mbox{keys}&\rightarrow&\mbox{indices}\\
\mbox{key}&\mapsto&h(\mbox{key})
\end{eqnarray}
What might this hashing function look like, well that depends on the
data, on the type of key, the performance constraints on the problem
and the amount of data. An obvious example for words would be to
convert the letters into numbers with \texttt{a} going to 0,
\texttt{b} to 1 and so on and then adding the values so
\begin{equation}
h(\mbox{\lq{}elbow\rq{}})=4+11+1+14+22=52
\end{equation}
Obviously this scheme would have to be changed slightly if there were
capitals or other letters, so, for example, ascii could be used. 

The idea now is to have a big list \texttt{hash\_table} and in this
table \texttt{hash\_table[52]} would store the bucket for 'elbow'. One
immeadiate problem is that we might have some large words, like zizzerzazzerzuzz, a creature in the Dr Seuss universe;
\begin{equation}
h(\mbox{\lq{}zizzerzazzerzuzz\rq{}})=295
\end{equation}
Obviously if this index was unexpectedly high and \texttt{hash\_table}
had less than 296 entries then this would be a problem. This problem
could be solve by making sure the hash function had a predefined
range, for example by using \texttt{mod} so
\begin{equation}
h(x)=\sum{(\mbox{value of letters in }x)}\% N
\end{equation}
where $N$ is the size of \texttt{hash\_table}.

However, this discussion exposes a greater problem: 296 is not a big
number compared to the number of words and if zizzerzazzerzuzz only
has the hash value 295 there must be lots of words that share the same
hash value. Obviously two words that are anagrams of each other have
the same hash value under this scheme:
\begin{equation}
h(\mbox{\lq{}male\rq{}})=h(\mbox{\lq{}lame\rq{}})
\end{equation}
This situation can be improved by using a better hash function, the
one I have described is terrible. An example hash function is given in
\texttt{djb2.c}. However, although the number of \textbf{collisions},
that is cases were
\begin{equation}
h(x_1)=h(x_2)
\end{equation}
when $x_1\not=x_2$, can be reduced, it is very hard to make sure it
never happens; perfect hash function where there are no collisions
are usually slow or impractical in other ways.

The answer to this problem is to make \texttt{hash\_table} an array of
linked lists instead of an array holding the buckets directly. Thus,
when looking for the bucket with key $x$, you go to the linked list at
\texttt{hash\_table[$h(x)$]} and then search down the linked list
until the bucket for key $x$ is found. This approach, called
\textbf{direct chaining} isn't the only way to deal with the problem
of collisions, but it is one of the most straight forward.



\end{document}
