%worksheet1.tex
%problem set for the course PandA1 COMS10002 taught at the University of Bristol
%2015-10-10 Conor Houghton conor.houghton@bristol.ac.uk

%To the extent possible under law, the author has dedicated all copyright 
%and related and neighboring rights to these notes to the public domain 
%worldwide. These notes are distributed without any warranty. 


\documentclass[11pt,a4paper]{scrartcl}
\typearea{12}
\usepackage{graphicx}
\usepackage{pstricks}
\usepackage{listings}
\usepackage{color}
\lstset{language=C}
\pagestyle{headings}
\markright{COMS10002 - PandA1 algorithms worksheet 1 - Conor}
\begin{document}

\subsection*{Algorithms Worksheet 1}

This worksheet contains a mostly pen and paper calculations. The
solutions should be submitted as plain text, files in other formats,
like .doc or .pdf will not be accepted. You can write $x^\wedge n$ and
$a\_n$ for $x^n$ and $a_n$. For each part of a question write the
answer and a short description of how the answer was obtained, so, for
example, each part of questions 1 and 2 should have a one sentence
description. Each question is worth a fifth of the marks, question 4
and 5 are the extension questions.

\begin{enumerate}

\item This question is about estimating the algorithmic
  complexity of simple operations. Take multiply and addition of a
  single digit as roughly one operation and assume that powers are
  calculated by doing the corresponding number of multiplications.

\begin{enumerate}
\item What is the big-oh complexity of adding two $n$-digit numbers?
\item What is the big-oh complexity of multiplying two $n$-digit numbers
  using the standard long multiplication method.
\item What is the big-oh complexity of calculating $x^n$ of a single
  digit number?
\item What is the big-oh complexity of calculating $x^n$ of an $m$
  digit number?
\end{enumerate}


\item This question is about estimating the algorithmic complexity of
  evaluating a polynomial. Here consider fixed sized variables, so
  multiplication and addition take roughly one step, irrespective of
  how many digits the number has. Once again, powers are calculated by
  multiplication.

\begin{enumerate}
\item What is the big-oh complexity of evaluating, that is finding the
  value of $p(x)$, of an order $n$ polynomial
$$p(x)=a_n x^n +a_{n-1}x^{n-1}+\ldots+a_1x+a_0$$
using straight-forward substitution?
\item Horner's method is a quicker method for evaluating a
  polynomial. If $x_o$ is the value that the polynomial needs to be
  evaluated on, let $b_n=a_n$ and then 
$$ b_{n-1}=a_{n-1}+x_o b_{n}$$
and
$$ b_{n-2}=a_{n-2}+x_0 b_{n-1}$$
right down to 
$$ b_0=a_0+x_0b_1$$ and $b_0=p(x_0)$ is the answer. What is the big-oh
complexity?
\end{enumerate}

\item This question is intended to help you practice counting
  things. In the Arthur C. Clarke short story \textsl{The Nine Billion
    Names of God} a \lq{}Mark V Automatic Sequence Computer\rq{} is
  purchase by a religious community to list all the possible names of
  their deity, a task whose importance is central to their belief
  system. We are told that this name is nine or fewer characters long
  and that no characters can occur more than three times in sequence,
  so AABBAABBA is a possible name, but AAABBAABB is not. We are not
  told how many characters the alphabet contains. Assuming there are
  18 letters and $n$ or few words, how many names are there.

\item Consider another arithmetic operation, or a different algorithm for one of the operations above, describe it and estimate the big-oh complexity. Examples might be long division or the peasant algorithm for multiplication.

\item Consider some other non-sorting algorithm, describe it and
  estimate the big-oh complexity. Examples might be the Tower of Hanoi
  or the Traveling Salesman problem.
\end{enumerate}



\end{document}
