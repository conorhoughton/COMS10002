%worksheet2.tex
%problem set for the course PandA1 COMS10002 taught at the University of Bristol
%2015-10-29 Conor Houghton conor.houghton@bristol.ac.uk

%To the extent possible under law, the author has dedicated all copyright 
%and related and neighboring rights to these notes to the public domain 
%worldwide. These notes are distributed without any warranty. 


\documentclass[11pt,a4paper]{scrartcl}
\typearea{12}
\usepackage{graphicx}
\usepackage{pstricks}
\usepackage{listings}
\usepackage{color}
\lstset{language=C}
\pagestyle{headings}
\markright{COMS10002 - PandA1 algorithms worksheet 2 - Conor}
\begin{document}

\subsection*{Algorithms Worksheet 2}

This worksheet contains mostly pen and paper calculations. The
solutions should be submitted as plain text, that is .txt, files in
other formats, like .rtf, .tex, .doc or .pdf will not be accepted. You
can write $x^\wedge n$ and $a\_n$ for $x^n$ and $a_n$. Where
applicable, write a short description of how the answer is
obtained. The graph in part 6 should be submitted as an eps or pdf
file.

\begin{enumerate}

\item This question is about solving recursion relations using
  telescoping. In each case find the value of $T(n)$ by
  telescoping. Check your answer by substitution, it is permisseable
  to combine these two steps by using telescoping to come up with an
  ansatz and then substituting it fix values in the ansatz. Write down
  the big-Theta for the solution. [25 marks] 

\begin{enumerate}
\item $T(n)=T(n-1)+3$ with $T(0)=1$
\item $T(n)=T(n-1)+3$ with $T(1)=1$
\item $T(n)=T(n-1)+3n$ with $T(1)=1$
\item $T(n)=2T(n-1)+3$ with $T(0)=1$
\item $T(n)=3T(n-1)+2$ with $T(0)=1$
\end{enumerate}

\item This question is about the asymptotic behavior of
  different functions, in each case give big-Theta for $T(n)$; if
  $T(n)$ was the worst case run-time this would give big-Oh. There is
  no need to give any working for this problem.  [30 marks]

\begin{enumerate}

\item $T(n)=n^5+\frac{1}{n}+n(n-1)(n+2)^4$
\item $T(n)=n^2\log{n}+n^3$
\item $T(n)=2^n+n!$
\item $T(n)=\sum_{i=0}^ni$
\item $T(n)=\sqrt{n}n+n$
\item $T(n)=n^2/\log{n}+n$
\item $T(n)=(n^5+345n^4+36n)/(n^2+2n+1)$
\item $T(n)=1/(n^2+2n+1)$
\item $T(n)=[(n+1)(n+2)(n+3)]/[(n+4)(n+5)]$
\item $T(n)=n!/(n-1)!$

\end{enumerate}

\item Solve for $T(n)$ using the ansatz $T(n)=r^n$ for the following
  two step recursion relations. Solving for $r$ will give two values
  $r_1$ and $r_2$, this means that general solution will be
  $T(n)=Ar_1^n+Br_2^n$. Use the two base values to find $A$ and $B$.  [10 marks]

\begin{enumerate}
\item $T(n)=T(n-1)+3T(n-2)$ with $T(0)=0$ and $T(1)=5$.
\item $T(n)=T(n-2)$ with $T(0)=0$ and $T(1)=2$.
\end{enumerate}

\item This question is about the master theorem. Use it to
  calculate big-Theta for $T(n)$ in each case.  [15 marks] 

\begin{enumerate}
\item $T(n)= 25T(n/5)+4n^2$
\item $T(n)= 20T(n/5)+4n$
\item $T(n)= 16T(n/2)+2n^4$
\end{enumerate}

\item In the section 1\_introduction the actual run time
  for insert sort was plotted against the size of the set. Using
  another algorithm whose asypmtotic behavior you know, make a similar
  plot. I use \texttt{gnuplot} to do plots and instructions for this
  are provided in the 1\_introduction folder, you can use other
  plotting packages but please submit the graph as pdf or eps.  [20 marks]

\end{enumerate}

\end{document}
