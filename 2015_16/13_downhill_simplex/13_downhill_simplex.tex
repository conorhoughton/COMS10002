%13_downhill_simplex.tex
%notes for the course PandA1 COMS10002 taught at the University of Bristol
%2015 Conor Houghton conor.houghton@bristol.ac.uk

%To the extent possible under law, the author has dedicated all copyright 
%and related and neighboring rights to these notes to the public domain 
%worldwide. These notes are distributed without any warranty. 

\documentclass[11pt,a4paper]{scrartcl}
\typearea{12}
\usepackage{graphicx}
\usepackage{pstricks}
\usepackage{listings}
\usepackage{tikz}
\usetikzlibrary{calc,positioning}
\usetikzlibrary{shapes,arrows}
\lstset{language=C}
\pagestyle{headings}
\markright{COMS10002 - PandA1 13\_downhill\_simplex - Conor}

\begin{document}

\subsection*{13 - downhill simplex}

\begin{figure}
\begin{center}
\begin{tikzpicture}
\coordinate (x1) at (0,0);
\coordinate (x2) at (5,0);
\coordinate (x3) at (2.5,4.33);
\node[draw=red](x1n) at (x1) {$x_1$};
\node[draw=red](x2n) at (x2) {$x_2$};
\node[draw=red](x3n) at (x3) {$x_3$};
\draw (x1n) -- (x2n);
\draw (x2n) -- (x3n);
\draw (x3n) -- (x1n);
\end{tikzpicture}
\end{center}
\caption{A simplex is a generalization of a triangle to higher numbers
  of dimensions, but here the 2-dimensional simplex is illustrated and
  this is just a triangle. A simplex is not regular in general. In the
  description of the downhill simplex algorithm the vertices is
  ordered so $f(\mathbf{x}_1)\le f(\mathbf{x}_2)\le \ldots f(\mathbf{x}_{n+1})$.\label{fig:simplex}}
\end{figure}

\begin{figure}
\begin{center}
\begin{tikzpicture}


\coordinate (a) at (0,14);
\node at (a) {\textbf{(a)}};

\coordinate (b) at (8,14);
\node at (b) {\textbf{(b)}};


\coordinate (x1) at (0,9.33);
\coordinate (x2) at (5,9.33);
\coordinate (x3) at (2.5,13.66);
\coordinate (x0) at (2.5,9.33);
\coordinate (xr) at (2.5,5);

\node[draw=red](x1n) at (x1) {$x_1$};
\node[draw=red](x2n) at (x2) {$x_2$};
\node[draw=red](x3n) at (x3) {$x_3$};
\node[draw=red](x0n) at (x0) {$x_0$};
\node[draw=red](xrn) at (xr) {$x_r$};

\draw (x1n) -- (x0n);
\draw (x0n) -- (x2n);

\draw (xrn) -- (x2n);
\draw (x1n) -- (xrn);

\draw[dashed] (x2n) -- (x3n);
\draw[dashed] (x3n) -- (x1n);


\coordinate (x1r) at (8,9.33);
\coordinate (x2r) at (13,9.33);
\coordinate (x3r) at (10.5,13.66);
\coordinate (x0r) at (10.5,9.33);
\coordinate (xer) at (10.5,0.77);

\node[draw=red](x1nr) at (x1r) {$x_1$};
\node[draw=red](x2nr) at (x2r) {$x_2$};
\node[draw=red](x3nr) at (x3r) {$x_3$};
\node[draw=red](x0nr) at (x0r) {$x_0$};
\node[draw=red](xenr) at (xer) {$x_e$};

\draw (x1nr) -- (x0nr);
\draw (x0nr) -- (x2nr);

\draw (xenr) -- (x2nr);
\draw (x1nr) -- (xenr);

\draw[dashed] (x2nr) -- (x3nr);
\draw[dashed] (x3nr) -- (x1nr);

\end{tikzpicture}
\end{center}

\caption{This illustates reflection \textbf{(a)} and expansion \textbf{(b)}. In reflection the worst point is reflected through $\mathbf{x}_0$, the center of the remaining points to give a new point $\mathbf{x}_r$. Expansion goes from $\mathbf{x}_0$ to $\mathbf{x}_r$ and continues the same distance again in the same direction to give $\mathbf{x}_e$\label{fig:refexp} }
\end{figure}

\begin{figure}
\begin{center}
\begin{tikzpicture}

\coordinate (a) at (0,5);
\node at (a) {\textbf{(a)}};

\coordinate (b) at (8,5);
\node at (b) {\textbf{(b)}};


\coordinate (x1) at (0,0);
\coordinate (x2) at (5,0);
\coordinate (x3) at (2.5,4.33);
\coordinate (x0) at (2.5,0);
\coordinate (xc) at (2.5,2.17);

\node[draw=red](x1n) at (x1) {$x_1$};
\node[draw=red](x2n) at (x2) {$x_2$};
\node[draw=red](x3n) at (x3) {$x_3$};
\node[draw=red](x0n) at (x0) {$x_0$};
\node[draw=red](xcn) at (xc) {$x_c$};

\draw (x1n) -- (x0n);
\draw (x0n) -- (x2n);

\draw (xcn) -- (x2n);
\draw (x1n) -- (xcn);

\draw[dashed] (x2n) -- (x3n);
\draw[dashed] (x3n) -- (x1n);

\coordinate (x1) at (8,0);
\coordinate (x2) at (13,0);
\coordinate (x3) at (10.5,4.33);
\coordinate (x2c) at (10.5,0);
\coordinate (x3c) at (9.25,2.17);

\node[draw=red](x1n) at (x1) {$x_1$};
\node[draw=red](x2n) at (x2) {$x_2$};
\node[draw=red](x3n) at (x3) {$x_3$};
\node[draw=red](x2cn) at (x2c) {$x'_{2}$};
\node[draw=red](x3cn) at (x3c) {$x'_{3}$};

\draw (x1n) -- (x2cn);
\draw (x1n) -- (x3cn);
\draw (x2cn) -- (x3cn);

\draw (x2cn)[dashed] -- (x2n);
\draw (x3cn)[dashed] -- (x3n);
\draw (x2n)[dashed] -- (x3n);


\end{tikzpicture}
\end{center}
\caption{Contraction \textbf{(a)} and reduction \textbf{(b)}. In
  contraction a new point $\mathbf{x}_c$ is chosen which is half way
  between the worst point $\mathbf{x}_{n+1}$ and $\mathbf{x}_0$, the
  centre of the remaining points. In reduction the triangle is shrunk
  to half its size while keeping the best point.\label{fig:conred}}
\end{figure}


% Define block styles
\tikzstyle{decision} = [diamond, draw, 
    text width=4.5em, text badly centered, node distance=3cm, inner sep=0pt]
\tikzstyle{block} = [rectangle, draw,
    text width=5em, text centered, rounded corners, minimum height=4em]
\tikzstyle{line} = [draw, -latex']
\tikzstyle{cloud} = [draw, ellipse, node distance=3cm,
    minimum height=2em] 
   

\begin{figure}
\begin{center}
\begin{tikzpicture}[node distance = 2cm, auto]
    % Place nodes
    \node [block] (order){$\mathbf{x}_1$ $\ldots$ $\mathbf{x}_{n+1}$};
    \node [block, below of= order] (reflect){reflect: $\mathbf{x}_r$};
    \node[decision, below of= reflect] (xr){is $\mathbf{x}_r$ the best point};
    \node[block, right=1cm of xr] (expand){expand: $\mathbf{x}_e$};
    \node[decision, right = 1cm of expand](xe){is $\mathbf{x}_e$ better than $\mathbf{x}_r$};
    \node[cloud,above of=xe](xn12xe){$\mathbf{x}_{n+1}\rightarrow \mathbf{x}_e$};
    \node[cloud,below of=xe](xn12xr){$\mathbf{x}_{n+1}\rightarrow \mathbf{x}_r$};
    \node[decision,below of=xr](xragain){is $\mathbf{x}_r$ the worst};
    \node[block,below =1cm of xragain](contract){contract: $\mathbf{x}_c$};
    \node[decision,right =0.7cm of contract](xc){is $\mathbf{x}_c$ better than $\mathbf{x}_{n+1}$};
    \node[cloud,right =0.7cm of xc](xn12xc){$\mathbf{x}_{n+1}\rightarrow \mathbf{x}_c$};
    \node[cloud,below of= xc](reduce){reduce};

    % Draw edges
    \path [line] (order) -> (reflect);
    \path [line] (reflect) -> (xr);

    \path [line] (expand) -> (xe);
    \path [line] (xragain) -> (contract);
    \path[line] (contract) -> (xc);
    \path [line] (xr) -> node{yes}(expand);    
    \path [line] (xr) -> node{no}(xragain);    
    \path [line] (xe) -> node {yes} (xn12xe);
    \path [line] (xe) -> node {no} (xn12xr);
    \path [line] (xragain) -> node {no} (xn12xr);
    \path [line] (xc) -> node {yes} (xn12xc);
    \path [line] (xc) -> node {no} (reduce);
\end{tikzpicture}
\end{center}
\caption{A flow chart for the downhill simplex algorithm. At the
  start, at the top, the points are put in order so that
  $\mathbf{x}_{n+1}$ is the worst point and $\mathbf{x}_1$ is the
  best. Next the reflected point is calculated to give $\mathbf{x}_r$
  and $f(\mathbf{x}_r)$ is calculated. By comparing to the
  $f(\mathbf{x_i})$ it can be decided if $\mathbf{x}_r$ is the best
  point, that is $f(\mathbf{x}_r)<f(\mathbf{x}_1)$ and the flow chart
  has two branches depending on the answer. This carries out until an
  oval is reached, one or more points are changed and the process
  repeats.\label{fig:flowchart}}
\end{figure}


\end{document}
