%master.tex
%adopted from 5_master.tex; the same notes but with the PandA references
%removed, 

%2015 Conor Houghton conor.houghton@bristol.ac.uk

%To the extent possible under law, the author has dedicated all copyright 
%and related and neighboring rights to these notes to the public domain 
%worldwide. These notes are distributed without any warranty. 


\documentclass[11pt,a4paper]{scrartcl}
\typearea{12}
\usepackage{graphicx}
\usepackage{pstricks}
\usepackage{listings}
\lstset{language=C}
\pagestyle{headings}
\markright{The Master Theorem - Conor}
\begin{document}

\subsection*{The Master Theorem}

This section is about the master theorem; it is based on the
treatment in Introduction to Algorithms by Cormen, Leiserson, Rivest
and Stein which is considered the authority on this subject.

The master method is an approach to solving recurrences of the form
\begin{equation}
T(n)=aT(n/b)+f(n)
\end{equation}
where $a\ge 1$ and $b\ge 1$ and $f(n)$ is some function which is
positive for large enough $n$. This applies to a problem which the
recursion requires the solution of $a$ sub-problems of size $n/b$. In
the case of binary search we had
\begin{equation}
T(n)=T(n/2)+t_0
\end{equation}
The problem requires the solution of just one $n/2$ sized problem and
$a=1$ and $b=2$. The $f(n)$ measures the extra work required to solve
the size $n$ problem when the $a$ size $n/b$ problems have been
solved. In the binary search example the size of this part of the
problem didn't depend on $n$ so $f(n)=t_0$ for some constant $t_0$, or put
another way, $f(n)\in O(1)$.

The master theorem tells us the big-oh complexity of $T(n)$ if the
large $n$ behavior of $f(n)$ satisfies some conditions which depend on
$a$ and $b$. Furthermore, it is a proper theorem, our aim in this
course is to get used to doing rough estimates of complexity without
try for much rigor, but our \textsl{ad hoc} approach is embedded in a
rigorous one and in the proper study of algorithms and, indeed, in
much of crypography, it is important to prove results about complexity
properly. The master theorem is often useful for this.

Before looking at the theorem, we need to note one confusing
aspect. The theorem tells us about the large $n$ behaviour of the $T(n)$ that
satisfies the recursion relation
\begin{equation}
T(n)=aT(n/b)+f(n)
\end{equation}
Usually the recursion relation applies to the worst case run time;
this means the theorem gives us a big-Theta description for the large
$n$ behaviour, but that doesn't mean that it is giving us a big-Theta
description for the algorithm. Typically the recursion relation only
applies to the worst-case senario, for example, for the binary search
algorithm the recursion relation is 
\begin{equation}
T(n)=T(n/2)+t_0
\end{equation}
if the last element examined is the element we are looking for; if the
first element examine is the one we are looking then the algorithm
takes fewer steps and, indeed, the best case, where the first element
we look at is the one we want, is $\Omega(1)$. In this way, if we
apply the master theorem to 
\begin{equation}
T(n)=T(n/2)+t_0
\end{equation}
it will tell us $T(n)\in \Theta(\log{n})$. However this only applies
to $T(n)$s satisfying the recursion relation, the $T(n)$ that is the
run time of the algorithm is $O(\log{n})$.

Now for the master theorem. It has three cases depending on how $f(n)$ behaves
\begin{enumerate}
\item If $f(n) \in O(n^c)$ for $c<\log_ba$ then $T(n) \in \Theta(n^{\log_ba})$
\item If $f(n) \in \Theta(n^c)$ for $c=\log_ba$ then $T(n) \in \Theta(n^c\log n)$
\item If $f(n) \in \Omega(n^c)$ for $c>\log_ba$ then, provided some other conditions on $f(n)$ hold then $T(n)\in \Theta(f(n))$ 
\end{enumerate}
Obviously we haven't stated the third posibility fully and, indeed, we
won't be proving the master theorem. We haven't worried about the fact
that $n/b$ may not be an integer, in practise in divide and conquer
algorithms $n/b$ gets rounded up or down to the nearest integer. We
saw this in case of binary search and, in fact, it doesn't make any
difference to the theorem.

Roughly speaking the large $n$ behaviour of the recursion
\begin{equation}
T(n)=aT(n/b)+f(n)
\end{equation}
is either dominated by the $aT(n/b)$ term or the $f(n)$ term depending
on how fast $f(n)$ grows. This is what gives the first and third
cases; in the first case $f(n)$ grows slowly enough to allow the first
term to dominate the behaviour, in the third case $f(n)$ grows so fast
it takes control. The second case is a kind of in-between case. Since
the first case is about $f(n)$ being growing slower than $n^{\log_ba}$
we specify the behavior of $f(n)$ using big-oh: $f(n)\in O(n^c)$ for
$c<\log_ba$. Conversely, the third case applies to functions that grow
faster than $n^c$ so we specify the behavior of $f(n)$ using
big-Omega. Finally, the middle case applies only to functions that
grow like $n^c$, not faster, not slower, so we specify $f(n)$ using big-Theta.

Now lets consider the binary search case we looked at before. In that case
$a=1$ and $b=2$ so $\log_ba=\log_2 1=0$. We also know $f(n)=1\in \Theta(1)$
so $c=0$. This means the second case applies and the master theorem
says the $T(n)\in O(\log n)$ which is what we worked out.

We will have the opportunity to see the master equation in action
again and it is a standard well-used tool in the study of
algorithms. Here we will just look at some more artificial examples
where we consider some recursion relations without examining any
algorithm they might have come from. 

Now say
\begin{equation}
T(n)=9T(n/3)+n.
\end{equation}
Here $a=9$, $b=3$ and $f(n)=n$, so $f(n)\in O(n)$ and $\log_ba=2>1$. This means the first case applies and
\begin{equation}
T(n)\in \Theta(n^2)
\end{equation}
If, instead
\begin{equation}
T(n)=5T(n/4)+n
\end{equation}
we have $a=5$, $b=4$ and $f(n)=n\in O(n)$. Now $\log_ba=\log_45\approx 1.16>1$ so we are again in the first case and, approximately 
\begin{equation}
T(n)\in \Theta(n^{1.16})
\end{equation}
Finally say we have
\begin{equation}
T(n)=4T(n/2)+n^2
\end{equation}
we have $\log_ba=2$ and $f(n)\in \Theta(n^2)$ so
\begin{equation}
T(n)\in \Theta(n^2\log n)
\end{equation}


\end{document}
