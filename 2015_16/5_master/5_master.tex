%5_master.tex
%notes for the course PandA1 COMS10002 taught at the University of Bristol
%2015 Conor Houghton conor.houghton@bristol.ac.uk

%To the extent possible under law, the author has dedicated all copyright 
%and related and neighboring rights to these notes to the public domain 
%worldwide. These notes are distributed without any warranty. 


\documentclass[11pt,a4paper]{scrartcl}
\typearea{12}
\usepackage{graphicx}
\usepackage{pstricks}
\usepackage{listings}
\lstset{language=C}
\pagestyle{headings}
\markright{COMS10002 - PandA1 5\_master - Conor}
\begin{document}

\subsection*{5 The Master Theorem}

The master method is an approach to solving recurrences of the form
\begin{equation}
T(n)=aT(n/b)+f(n)
\end{equation}
where $a\ge 1$ and $b\ge 1$ and $f(n)$ is some function which is
positive for large enough $n$. This applies to a problem which the
recursion requires the solution of $a$ sub-problems of size $n/b$. In
the case of binary search for example, the problem required the
solution of just one $n/2$ sized problem, so $a=1$ and $b=2$. The
$f(n)$ measures the extra work required to solve the size $n$ problem
when the $a$ size $n/b$ problems have been solved. In the binary
search example the size of this part of the problem didn't depend on
$n$ so $f(n)=c$ for some constant, or put another way, $f(n)\in O(1)$.

The master theorem tells us how the big-oh complexity of $T(n)$ if the large $n$ behavior of $f(n)$ satisfies some conditions which depend on $a$ and $b$. 
\begin{enumerate}
\item If $f(n) \in O(n^c)$ for $c<\log_ba$ then $T(n) \in O(n^{log_ba})$
\item If $f(n) \in \Theta(n^c)$ for $c=\log_ba$ then $T(n) \in O(n^c\log n)$
\end{enumerate}
This isn't quite all, there is one more case we won't consider here
and, in fact, we can replace the big-ohs with big-thetas in the sets
$T(n)$ belong to, but we won't worry about that here.

\end{document}
