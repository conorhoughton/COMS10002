\documentclass[11pt,a4paper]{scrartcl}
\typearea{12}
\usepackage{graphicx}
\usepackage{pstricks}
\usepackage{listings}
\lstset{language=C}
\pagestyle{headings}
\markright{COMS11600 - Principles of programming I.2}
\begin{document}

\subsection*{I.2 Big Oh notation}

The \lq{}Big Oh\rq{} notation has already been used to describe the
behavior of the running time of insert sort, we said
\begin{equation}
T(n)\in O(n^2)
\end{equation}
Here we want to formalize this notation. Basically $O(n^2)$ is a set
of function, it is all the function which, for large values of $n$ go
to infinity like $n^2$ at the fastest. By saying $T(n)\in O(n^2)$ we
are saying that $T(n)$ is one of these functions, its large $n$
behavior is, as worst, like $n^2$. 

Specifically, the definition for $g(n)$ for all $n$
\begin{equation}
O(g(n))=\{f(n)| \exists n_0>0\in {\bf N}\mbox{ and }c>0\in {\bf R}\mbox{ with }|f(n)|\le c|g(n)|\,\forall n\ge n_0\}
\end{equation}
This definition is quite dense, but we can break it down: it says that
$O(g(n))$ is a set of functions, the curly brackets mean
\lq{}set\rq{}. $f(n)$ is in the set if it has a particular
property: the \lq$|$\rq{} can be read as \lq{}such that\rq{} or
\lq{}with the property that\rq{} and so this is the set of $f(n)$'s
where $f(n)$ has the property on the right of the $|$. Now,
\lq{}$\exists$\rq{} means \lq{}there exists\rq{} and
\lq{}$\forall$\rq{} means \lq{}for all\rq{}, so the defining property
says it is possible to find a positive natural number $n_0$ and a
positive real number $c$ for that if you choose a value of $n$ at
least as big as $n_0$ then $f(n)$ is no bigger than $cg(n)$, ${\bf N}$
and ${\bf R}$ stand for the natural and real numbers. Notice the
absolute value signs, this is about $|f(n)|$ and $|g(n)|$, in fact,
here we are interested in run times, so we will deal with functions
that are non-negative, or are non-negative provided $n$ is larger than
some threshold, for example, $\log_2{n}$ will be important,
$\log_2{n}$ is positive provided $n>1$.

In short, $f(n)$ can do all sorts of crazy stuff for small values of
$n$ but, if you take $n$ large enough, its behavior is bounded by the
behavior of $g(n)$. Now it doesn't say it is bounded by $g(n)$, it is
a statement about the behavior, that's the role of the $c$. If you
know the formal definition of limits you can see that the definition
of $O(g(n))$ has this wrapped up in it, if says
\begin{equation}
f(n)\in O(g(n))\iff \lim_{n\rightarrow \infty}\frac{f(n)}{g(n)}<\infty
\end{equation}

Here are some examples, say 
\begin{equation}
T(n)=5n^2+n+6
\end{equation}
then 
\begin{equation}
T(n)\in O(n^2)
\end{equation}
by, for example, taking $c=5+1+6=12$ then 
\begin{equation}
12n^2\ge 5n^2+n+6
\end{equation}
provided $n\ge 1$ so $n_0=1$ here. This could also be succinctly demonstrated using the limit
\begin{equation}
\lim_{n\rightarrow \infty} \frac{5n^2+n+6}{n^2}=\lim_{n\rightarrow \infty}5+\lim_{n\rightarrow \infty}\frac{1}{n}+\lim_{n\rightarrow \infty}\frac{6}{n^2}=5<\infty
\end{equation}

However, 
\begin{equation}
T(n)\not\in O(n)
\end{equation}
Say we chosen some value $c$ then
\begin{equation}
5n^2+n+6>cn
\end{equation}
for large enough $n$, to check this divide both sides by $n$ so we need to show that $n$ can be chosen so that
\begin{equation}
5n+1+\frac{c}{n}>c
\end{equation}
Since 
\begin{equation}
5n+1+\frac{c}{n}>5n+1
\end{equation}
then, if $n>c/5$
\begin{equation}
5n+1+\frac{c}{n}>5\frac{c}{5}+1=c+1>c
\end{equation}
so, no matter what value of $c$ is chosen, making $n>5/n$ implies
\begin{equation}
5n^2+n+6>cn
\end{equation}
so $5n^2+n+6\not\in O(n)$. Again, the limit does the same job
\begin{equation}
\lim_{n\rightarrow \infty} \frac{5n^2+n+6}{n}=\lim_{n\rightarrow \infty}5n+\lim_{n\rightarrow \infty}1+\lim_{n\rightarrow \infty}\frac{6}{n}=\infty
\end{equation}


In practice, if
\begin{equation}
T(n)=a_rn^r+a_{r-1}n^{r-1}+\ldots+a_1n+a_0
\end{equation}
then $T(n)\in O(n^r)$. 

The logarithm has the funny property that it goes to infinity, but it
does so slower than $n$:
\begin{equation}
\lim_{n\rightarrow \infty}\frac{\log_2{n}}{n}=0
\end{equation}
Here, in line with standard practice in computer science, we are using
the log to the base two, in fact changing bases only causes a change
of an overall constant, see Table~\ref{math_logs} for a reminder of
the properties of the log. The limit of $\log_2{n}/n$ can be calculated
using l'H\^{o}pital's rule, here we'll just look at a plot,
Fig.~\ref{fig_log}. Now, just as $\log_2{n}$ grows very slowly, $2^n$ grows very fast, 
\begin{equation}
\lim_{n\rightarrow \infty}\frac{n^r}{2^n}=0
\end{equation}
for any finite value of $r$, worse still is $n!$, pronounced $n$-factorial
\begin{equation}
n!=n(n-1)(n-2) . . . 1
\end{equation}
If you algorithm is in $O(n!)$ you will probably need a different
algorithm. A table of different values is given as
Table~\ref{table_n_values}, mostly to emphasis how quickly $n!$ gets
big.

Now, in mathematics we call something \lq{}an abuse of notation\rq{}
if it is common to write something that doesn't quite make sense but
acts as a shorthand for something that does. Now $O(g(n))$ is a set of
functions whose large $n$ behavior is bounded by $g(n)$ so in
algorithms, being in $O(g(n))$ is a property of $T(n)$, the formula
for the running time of the algorithm. However, it is a standard abuse
of notation to say an algorithm is $O(g(n))$ for some $g(n)$ is
$T(n)\in O(g(n))$ for all cases. This is another way of saying that
the worst behavior of the algorithm isn't any worse than the behavior
of $g(n)$ for large $n$ so all possible $T(n)$ are elements of
$O(g(n))$.

\begin{table}
The logarithm is the opposite of the exponent: if
\begin{equation}
a^b=c
\end{equation}
then 
\begin{equation}
\log_a{c}=b
\end{equation}
or, written in one line
\begin{equation}
a^{\log_a{c}}=c
\end{equation}
All the laws of logs can be worked out from the laws of
exponents. Hence, since $a^0=1$ we have $\log_a{1}=0$. In a similar
way, other rules of logs can be deduced like
\begin{eqnarray}
\log_a{c_1c_2}&=&\log_a{c_1}+\log_a{c_2}\cr
\log_a{\frac{c_1}{c_2}}&=&\log_a{c_1}-\log_a{c_2}\cr
\log_a c^d&=&d\log_a c
\end{eqnarray}
and so on.

As for the change of base, let $b=\log_{a_1}{c}$ so $a_1^{b}=c$. Now take the log to the base $a_2$ of both sides
\begin{equation}
b\log_{a_2}a_1=\log_{a_2}{c}
\end{equation}
and then solve for $b$
\begin{equation}
b=\frac{\log_{a_2}c}{\log_{a_2}a_1}
\end{equation}
and, substituting back the formula for $b$
\begin{equation}
\log_{a_1}{c}=\frac{\log_{a_2}c}{\log_{a_2}a_1}
\end{equation}
Thus we see, that changing bases is just a matter of a
multiplicative factor. For example, to change from base $e$ to base two
\begin{equation}
\log_{2}{x}=\frac{\log_e{x}}{\log_e{2}}\approx\frac{\log_e{x}}{0.6931}
\end{equation}
Common bases are $\log_2{x}$ used in computer science, $\log_e{x}$
sometimes written $\ln{x}$ used in mathematics and $\log_{10}{x}$ used
in chemistry. The base two is used because of its link to bits and
also, as we will see, because of its relationship with algorithms that
divide data into two piles. The natural log $\ln{x}$ is used where
differential equations are common since
\begin{equation}
\frac{d}{dx}\ln{x}=\frac{1}{x}
\end{equation}
\caption{A reminder about logarithms. This is a quick summary of some of the laws of logs.\label{math_logs}}
\end{table}

\begin{figure}
\documentclass[11pt,a4paper]{scrartcl}
\typearea{12}
\usepackage{graphicx}
\usepackage{pstricks}
\usepackage{listings}
\lstset{language=C}
\pagestyle{headings}
\markright{COMS11600 - Principles of programming I.2}
\begin{document}

\subsection*{I.2 Big Oh notation}

The \lq{}Big Oh\rq{} notation has already been used to describe the
behavior of the running time of insert sort, we said
\begin{equation}
T(n)\in O(n^2)
\end{equation}
Here we want to formalize this notation. Basically $O(n^2)$ is a set
of function, it is all the function which, for large values of $n$ go
to infinity like $n^2$ at the fastest. By saying $T(n)\in O(n^2)$ we
are saying that $T(n)$ is one of these functions, its large $n$
behavior is, as worst, like $n^2$. 

Specifically, the definition for $g(n)$ for all $n$
\begin{equation}
O(g(n))=\{f(n)| \exists n_0>0\in {\bf N}\mbox{ and }c>0\in {\bf R}\mbox{ with }|f(n)|\le c|g(n)|\,\forall n\ge n_0\}
\end{equation}
This definition is quite dense, but we can break it down: it says that
$O(g(n))$ is a set of functions, the curly brackets mean
\lq{}set\rq{}. $f(n)$ is in the set if it has a particular
property: the \lq$|$\rq{} can be read as \lq{}such that\rq{} or
\lq{}with the property that\rq{} and so this is the set of $f(n)$'s
where $f(n)$ has the property on the right of the $|$. Now,
\lq{}$\exists$\rq{} means \lq{}there exists\rq{} and
\lq{}$\forall$\rq{} means \lq{}for all\rq{}, so the defining property
says it is possible to find a positive natural number $n_0$ and a
positive real number $c$ for that if you choose a value of $n$ at
least as big as $n_0$ then $f(n)$ is no bigger than $cg(n)$, ${\bf N}$
and ${\bf R}$ stand for the natural and real numbers. Notice the
absolute value signs, this is about $|f(n)|$ and $|g(n)|$, in fact,
here we are interested in run times, so we will deal with functions
that are non-negative, or are non-negative provided $n$ is larger than
some threshold, for example, $\log_2{n}$ will be important,
$\log_2{n}$ is positive provided $n>1$.

In short, $f(n)$ can do all sorts of crazy stuff for small values of
$n$ but, if you take $n$ large enough, its behavior is bounded by the
behavior of $g(n)$. Now it doesn't say it is bounded by $g(n)$, it is
a statement about the behavior, that's the role of the $c$. If you
know the formal definition of limits you can see that the definition
of $O(g(n))$ has this wrapped up in it, if says
\begin{equation}
f(n)\in O(g(n))\iff \lim_{n\rightarrow \infty}\frac{f(n)}{g(n)}<\infty
\end{equation}

Here are some examples, say 
\begin{equation}
T(n)=5n^2+n+6
\end{equation}
then 
\begin{equation}
T(n)\in O(n^2)
\end{equation}
by, for example, taking $c=5+1+6=12$ then 
\begin{equation}
12n^2\ge 5n^2+n+6
\end{equation}
provided $n\ge 1$ so $n_0=1$ here. This could also be succinctly demonstrated using the limit
\begin{equation}
\lim_{n\rightarrow \infty} \frac{5n^2+n+6}{n^2}=\lim_{n\rightarrow \infty}5+\lim_{n\rightarrow \infty}\frac{1}{n}+\lim_{n\rightarrow \infty}\frac{6}{n^2}=5<\infty
\end{equation}

However, 
\begin{equation}
T(n)\not\in O(n)
\end{equation}
Say we chosen some value $c$ then
\begin{equation}
5n^2+n+6>cn
\end{equation}
for large enough $n$, to check this divide both sides by $n$ so we need to show that $n$ can be chosen so that
\begin{equation}
5n+1+\frac{c}{n}>c
\end{equation}
Since 
\begin{equation}
5n+1+\frac{c}{n}>5n+1
\end{equation}
then, if $n>c/5$
\begin{equation}
5n+1+\frac{c}{n}>5\frac{c}{5}+1=c+1>c
\end{equation}
so, no matter what value of $c$ is chosen, making $n>5/n$ implies
\begin{equation}
5n^2+n+6>cn
\end{equation}
so $5n^2+n+6\not\in O(n)$. Again, the limit does the same job
\begin{equation}
\lim_{n\rightarrow \infty} \frac{5n^2+n+6}{n}=\lim_{n\rightarrow \infty}5n+\lim_{n\rightarrow \infty}1+\lim_{n\rightarrow \infty}\frac{6}{n}=\infty
\end{equation}


In practice, if
\begin{equation}
T(n)=a_rn^r+a_{r-1}n^{r-1}+\ldots+a_1n+a_0
\end{equation}
then $T(n)\in O(n^r)$. 

The logarithm has the funny property that it goes to infinity, but it
does so slower than $n$:
\begin{equation}
\lim_{n\rightarrow \infty}\frac{\log_2{n}}{n}=0
\end{equation}
Here, in line with standard practice in computer science, we are using
the log to the base two, in fact changing bases only causes a change
of an overall constant, see Table~\ref{math_logs} for a reminder of
the properties of the log. The limit of $\log_2{n}/n$ can be calculated
using l'H\^{o}pital's rule, here we'll just look at a plot,
Fig.~\ref{fig_log}. Now, just as $\log_2{n}$ grows very slowly, $2^n$ grows very fast, 
\begin{equation}
\lim_{n\rightarrow \infty}\frac{n^r}{2^n}=0
\end{equation}
for any finite value of $r$, worse still is $n!$, pronounced $n$-factorial
\begin{equation}
n!=n(n-1)(n-2) . . . 1
\end{equation}
If you algorithm is in $O(n!)$ you will probably need a different
algorithm. A table of different values is given as
Table~\ref{table_n_values}, mostly to emphasis how quickly $n!$ gets
big.

Now, in mathematics we call something \lq{}an abuse of notation\rq{}
if it is common to write something that doesn't quite make sense but
acts as a shorthand for something that does. Now $O(g(n))$ is a set of
functions whose large $n$ behavior is bounded by $g(n)$ so in
algorithms, being in $O(g(n))$ is a property of $T(n)$, the formula
for the running time of the algorithm. However, it is a standard abuse
of notation to say an algorithm is $O(g(n))$ for some $g(n)$ is
$T(n)\in O(g(n))$ for all cases. This is another way of saying that
the worst behavior of the algorithm isn't any worse than the behavior
of $g(n)$ for large $n$ so all possible $T(n)$ are elements of
$O(g(n))$.

\begin{table}
The logarithm is the opposite of the exponent: if
\begin{equation}
a^b=c
\end{equation}
then 
\begin{equation}
\log_a{c}=b
\end{equation}
or, written in one line
\begin{equation}
a^{\log_a{c}}=c
\end{equation}
All the laws of logs can be worked out from the laws of
exponents. Hence, since $a^0=1$ we have $\log_a{1}=0$. In a similar
way, other rules of logs can be deduced like
\begin{eqnarray}
\log_a{c_1c_2}&=&\log_a{c_1}+\log_a{c_2}\cr
\log_a{\frac{c_1}{c_2}}&=&\log_a{c_1}-\log_a{c_2}\cr
\log_a c^d&=&d\log_a c
\end{eqnarray}
and so on.

As for the change of base, let $b=\log_{a_1}{c}$ so $a_1^{b}=c$. Now take the log to the base $a_2$ of both sides
\begin{equation}
b\log_{a_2}a_1=\log_{a_2}{c}
\end{equation}
and then solve for $b$
\begin{equation}
b=\frac{\log_{a_2}c}{\log_{a_2}a_1}
\end{equation}
and, substituting back the formula for $b$
\begin{equation}
\log_{a_1}{c}=\frac{\log_{a_2}c}{\log_{a_2}a_1}
\end{equation}
Thus we see, that changing bases is just a matter of a
multiplicative factor. For example, to change from base $e$ to base two
\begin{equation}
\log_{2}{x}=\frac{\log_e{x}}{\log_e{2}}\approx\frac{\log_e{x}}{0.6931}
\end{equation}
Common bases are $\log_2{x}$ used in computer science, $\log_e{x}$
sometimes written $\ln{x}$ used in mathematics and $\log_{10}{x}$ used
in chemistry. The base two is used because of its link to bits and
also, as we will see, because of its relationship with algorithms that
divide data into two piles. The natural log $\ln{x}$ is used where
differential equations are common since
\begin{equation}
\frac{d}{dx}\ln{x}=\frac{1}{x}
\end{equation}
\caption{A reminder about logarithms. This is a quick summary of some of the laws of logs.\label{math_logs}}
\end{table}

\begin{figure}
\documentclass[11pt,a4paper]{scrartcl}
\typearea{12}
\usepackage{graphicx}
\usepackage{pstricks}
\usepackage{listings}
\lstset{language=C}
\pagestyle{headings}
\markright{COMS11600 - Principles of programming I.2}
\begin{document}

\subsection*{I.2 Big Oh notation}

The \lq{}Big Oh\rq{} notation has already been used to describe the
behavior of the running time of insert sort, we said
\begin{equation}
T(n)\in O(n^2)
\end{equation}
Here we want to formalize this notation. Basically $O(n^2)$ is a set
of function, it is all the function which, for large values of $n$ go
to infinity like $n^2$ at the fastest. By saying $T(n)\in O(n^2)$ we
are saying that $T(n)$ is one of these functions, its large $n$
behavior is, as worst, like $n^2$. 

Specifically, the definition for $g(n)$ for all $n$
\begin{equation}
O(g(n))=\{f(n)| \exists n_0>0\in {\bf N}\mbox{ and }c>0\in {\bf R}\mbox{ with }|f(n)|\le c|g(n)|\,\forall n\ge n_0\}
\end{equation}
This definition is quite dense, but we can break it down: it says that
$O(g(n))$ is a set of functions, the curly brackets mean
\lq{}set\rq{}. $f(n)$ is in the set if it has a particular
property: the \lq$|$\rq{} can be read as \lq{}such that\rq{} or
\lq{}with the property that\rq{} and so this is the set of $f(n)$'s
where $f(n)$ has the property on the right of the $|$. Now,
\lq{}$\exists$\rq{} means \lq{}there exists\rq{} and
\lq{}$\forall$\rq{} means \lq{}for all\rq{}, so the defining property
says it is possible to find a positive natural number $n_0$ and a
positive real number $c$ for that if you choose a value of $n$ at
least as big as $n_0$ then $f(n)$ is no bigger than $cg(n)$, ${\bf N}$
and ${\bf R}$ stand for the natural and real numbers. Notice the
absolute value signs, this is about $|f(n)|$ and $|g(n)|$, in fact,
here we are interested in run times, so we will deal with functions
that are non-negative, or are non-negative provided $n$ is larger than
some threshold, for example, $\log_2{n}$ will be important,
$\log_2{n}$ is positive provided $n>1$.

In short, $f(n)$ can do all sorts of crazy stuff for small values of
$n$ but, if you take $n$ large enough, its behavior is bounded by the
behavior of $g(n)$. Now it doesn't say it is bounded by $g(n)$, it is
a statement about the behavior, that's the role of the $c$. If you
know the formal definition of limits you can see that the definition
of $O(g(n))$ has this wrapped up in it, if says
\begin{equation}
f(n)\in O(g(n))\iff \lim_{n\rightarrow \infty}\frac{f(n)}{g(n)}<\infty
\end{equation}

Here are some examples, say 
\begin{equation}
T(n)=5n^2+n+6
\end{equation}
then 
\begin{equation}
T(n)\in O(n^2)
\end{equation}
by, for example, taking $c=5+1+6=12$ then 
\begin{equation}
12n^2\ge 5n^2+n+6
\end{equation}
provided $n\ge 1$ so $n_0=1$ here. This could also be succinctly demonstrated using the limit
\begin{equation}
\lim_{n\rightarrow \infty} \frac{5n^2+n+6}{n^2}=\lim_{n\rightarrow \infty}5+\lim_{n\rightarrow \infty}\frac{1}{n}+\lim_{n\rightarrow \infty}\frac{6}{n^2}=5<\infty
\end{equation}

However, 
\begin{equation}
T(n)\not\in O(n)
\end{equation}
Say we chosen some value $c$ then
\begin{equation}
5n^2+n+6>cn
\end{equation}
for large enough $n$, to check this divide both sides by $n$ so we need to show that $n$ can be chosen so that
\begin{equation}
5n+1+\frac{c}{n}>c
\end{equation}
Since 
\begin{equation}
5n+1+\frac{c}{n}>5n+1
\end{equation}
then, if $n>c/5$
\begin{equation}
5n+1+\frac{c}{n}>5\frac{c}{5}+1=c+1>c
\end{equation}
so, no matter what value of $c$ is chosen, making $n>5/n$ implies
\begin{equation}
5n^2+n+6>cn
\end{equation}
so $5n^2+n+6\not\in O(n)$. Again, the limit does the same job
\begin{equation}
\lim_{n\rightarrow \infty} \frac{5n^2+n+6}{n}=\lim_{n\rightarrow \infty}5n+\lim_{n\rightarrow \infty}1+\lim_{n\rightarrow \infty}\frac{6}{n}=\infty
\end{equation}


In practice, if
\begin{equation}
T(n)=a_rn^r+a_{r-1}n^{r-1}+\ldots+a_1n+a_0
\end{equation}
then $T(n)\in O(n^r)$. 

The logarithm has the funny property that it goes to infinity, but it
does so slower than $n$:
\begin{equation}
\lim_{n\rightarrow \infty}\frac{\log_2{n}}{n}=0
\end{equation}
Here, in line with standard practice in computer science, we are using
the log to the base two, in fact changing bases only causes a change
of an overall constant, see Table~\ref{math_logs} for a reminder of
the properties of the log. The limit of $\log_2{n}/n$ can be calculated
using l'H\^{o}pital's rule, here we'll just look at a plot,
Fig.~\ref{fig_log}. Now, just as $\log_2{n}$ grows very slowly, $2^n$ grows very fast, 
\begin{equation}
\lim_{n\rightarrow \infty}\frac{n^r}{2^n}=0
\end{equation}
for any finite value of $r$, worse still is $n!$, pronounced $n$-factorial
\begin{equation}
n!=n(n-1)(n-2) . . . 1
\end{equation}
If you algorithm is in $O(n!)$ you will probably need a different
algorithm. A table of different values is given as
Table~\ref{table_n_values}, mostly to emphasis how quickly $n!$ gets
big.

Now, in mathematics we call something \lq{}an abuse of notation\rq{}
if it is common to write something that doesn't quite make sense but
acts as a shorthand for something that does. Now $O(g(n))$ is a set of
functions whose large $n$ behavior is bounded by $g(n)$ so in
algorithms, being in $O(g(n))$ is a property of $T(n)$, the formula
for the running time of the algorithm. However, it is a standard abuse
of notation to say an algorithm is $O(g(n))$ for some $g(n)$ is
$T(n)\in O(g(n))$ for all cases. This is another way of saying that
the worst behavior of the algorithm isn't any worse than the behavior
of $g(n)$ for large $n$ so all possible $T(n)$ are elements of
$O(g(n))$.

\begin{table}
The logarithm is the opposite of the exponent: if
\begin{equation}
a^b=c
\end{equation}
then 
\begin{equation}
\log_a{c}=b
\end{equation}
or, written in one line
\begin{equation}
a^{\log_a{c}}=c
\end{equation}
All the laws of logs can be worked out from the laws of
exponents. Hence, since $a^0=1$ we have $\log_a{1}=0$. In a similar
way, other rules of logs can be deduced like
\begin{eqnarray}
\log_a{c_1c_2}&=&\log_a{c_1}+\log_a{c_2}\cr
\log_a{\frac{c_1}{c_2}}&=&\log_a{c_1}-\log_a{c_2}\cr
\log_a c^d&=&d\log_a c
\end{eqnarray}
and so on.

As for the change of base, let $b=\log_{a_1}{c}$ so $a_1^{b}=c$. Now take the log to the base $a_2$ of both sides
\begin{equation}
b\log_{a_2}a_1=\log_{a_2}{c}
\end{equation}
and then solve for $b$
\begin{equation}
b=\frac{\log_{a_2}c}{\log_{a_2}a_1}
\end{equation}
and, substituting back the formula for $b$
\begin{equation}
\log_{a_1}{c}=\frac{\log_{a_2}c}{\log_{a_2}a_1}
\end{equation}
Thus we see, that changing bases is just a matter of a
multiplicative factor. For example, to change from base $e$ to base two
\begin{equation}
\log_{2}{x}=\frac{\log_e{x}}{\log_e{2}}\approx\frac{\log_e{x}}{0.6931}
\end{equation}
Common bases are $\log_2{x}$ used in computer science, $\log_e{x}$
sometimes written $\ln{x}$ used in mathematics and $\log_{10}{x}$ used
in chemistry. The base two is used because of its link to bits and
also, as we will see, because of its relationship with algorithms that
divide data into two piles. The natural log $\ln{x}$ is used where
differential equations are common since
\begin{equation}
\frac{d}{dx}\ln{x}=\frac{1}{x}
\end{equation}
\caption{A reminder about logarithms. This is a quick summary of some of the laws of logs.\label{math_logs}}
\end{table}

\begin{figure}
\documentclass[11pt,a4paper]{scrartcl}
\typearea{12}
\usepackage{graphicx}
\usepackage{pstricks}
\usepackage{listings}
\lstset{language=C}
\pagestyle{headings}
\markright{COMS11600 - Principles of programming I.2}
\begin{document}

\subsection*{I.2 Big Oh notation}

The \lq{}Big Oh\rq{} notation has already been used to describe the
behavior of the running time of insert sort, we said
\begin{equation}
T(n)\in O(n^2)
\end{equation}
Here we want to formalize this notation. Basically $O(n^2)$ is a set
of function, it is all the function which, for large values of $n$ go
to infinity like $n^2$ at the fastest. By saying $T(n)\in O(n^2)$ we
are saying that $T(n)$ is one of these functions, its large $n$
behavior is, as worst, like $n^2$. 

Specifically, the definition for $g(n)$ for all $n$
\begin{equation}
O(g(n))=\{f(n)| \exists n_0>0\in {\bf N}\mbox{ and }c>0\in {\bf R}\mbox{ with }|f(n)|\le c|g(n)|\,\forall n\ge n_0\}
\end{equation}
This definition is quite dense, but we can break it down: it says that
$O(g(n))$ is a set of functions, the curly brackets mean
\lq{}set\rq{}. $f(n)$ is in the set if it has a particular
property: the \lq$|$\rq{} can be read as \lq{}such that\rq{} or
\lq{}with the property that\rq{} and so this is the set of $f(n)$'s
where $f(n)$ has the property on the right of the $|$. Now,
\lq{}$\exists$\rq{} means \lq{}there exists\rq{} and
\lq{}$\forall$\rq{} means \lq{}for all\rq{}, so the defining property
says it is possible to find a positive natural number $n_0$ and a
positive real number $c$ for that if you choose a value of $n$ at
least as big as $n_0$ then $f(n)$ is no bigger than $cg(n)$, ${\bf N}$
and ${\bf R}$ stand for the natural and real numbers. Notice the
absolute value signs, this is about $|f(n)|$ and $|g(n)|$, in fact,
here we are interested in run times, so we will deal with functions
that are non-negative, or are non-negative provided $n$ is larger than
some threshold, for example, $\log_2{n}$ will be important,
$\log_2{n}$ is positive provided $n>1$.

In short, $f(n)$ can do all sorts of crazy stuff for small values of
$n$ but, if you take $n$ large enough, its behavior is bounded by the
behavior of $g(n)$. Now it doesn't say it is bounded by $g(n)$, it is
a statement about the behavior, that's the role of the $c$. If you
know the formal definition of limits you can see that the definition
of $O(g(n))$ has this wrapped up in it, if says
\begin{equation}
f(n)\in O(g(n))\iff \lim_{n\rightarrow \infty}\frac{f(n)}{g(n)}<\infty
\end{equation}

Here are some examples, say 
\begin{equation}
T(n)=5n^2+n+6
\end{equation}
then 
\begin{equation}
T(n)\in O(n^2)
\end{equation}
by, for example, taking $c=5+1+6=12$ then 
\begin{equation}
12n^2\ge 5n^2+n+6
\end{equation}
provided $n\ge 1$ so $n_0=1$ here. This could also be succinctly demonstrated using the limit
\begin{equation}
\lim_{n\rightarrow \infty} \frac{5n^2+n+6}{n^2}=\lim_{n\rightarrow \infty}5+\lim_{n\rightarrow \infty}\frac{1}{n}+\lim_{n\rightarrow \infty}\frac{6}{n^2}=5<\infty
\end{equation}

However, 
\begin{equation}
T(n)\not\in O(n)
\end{equation}
Say we chosen some value $c$ then
\begin{equation}
5n^2+n+6>cn
\end{equation}
for large enough $n$, to check this divide both sides by $n$ so we need to show that $n$ can be chosen so that
\begin{equation}
5n+1+\frac{c}{n}>c
\end{equation}
Since 
\begin{equation}
5n+1+\frac{c}{n}>5n+1
\end{equation}
then, if $n>c/5$
\begin{equation}
5n+1+\frac{c}{n}>5\frac{c}{5}+1=c+1>c
\end{equation}
so, no matter what value of $c$ is chosen, making $n>5/n$ implies
\begin{equation}
5n^2+n+6>cn
\end{equation}
so $5n^2+n+6\not\in O(n)$. Again, the limit does the same job
\begin{equation}
\lim_{n\rightarrow \infty} \frac{5n^2+n+6}{n}=\lim_{n\rightarrow \infty}5n+\lim_{n\rightarrow \infty}1+\lim_{n\rightarrow \infty}\frac{6}{n}=\infty
\end{equation}


In practice, if
\begin{equation}
T(n)=a_rn^r+a_{r-1}n^{r-1}+\ldots+a_1n+a_0
\end{equation}
then $T(n)\in O(n^r)$. 

The logarithm has the funny property that it goes to infinity, but it
does so slower than $n$:
\begin{equation}
\lim_{n\rightarrow \infty}\frac{\log_2{n}}{n}=0
\end{equation}
Here, in line with standard practice in computer science, we are using
the log to the base two, in fact changing bases only causes a change
of an overall constant, see Table~\ref{math_logs} for a reminder of
the properties of the log. The limit of $\log_2{n}/n$ can be calculated
using l'H\^{o}pital's rule, here we'll just look at a plot,
Fig.~\ref{fig_log}. Now, just as $\log_2{n}$ grows very slowly, $2^n$ grows very fast, 
\begin{equation}
\lim_{n\rightarrow \infty}\frac{n^r}{2^n}=0
\end{equation}
for any finite value of $r$, worse still is $n!$, pronounced $n$-factorial
\begin{equation}
n!=n(n-1)(n-2) . . . 1
\end{equation}
If you algorithm is in $O(n!)$ you will probably need a different
algorithm. A table of different values is given as
Table~\ref{table_n_values}, mostly to emphasis how quickly $n!$ gets
big.

Now, in mathematics we call something \lq{}an abuse of notation\rq{}
if it is common to write something that doesn't quite make sense but
acts as a shorthand for something that does. Now $O(g(n))$ is a set of
functions whose large $n$ behavior is bounded by $g(n)$ so in
algorithms, being in $O(g(n))$ is a property of $T(n)$, the formula
for the running time of the algorithm. However, it is a standard abuse
of notation to say an algorithm is $O(g(n))$ for some $g(n)$ is
$T(n)\in O(g(n))$ for all cases. This is another way of saying that
the worst behavior of the algorithm isn't any worse than the behavior
of $g(n)$ for large $n$ so all possible $T(n)$ are elements of
$O(g(n))$.

\begin{table}
The logarithm is the opposite of the exponent: if
\begin{equation}
a^b=c
\end{equation}
then 
\begin{equation}
\log_a{c}=b
\end{equation}
or, written in one line
\begin{equation}
a^{\log_a{c}}=c
\end{equation}
All the laws of logs can be worked out from the laws of
exponents. Hence, since $a^0=1$ we have $\log_a{1}=0$. In a similar
way, other rules of logs can be deduced like
\begin{eqnarray}
\log_a{c_1c_2}&=&\log_a{c_1}+\log_a{c_2}\cr
\log_a{\frac{c_1}{c_2}}&=&\log_a{c_1}-\log_a{c_2}\cr
\log_a c^d&=&d\log_a c
\end{eqnarray}
and so on.

As for the change of base, let $b=\log_{a_1}{c}$ so $a_1^{b}=c$. Now take the log to the base $a_2$ of both sides
\begin{equation}
b\log_{a_2}a_1=\log_{a_2}{c}
\end{equation}
and then solve for $b$
\begin{equation}
b=\frac{\log_{a_2}c}{\log_{a_2}a_1}
\end{equation}
and, substituting back the formula for $b$
\begin{equation}
\log_{a_1}{c}=\frac{\log_{a_2}c}{\log_{a_2}a_1}
\end{equation}
Thus we see, that changing bases is just a matter of a
multiplicative factor. For example, to change from base $e$ to base two
\begin{equation}
\log_{2}{x}=\frac{\log_e{x}}{\log_e{2}}\approx\frac{\log_e{x}}{0.6931}
\end{equation}
Common bases are $\log_2{x}$ used in computer science, $\log_e{x}$
sometimes written $\ln{x}$ used in mathematics and $\log_{10}{x}$ used
in chemistry. The base two is used because of its link to bits and
also, as we will see, because of its relationship with algorithms that
divide data into two piles. The natural log $\ln{x}$ is used where
differential equations are common since
\begin{equation}
\frac{d}{dx}\ln{x}=\frac{1}{x}
\end{equation}
\caption{A reminder about logarithms. This is a quick summary of some of the laws of logs.\label{math_logs}}
\end{table}

\begin{figure}
\include{I.2.log}
\caption{This shows $\log_2{x}$ and $x-1$ plots for $x\in[1,4]$, the one has been taken from $x$ to make them easier to compare, the key point is that the $x$ grows faster.\label{fig_log}}
\end{figure}


\begin{figure}
\include{I.2.exp}
\caption{This shows $2^x$ and $x^2-1$ plots for $x\in[0,6]$, clearly $2^x$ quickly overtakes $x^2+1$, this will happen for any power of $x$. \label{fig_log}}
\end{figure}

\begin{table}
\begin{tabular}{l|cccccc}
        $n$    &1   &2&4   &16  &128&1024\\
$\log{n}$      &0   &1&2   &4   &7  &10\\
$n\log{n}$     &0   &2&8   &64  &896&10240\\
$n^2$     &1   &4&16&256&16384&1048576\\
$2^n$     &2   &4&16&65536&$3.4\times 10^{38}$&$1.8\times 10^{307}$\\
$n!$      &1   &2&24&$2.1\times 10^{13}$&$3.85\times 10^{305}$&$5.4\times10^{2369}$
\end{tabular}
\vskip 1cm The website\\ {\tt
  http://markknowsnothing.com/cgi-bin/calculator.php}\\ was used for
the $2^n$ calculations and\\ {\tt
  http://www.calculatorsoup.com/calculators/discretemathematics/factorials.php}\\ for
the $n!$ calculations; these give answers even when the answer is very
large.

\caption{Different values of $n$ for some functions.  \label{table_n_values}
}
\end{table}

\subsubsection*{I.2.1 Examples: linear and binary search}

Lets do a quick example; searching for the index of an element in a
sorted list. A completely terrible way to do this is linear search,
this is terrible because it doesn't make use of the fact that the list
is sorted. A code listing is given in Table~\ref{c_linear_search}. We
can see straight away that the code between lines 6 and 10 is run $n$
times in the worst case, everything else is run once and so this
algorithm is $O(n)$.

\begin{table}
\begin{lstlisting}[numbers=left]
int search(int a[],int n, int val)
{

  int i;

  for(i=0;i<n;i++)
    {
      if(a[i]==val)
	return i;
    }

  return -1;
}
\end{lstlisting}
\caption{Linear search. This function searches the entries in the array a and returns the index when it finds val, if it doesn't find val it returns -1. The program {\tt linear\_search.c} implements this.\label{c_linear_search}.}
\end{table}

A much better way to search a sorted array is binary search. This is
an example of a \lq{}divide and conquer\rq{} algorithm, many of the
fastest algorithms use divide and conquer. It will be clear to you
that this algorithm would be better written using recursion, this is
typical of divide and conquer, but we haven't looked at analysing
recursion yet. The idea is to divide the array in half and check which
half, the half with bigger numbers or the half with smaller numbers,
the value we are searching for belongs to and to keep doing this,
dividing the remaining part of the array into two parts again and again
until the remaining part of the array that is being search has only
one element. A code listing is given in Table~\ref{c_binary_search}.

\begin{table}
\begin{lstlisting}[numbers=left] 
int search(int a[],int n, int val)
{
  int mid, low=0, high=n-1;

  while(low<=high)
    {
      mid=(low+high)/2;
      if(a[mid]==val)
	return mid;
      else if(val>a[mid])
	low=mid+1;
      else
	high=mid -1;
    }

  return -1;
}
\end{lstlisting}
\caption{Binary search. This function starts in the middle of the array
  and checks if the value there is bigger or smaller than val, if it
  is bigger then it does the same in the top half of the array, if it
  is smaller, in the bottom half and then repeats until there are no
  elements left. The program {\tt binary\_search.c} implements
  this.\label{c_binary_search}.}
\end{table}

This search is extremely fast. There is a chance that a[mid]==val,
after a small number of iterations, indeed, if the middle value of the
array is the search value it will halt after only one
iteration. However, as usual, we assume the worst case, in which case
the algorithm runs to end, dividing the number of elements in half
each time. Ignoring the integer rounding effects, it goes like
Table~\ref{tab_binary_search}. Starting with $n$ states each
subsequent iteration halves the number of states until the last one
when there is one state left. Thus
\begin{equation}
1=\frac{n}{2^{T(n)-1}}
\end{equation}
and taking the log of both sides
\begin{equation}
0=\log_2{n}-(T(n)-1)
\end{equation}
using $\log_2{1}=1$, $\log_2{a^b}=b\log_2{a}$ and $\log_2{2}=1$. Hence
\begin{equation}
T(n)=\log_2{n}+1
\end{equation}
and this algorithm is $O(\log_2{n})$.

\begin{table}
\begin{center}
\begin{tabular}{llllllll}
1&2&3&4&\ldots&k&\ldots&T(n)\\
\hline
$n$&$\frac{n}{2}$&$\frac{n}{4}$&$\frac{n}{8}$&$\ldots$&$\frac{n}{2^{k-1}}$&$\ldots$&1
\end{tabular}
\end{center}
\caption{The number of elements left for binary search.\label{tab_binary_search}}
\end{table}

So, to reiterate; the usual way to examine the behavior of an
algorithm is to look at the worst case run time. This is because the
best case run time is often exceptional, like the one for binary
search if the first guess happens to be correct. The average run time
is often hard to calculate, both because it is often difficult to do
the mathematics and because it would often mean having some
description of how the initial data is distributed. Typically the
worst run time is also \lq{}of the same order\rq{} as the average run
time. We will see an exception to this later on in the case of quick
sort in which the worst case behavior is unusual. The big-Oh notation
is used for describing an algorithm, if the algorithm is said to be
$O(g(n))$ we mean $T(n)\in O(g(n))$ no matter what the initial
condition. Since $O(g(n))$ involves an upper bound $T(n)<cg(n)$ this
makes sense.


\subsubsection*{Other big Letter notations, small oh notation.}

There is another set, $\Omega(g(n))$ with a definition similar to
$O(g(n))$ that is used for describing the best case behavior. This
requires a lower bound rather than an upper bound, so the obvious
definition is
\begin{equation}
\Omega(g(n))=\{f(n)| \exists n_0>0\in {\bf N}\mbox{ and }c>0\in {\bf R}\mbox{ with }|f(n)|\ge c|g(n)|\,\forall n\ge n_0\}
\end{equation}
in other words, the same thing, but with the $\le$ symbol replaced by
a $\ge$. In fact, there is some ambiguity about this definition,
number theorist use a slightly different one. Either way, it isn't
used very often in computer science because algorithms are very
frequently $\Omega(1)$; in the best case scenario the problem is in
some sense already solve, the array already sorted for example, and
the algorithm finishs in one step. 

There is also a set of function that are both bounded above and below
by the same $g(n)$
\begin{equation}
\Theta(g(n))=\Omega(g(n))\cap O(g(n))
\end{equation}
This works because it is possible for
\begin{equation}
c_1 g(n)\le f(n)\le c_2g(n)
\end{equation}
for different $c_1$ and $c_2$. It would be very unusual for this to
apply to an algorithm, it would mean that $T(n)$ has the same behavior
for large $n$ no matter whether it is the best case or the worst case
scenario. There is a na\"ive largest element function in
Table~\ref{c_largest_linear_search} which is $\Theta(n)$. It searches
for the largest value in an unsorted array by looking at each element
in turn. In fact, for a completely unsorted array this is the best
algorithm, but, in practice, if finding the largest element in a set
is an important and frequent procedure, a special data structure,
called a heap, is used to keep track of which element is largest.

\begin{table}
\begin{lstlisting}[numbers=left]
int search(int a[],int n)
{

  int i;
  int best_val=a[0];

  for(i=1;i<n;i++)
    {
      if(a[i]>best_val)
	best_val= a[i];
    }

  return best_val;
}
\end{lstlisting}
\caption{Search for the largest element in an unsorted list. This
  function searches all the elements to see which is the largest, the
  inner loop always runs $n-1$ times since it doesn't know until it
  has looked at every element which is going to be the largest. This
  program is implemented as {\tt
    find\_largest.c}.\label{c_largest_linear_search}.}
\end{table}

Finally, little oh notation is a stricter version of big Oh notation
that is used in some more mathematical context, basically $f(n)\in
o(g(n))$ is $f(n)$ is less than $cg(n)$ for any choice of $c$, if $n$
is large enough:
\begin{equation}
\o(g(n))=\{f(n)| \exists n_0>0\in {\bf N}\mbox{ so that }|f(n)|\ge
c|g(n)|\,\forall n\ge n_0 \mbox{ and }\forall\,c>0\in {\bf R}\}
\end{equation}

\end{document}

\caption{This shows $\log_2{x}$ and $x-1$ plots for $x\in[1,4]$, the one has been taken from $x$ to make them easier to compare, the key point is that the $x$ grows faster.\label{fig_log}}
\end{figure}


\begin{figure}
\documentclass[11pt,a4paper]{scrartcl}
\typearea{12}
\usepackage{graphicx}
\usepackage{pstricks}
\usepackage{listings}
\lstset{language=C}
\pagestyle{headings}
\markright{COMS11600 - Principles of programming I.2}
\begin{document}

\subsection*{I.2 Big Oh notation}

The \lq{}Big Oh\rq{} notation has already been used to describe the
behavior of the running time of insert sort, we said
\begin{equation}
T(n)\in O(n^2)
\end{equation}
Here we want to formalize this notation. Basically $O(n^2)$ is a set
of function, it is all the function which, for large values of $n$ go
to infinity like $n^2$ at the fastest. By saying $T(n)\in O(n^2)$ we
are saying that $T(n)$ is one of these functions, its large $n$
behavior is, as worst, like $n^2$. 

Specifically, the definition for $g(n)$ for all $n$
\begin{equation}
O(g(n))=\{f(n)| \exists n_0>0\in {\bf N}\mbox{ and }c>0\in {\bf R}\mbox{ with }|f(n)|\le c|g(n)|\,\forall n\ge n_0\}
\end{equation}
This definition is quite dense, but we can break it down: it says that
$O(g(n))$ is a set of functions, the curly brackets mean
\lq{}set\rq{}. $f(n)$ is in the set if it has a particular
property: the \lq$|$\rq{} can be read as \lq{}such that\rq{} or
\lq{}with the property that\rq{} and so this is the set of $f(n)$'s
where $f(n)$ has the property on the right of the $|$. Now,
\lq{}$\exists$\rq{} means \lq{}there exists\rq{} and
\lq{}$\forall$\rq{} means \lq{}for all\rq{}, so the defining property
says it is possible to find a positive natural number $n_0$ and a
positive real number $c$ for that if you choose a value of $n$ at
least as big as $n_0$ then $f(n)$ is no bigger than $cg(n)$, ${\bf N}$
and ${\bf R}$ stand for the natural and real numbers. Notice the
absolute value signs, this is about $|f(n)|$ and $|g(n)|$, in fact,
here we are interested in run times, so we will deal with functions
that are non-negative, or are non-negative provided $n$ is larger than
some threshold, for example, $\log_2{n}$ will be important,
$\log_2{n}$ is positive provided $n>1$.

In short, $f(n)$ can do all sorts of crazy stuff for small values of
$n$ but, if you take $n$ large enough, its behavior is bounded by the
behavior of $g(n)$. Now it doesn't say it is bounded by $g(n)$, it is
a statement about the behavior, that's the role of the $c$. If you
know the formal definition of limits you can see that the definition
of $O(g(n))$ has this wrapped up in it, if says
\begin{equation}
f(n)\in O(g(n))\iff \lim_{n\rightarrow \infty}\frac{f(n)}{g(n)}<\infty
\end{equation}

Here are some examples, say 
\begin{equation}
T(n)=5n^2+n+6
\end{equation}
then 
\begin{equation}
T(n)\in O(n^2)
\end{equation}
by, for example, taking $c=5+1+6=12$ then 
\begin{equation}
12n^2\ge 5n^2+n+6
\end{equation}
provided $n\ge 1$ so $n_0=1$ here. This could also be succinctly demonstrated using the limit
\begin{equation}
\lim_{n\rightarrow \infty} \frac{5n^2+n+6}{n^2}=\lim_{n\rightarrow \infty}5+\lim_{n\rightarrow \infty}\frac{1}{n}+\lim_{n\rightarrow \infty}\frac{6}{n^2}=5<\infty
\end{equation}

However, 
\begin{equation}
T(n)\not\in O(n)
\end{equation}
Say we chosen some value $c$ then
\begin{equation}
5n^2+n+6>cn
\end{equation}
for large enough $n$, to check this divide both sides by $n$ so we need to show that $n$ can be chosen so that
\begin{equation}
5n+1+\frac{c}{n}>c
\end{equation}
Since 
\begin{equation}
5n+1+\frac{c}{n}>5n+1
\end{equation}
then, if $n>c/5$
\begin{equation}
5n+1+\frac{c}{n}>5\frac{c}{5}+1=c+1>c
\end{equation}
so, no matter what value of $c$ is chosen, making $n>5/n$ implies
\begin{equation}
5n^2+n+6>cn
\end{equation}
so $5n^2+n+6\not\in O(n)$. Again, the limit does the same job
\begin{equation}
\lim_{n\rightarrow \infty} \frac{5n^2+n+6}{n}=\lim_{n\rightarrow \infty}5n+\lim_{n\rightarrow \infty}1+\lim_{n\rightarrow \infty}\frac{6}{n}=\infty
\end{equation}


In practice, if
\begin{equation}
T(n)=a_rn^r+a_{r-1}n^{r-1}+\ldots+a_1n+a_0
\end{equation}
then $T(n)\in O(n^r)$. 

The logarithm has the funny property that it goes to infinity, but it
does so slower than $n$:
\begin{equation}
\lim_{n\rightarrow \infty}\frac{\log_2{n}}{n}=0
\end{equation}
Here, in line with standard practice in computer science, we are using
the log to the base two, in fact changing bases only causes a change
of an overall constant, see Table~\ref{math_logs} for a reminder of
the properties of the log. The limit of $\log_2{n}/n$ can be calculated
using l'H\^{o}pital's rule, here we'll just look at a plot,
Fig.~\ref{fig_log}. Now, just as $\log_2{n}$ grows very slowly, $2^n$ grows very fast, 
\begin{equation}
\lim_{n\rightarrow \infty}\frac{n^r}{2^n}=0
\end{equation}
for any finite value of $r$, worse still is $n!$, pronounced $n$-factorial
\begin{equation}
n!=n(n-1)(n-2) . . . 1
\end{equation}
If you algorithm is in $O(n!)$ you will probably need a different
algorithm. A table of different values is given as
Table~\ref{table_n_values}, mostly to emphasis how quickly $n!$ gets
big.

Now, in mathematics we call something \lq{}an abuse of notation\rq{}
if it is common to write something that doesn't quite make sense but
acts as a shorthand for something that does. Now $O(g(n))$ is a set of
functions whose large $n$ behavior is bounded by $g(n)$ so in
algorithms, being in $O(g(n))$ is a property of $T(n)$, the formula
for the running time of the algorithm. However, it is a standard abuse
of notation to say an algorithm is $O(g(n))$ for some $g(n)$ is
$T(n)\in O(g(n))$ for all cases. This is another way of saying that
the worst behavior of the algorithm isn't any worse than the behavior
of $g(n)$ for large $n$ so all possible $T(n)$ are elements of
$O(g(n))$.

\begin{table}
The logarithm is the opposite of the exponent: if
\begin{equation}
a^b=c
\end{equation}
then 
\begin{equation}
\log_a{c}=b
\end{equation}
or, written in one line
\begin{equation}
a^{\log_a{c}}=c
\end{equation}
All the laws of logs can be worked out from the laws of
exponents. Hence, since $a^0=1$ we have $\log_a{1}=0$. In a similar
way, other rules of logs can be deduced like
\begin{eqnarray}
\log_a{c_1c_2}&=&\log_a{c_1}+\log_a{c_2}\cr
\log_a{\frac{c_1}{c_2}}&=&\log_a{c_1}-\log_a{c_2}\cr
\log_a c^d&=&d\log_a c
\end{eqnarray}
and so on.

As for the change of base, let $b=\log_{a_1}{c}$ so $a_1^{b}=c$. Now take the log to the base $a_2$ of both sides
\begin{equation}
b\log_{a_2}a_1=\log_{a_2}{c}
\end{equation}
and then solve for $b$
\begin{equation}
b=\frac{\log_{a_2}c}{\log_{a_2}a_1}
\end{equation}
and, substituting back the formula for $b$
\begin{equation}
\log_{a_1}{c}=\frac{\log_{a_2}c}{\log_{a_2}a_1}
\end{equation}
Thus we see, that changing bases is just a matter of a
multiplicative factor. For example, to change from base $e$ to base two
\begin{equation}
\log_{2}{x}=\frac{\log_e{x}}{\log_e{2}}\approx\frac{\log_e{x}}{0.6931}
\end{equation}
Common bases are $\log_2{x}$ used in computer science, $\log_e{x}$
sometimes written $\ln{x}$ used in mathematics and $\log_{10}{x}$ used
in chemistry. The base two is used because of its link to bits and
also, as we will see, because of its relationship with algorithms that
divide data into two piles. The natural log $\ln{x}$ is used where
differential equations are common since
\begin{equation}
\frac{d}{dx}\ln{x}=\frac{1}{x}
\end{equation}
\caption{A reminder about logarithms. This is a quick summary of some of the laws of logs.\label{math_logs}}
\end{table}

\begin{figure}
\include{I.2.log}
\caption{This shows $\log_2{x}$ and $x-1$ plots for $x\in[1,4]$, the one has been taken from $x$ to make them easier to compare, the key point is that the $x$ grows faster.\label{fig_log}}
\end{figure}


\begin{figure}
\include{I.2.exp}
\caption{This shows $2^x$ and $x^2-1$ plots for $x\in[0,6]$, clearly $2^x$ quickly overtakes $x^2+1$, this will happen for any power of $x$. \label{fig_log}}
\end{figure}

\begin{table}
\begin{tabular}{l|cccccc}
        $n$    &1   &2&4   &16  &128&1024\\
$\log{n}$      &0   &1&2   &4   &7  &10\\
$n\log{n}$     &0   &2&8   &64  &896&10240\\
$n^2$     &1   &4&16&256&16384&1048576\\
$2^n$     &2   &4&16&65536&$3.4\times 10^{38}$&$1.8\times 10^{307}$\\
$n!$      &1   &2&24&$2.1\times 10^{13}$&$3.85\times 10^{305}$&$5.4\times10^{2369}$
\end{tabular}
\vskip 1cm The website\\ {\tt
  http://markknowsnothing.com/cgi-bin/calculator.php}\\ was used for
the $2^n$ calculations and\\ {\tt
  http://www.calculatorsoup.com/calculators/discretemathematics/factorials.php}\\ for
the $n!$ calculations; these give answers even when the answer is very
large.

\caption{Different values of $n$ for some functions.  \label{table_n_values}
}
\end{table}

\subsubsection*{I.2.1 Examples: linear and binary search}

Lets do a quick example; searching for the index of an element in a
sorted list. A completely terrible way to do this is linear search,
this is terrible because it doesn't make use of the fact that the list
is sorted. A code listing is given in Table~\ref{c_linear_search}. We
can see straight away that the code between lines 6 and 10 is run $n$
times in the worst case, everything else is run once and so this
algorithm is $O(n)$.

\begin{table}
\begin{lstlisting}[numbers=left]
int search(int a[],int n, int val)
{

  int i;

  for(i=0;i<n;i++)
    {
      if(a[i]==val)
	return i;
    }

  return -1;
}
\end{lstlisting}
\caption{Linear search. This function searches the entries in the array a and returns the index when it finds val, if it doesn't find val it returns -1. The program {\tt linear\_search.c} implements this.\label{c_linear_search}.}
\end{table}

A much better way to search a sorted array is binary search. This is
an example of a \lq{}divide and conquer\rq{} algorithm, many of the
fastest algorithms use divide and conquer. It will be clear to you
that this algorithm would be better written using recursion, this is
typical of divide and conquer, but we haven't looked at analysing
recursion yet. The idea is to divide the array in half and check which
half, the half with bigger numbers or the half with smaller numbers,
the value we are searching for belongs to and to keep doing this,
dividing the remaining part of the array into two parts again and again
until the remaining part of the array that is being search has only
one element. A code listing is given in Table~\ref{c_binary_search}.

\begin{table}
\begin{lstlisting}[numbers=left] 
int search(int a[],int n, int val)
{
  int mid, low=0, high=n-1;

  while(low<=high)
    {
      mid=(low+high)/2;
      if(a[mid]==val)
	return mid;
      else if(val>a[mid])
	low=mid+1;
      else
	high=mid -1;
    }

  return -1;
}
\end{lstlisting}
\caption{Binary search. This function starts in the middle of the array
  and checks if the value there is bigger or smaller than val, if it
  is bigger then it does the same in the top half of the array, if it
  is smaller, in the bottom half and then repeats until there are no
  elements left. The program {\tt binary\_search.c} implements
  this.\label{c_binary_search}.}
\end{table}

This search is extremely fast. There is a chance that a[mid]==val,
after a small number of iterations, indeed, if the middle value of the
array is the search value it will halt after only one
iteration. However, as usual, we assume the worst case, in which case
the algorithm runs to end, dividing the number of elements in half
each time. Ignoring the integer rounding effects, it goes like
Table~\ref{tab_binary_search}. Starting with $n$ states each
subsequent iteration halves the number of states until the last one
when there is one state left. Thus
\begin{equation}
1=\frac{n}{2^{T(n)-1}}
\end{equation}
and taking the log of both sides
\begin{equation}
0=\log_2{n}-(T(n)-1)
\end{equation}
using $\log_2{1}=1$, $\log_2{a^b}=b\log_2{a}$ and $\log_2{2}=1$. Hence
\begin{equation}
T(n)=\log_2{n}+1
\end{equation}
and this algorithm is $O(\log_2{n})$.

\begin{table}
\begin{center}
\begin{tabular}{llllllll}
1&2&3&4&\ldots&k&\ldots&T(n)\\
\hline
$n$&$\frac{n}{2}$&$\frac{n}{4}$&$\frac{n}{8}$&$\ldots$&$\frac{n}{2^{k-1}}$&$\ldots$&1
\end{tabular}
\end{center}
\caption{The number of elements left for binary search.\label{tab_binary_search}}
\end{table}

So, to reiterate; the usual way to examine the behavior of an
algorithm is to look at the worst case run time. This is because the
best case run time is often exceptional, like the one for binary
search if the first guess happens to be correct. The average run time
is often hard to calculate, both because it is often difficult to do
the mathematics and because it would often mean having some
description of how the initial data is distributed. Typically the
worst run time is also \lq{}of the same order\rq{} as the average run
time. We will see an exception to this later on in the case of quick
sort in which the worst case behavior is unusual. The big-Oh notation
is used for describing an algorithm, if the algorithm is said to be
$O(g(n))$ we mean $T(n)\in O(g(n))$ no matter what the initial
condition. Since $O(g(n))$ involves an upper bound $T(n)<cg(n)$ this
makes sense.


\subsubsection*{Other big Letter notations, small oh notation.}

There is another set, $\Omega(g(n))$ with a definition similar to
$O(g(n))$ that is used for describing the best case behavior. This
requires a lower bound rather than an upper bound, so the obvious
definition is
\begin{equation}
\Omega(g(n))=\{f(n)| \exists n_0>0\in {\bf N}\mbox{ and }c>0\in {\bf R}\mbox{ with }|f(n)|\ge c|g(n)|\,\forall n\ge n_0\}
\end{equation}
in other words, the same thing, but with the $\le$ symbol replaced by
a $\ge$. In fact, there is some ambiguity about this definition,
number theorist use a slightly different one. Either way, it isn't
used very often in computer science because algorithms are very
frequently $\Omega(1)$; in the best case scenario the problem is in
some sense already solve, the array already sorted for example, and
the algorithm finishs in one step. 

There is also a set of function that are both bounded above and below
by the same $g(n)$
\begin{equation}
\Theta(g(n))=\Omega(g(n))\cap O(g(n))
\end{equation}
This works because it is possible for
\begin{equation}
c_1 g(n)\le f(n)\le c_2g(n)
\end{equation}
for different $c_1$ and $c_2$. It would be very unusual for this to
apply to an algorithm, it would mean that $T(n)$ has the same behavior
for large $n$ no matter whether it is the best case or the worst case
scenario. There is a na\"ive largest element function in
Table~\ref{c_largest_linear_search} which is $\Theta(n)$. It searches
for the largest value in an unsorted array by looking at each element
in turn. In fact, for a completely unsorted array this is the best
algorithm, but, in practice, if finding the largest element in a set
is an important and frequent procedure, a special data structure,
called a heap, is used to keep track of which element is largest.

\begin{table}
\begin{lstlisting}[numbers=left]
int search(int a[],int n)
{

  int i;
  int best_val=a[0];

  for(i=1;i<n;i++)
    {
      if(a[i]>best_val)
	best_val= a[i];
    }

  return best_val;
}
\end{lstlisting}
\caption{Search for the largest element in an unsorted list. This
  function searches all the elements to see which is the largest, the
  inner loop always runs $n-1$ times since it doesn't know until it
  has looked at every element which is going to be the largest. This
  program is implemented as {\tt
    find\_largest.c}.\label{c_largest_linear_search}.}
\end{table}

Finally, little oh notation is a stricter version of big Oh notation
that is used in some more mathematical context, basically $f(n)\in
o(g(n))$ is $f(n)$ is less than $cg(n)$ for any choice of $c$, if $n$
is large enough:
\begin{equation}
\o(g(n))=\{f(n)| \exists n_0>0\in {\bf N}\mbox{ so that }|f(n)|\ge
c|g(n)|\,\forall n\ge n_0 \mbox{ and }\forall\,c>0\in {\bf R}\}
\end{equation}

\end{document}

\caption{This shows $2^x$ and $x^2-1$ plots for $x\in[0,6]$, clearly $2^x$ quickly overtakes $x^2+1$, this will happen for any power of $x$. \label{fig_log}}
\end{figure}

\begin{table}
\begin{tabular}{l|cccccc}
        $n$    &1   &2&4   &16  &128&1024\\
$\log{n}$      &0   &1&2   &4   &7  &10\\
$n\log{n}$     &0   &2&8   &64  &896&10240\\
$n^2$     &1   &4&16&256&16384&1048576\\
$2^n$     &2   &4&16&65536&$3.4\times 10^{38}$&$1.8\times 10^{307}$\\
$n!$      &1   &2&24&$2.1\times 10^{13}$&$3.85\times 10^{305}$&$5.4\times10^{2369}$
\end{tabular}
\vskip 1cm The website\\ {\tt
  http://markknowsnothing.com/cgi-bin/calculator.php}\\ was used for
the $2^n$ calculations and\\ {\tt
  http://www.calculatorsoup.com/calculators/discretemathematics/factorials.php}\\ for
the $n!$ calculations; these give answers even when the answer is very
large.

\caption{Different values of $n$ for some functions.  \label{table_n_values}
}
\end{table}

\subsubsection*{I.2.1 Examples: linear and binary search}

Lets do a quick example; searching for the index of an element in a
sorted list. A completely terrible way to do this is linear search,
this is terrible because it doesn't make use of the fact that the list
is sorted. A code listing is given in Table~\ref{c_linear_search}. We
can see straight away that the code between lines 6 and 10 is run $n$
times in the worst case, everything else is run once and so this
algorithm is $O(n)$.

\begin{table}
\begin{lstlisting}[numbers=left]
int search(int a[],int n, int val)
{

  int i;

  for(i=0;i<n;i++)
    {
      if(a[i]==val)
	return i;
    }

  return -1;
}
\end{lstlisting}
\caption{Linear search. This function searches the entries in the array a and returns the index when it finds val, if it doesn't find val it returns -1. The program {\tt linear\_search.c} implements this.\label{c_linear_search}.}
\end{table}

A much better way to search a sorted array is binary search. This is
an example of a \lq{}divide and conquer\rq{} algorithm, many of the
fastest algorithms use divide and conquer. It will be clear to you
that this algorithm would be better written using recursion, this is
typical of divide and conquer, but we haven't looked at analysing
recursion yet. The idea is to divide the array in half and check which
half, the half with bigger numbers or the half with smaller numbers,
the value we are searching for belongs to and to keep doing this,
dividing the remaining part of the array into two parts again and again
until the remaining part of the array that is being search has only
one element. A code listing is given in Table~\ref{c_binary_search}.

\begin{table}
\begin{lstlisting}[numbers=left] 
int search(int a[],int n, int val)
{
  int mid, low=0, high=n-1;

  while(low<=high)
    {
      mid=(low+high)/2;
      if(a[mid]==val)
	return mid;
      else if(val>a[mid])
	low=mid+1;
      else
	high=mid -1;
    }

  return -1;
}
\end{lstlisting}
\caption{Binary search. This function starts in the middle of the array
  and checks if the value there is bigger or smaller than val, if it
  is bigger then it does the same in the top half of the array, if it
  is smaller, in the bottom half and then repeats until there are no
  elements left. The program {\tt binary\_search.c} implements
  this.\label{c_binary_search}.}
\end{table}

This search is extremely fast. There is a chance that a[mid]==val,
after a small number of iterations, indeed, if the middle value of the
array is the search value it will halt after only one
iteration. However, as usual, we assume the worst case, in which case
the algorithm runs to end, dividing the number of elements in half
each time. Ignoring the integer rounding effects, it goes like
Table~\ref{tab_binary_search}. Starting with $n$ states each
subsequent iteration halves the number of states until the last one
when there is one state left. Thus
\begin{equation}
1=\frac{n}{2^{T(n)-1}}
\end{equation}
and taking the log of both sides
\begin{equation}
0=\log_2{n}-(T(n)-1)
\end{equation}
using $\log_2{1}=1$, $\log_2{a^b}=b\log_2{a}$ and $\log_2{2}=1$. Hence
\begin{equation}
T(n)=\log_2{n}+1
\end{equation}
and this algorithm is $O(\log_2{n})$.

\begin{table}
\begin{center}
\begin{tabular}{llllllll}
1&2&3&4&\ldots&k&\ldots&T(n)\\
\hline
$n$&$\frac{n}{2}$&$\frac{n}{4}$&$\frac{n}{8}$&$\ldots$&$\frac{n}{2^{k-1}}$&$\ldots$&1
\end{tabular}
\end{center}
\caption{The number of elements left for binary search.\label{tab_binary_search}}
\end{table}

So, to reiterate; the usual way to examine the behavior of an
algorithm is to look at the worst case run time. This is because the
best case run time is often exceptional, like the one for binary
search if the first guess happens to be correct. The average run time
is often hard to calculate, both because it is often difficult to do
the mathematics and because it would often mean having some
description of how the initial data is distributed. Typically the
worst run time is also \lq{}of the same order\rq{} as the average run
time. We will see an exception to this later on in the case of quick
sort in which the worst case behavior is unusual. The big-Oh notation
is used for describing an algorithm, if the algorithm is said to be
$O(g(n))$ we mean $T(n)\in O(g(n))$ no matter what the initial
condition. Since $O(g(n))$ involves an upper bound $T(n)<cg(n)$ this
makes sense.


\subsubsection*{Other big Letter notations, small oh notation.}

There is another set, $\Omega(g(n))$ with a definition similar to
$O(g(n))$ that is used for describing the best case behavior. This
requires a lower bound rather than an upper bound, so the obvious
definition is
\begin{equation}
\Omega(g(n))=\{f(n)| \exists n_0>0\in {\bf N}\mbox{ and }c>0\in {\bf R}\mbox{ with }|f(n)|\ge c|g(n)|\,\forall n\ge n_0\}
\end{equation}
in other words, the same thing, but with the $\le$ symbol replaced by
a $\ge$. In fact, there is some ambiguity about this definition,
number theorist use a slightly different one. Either way, it isn't
used very often in computer science because algorithms are very
frequently $\Omega(1)$; in the best case scenario the problem is in
some sense already solve, the array already sorted for example, and
the algorithm finishs in one step. 

There is also a set of function that are both bounded above and below
by the same $g(n)$
\begin{equation}
\Theta(g(n))=\Omega(g(n))\cap O(g(n))
\end{equation}
This works because it is possible for
\begin{equation}
c_1 g(n)\le f(n)\le c_2g(n)
\end{equation}
for different $c_1$ and $c_2$. It would be very unusual for this to
apply to an algorithm, it would mean that $T(n)$ has the same behavior
for large $n$ no matter whether it is the best case or the worst case
scenario. There is a na\"ive largest element function in
Table~\ref{c_largest_linear_search} which is $\Theta(n)$. It searches
for the largest value in an unsorted array by looking at each element
in turn. In fact, for a completely unsorted array this is the best
algorithm, but, in practice, if finding the largest element in a set
is an important and frequent procedure, a special data structure,
called a heap, is used to keep track of which element is largest.

\begin{table}
\begin{lstlisting}[numbers=left]
int search(int a[],int n)
{

  int i;
  int best_val=a[0];

  for(i=1;i<n;i++)
    {
      if(a[i]>best_val)
	best_val= a[i];
    }

  return best_val;
}
\end{lstlisting}
\caption{Search for the largest element in an unsorted list. This
  function searches all the elements to see which is the largest, the
  inner loop always runs $n-1$ times since it doesn't know until it
  has looked at every element which is going to be the largest. This
  program is implemented as {\tt
    find\_largest.c}.\label{c_largest_linear_search}.}
\end{table}

Finally, little oh notation is a stricter version of big Oh notation
that is used in some more mathematical context, basically $f(n)\in
o(g(n))$ is $f(n)$ is less than $cg(n)$ for any choice of $c$, if $n$
is large enough:
\begin{equation}
\o(g(n))=\{f(n)| \exists n_0>0\in {\bf N}\mbox{ so that }|f(n)|\ge
c|g(n)|\,\forall n\ge n_0 \mbox{ and }\forall\,c>0\in {\bf R}\}
\end{equation}

\end{document}

\caption{This shows $\log_2{x}$ and $x-1$ plots for $x\in[1,4]$, the one has been taken from $x$ to make them easier to compare, the key point is that the $x$ grows faster.\label{fig_log}}
\end{figure}


\begin{figure}
\documentclass[11pt,a4paper]{scrartcl}
\typearea{12}
\usepackage{graphicx}
\usepackage{pstricks}
\usepackage{listings}
\lstset{language=C}
\pagestyle{headings}
\markright{COMS11600 - Principles of programming I.2}
\begin{document}

\subsection*{I.2 Big Oh notation}

The \lq{}Big Oh\rq{} notation has already been used to describe the
behavior of the running time of insert sort, we said
\begin{equation}
T(n)\in O(n^2)
\end{equation}
Here we want to formalize this notation. Basically $O(n^2)$ is a set
of function, it is all the function which, for large values of $n$ go
to infinity like $n^2$ at the fastest. By saying $T(n)\in O(n^2)$ we
are saying that $T(n)$ is one of these functions, its large $n$
behavior is, as worst, like $n^2$. 

Specifically, the definition for $g(n)$ for all $n$
\begin{equation}
O(g(n))=\{f(n)| \exists n_0>0\in {\bf N}\mbox{ and }c>0\in {\bf R}\mbox{ with }|f(n)|\le c|g(n)|\,\forall n\ge n_0\}
\end{equation}
This definition is quite dense, but we can break it down: it says that
$O(g(n))$ is a set of functions, the curly brackets mean
\lq{}set\rq{}. $f(n)$ is in the set if it has a particular
property: the \lq$|$\rq{} can be read as \lq{}such that\rq{} or
\lq{}with the property that\rq{} and so this is the set of $f(n)$'s
where $f(n)$ has the property on the right of the $|$. Now,
\lq{}$\exists$\rq{} means \lq{}there exists\rq{} and
\lq{}$\forall$\rq{} means \lq{}for all\rq{}, so the defining property
says it is possible to find a positive natural number $n_0$ and a
positive real number $c$ for that if you choose a value of $n$ at
least as big as $n_0$ then $f(n)$ is no bigger than $cg(n)$, ${\bf N}$
and ${\bf R}$ stand for the natural and real numbers. Notice the
absolute value signs, this is about $|f(n)|$ and $|g(n)|$, in fact,
here we are interested in run times, so we will deal with functions
that are non-negative, or are non-negative provided $n$ is larger than
some threshold, for example, $\log_2{n}$ will be important,
$\log_2{n}$ is positive provided $n>1$.

In short, $f(n)$ can do all sorts of crazy stuff for small values of
$n$ but, if you take $n$ large enough, its behavior is bounded by the
behavior of $g(n)$. Now it doesn't say it is bounded by $g(n)$, it is
a statement about the behavior, that's the role of the $c$. If you
know the formal definition of limits you can see that the definition
of $O(g(n))$ has this wrapped up in it, if says
\begin{equation}
f(n)\in O(g(n))\iff \lim_{n\rightarrow \infty}\frac{f(n)}{g(n)}<\infty
\end{equation}

Here are some examples, say 
\begin{equation}
T(n)=5n^2+n+6
\end{equation}
then 
\begin{equation}
T(n)\in O(n^2)
\end{equation}
by, for example, taking $c=5+1+6=12$ then 
\begin{equation}
12n^2\ge 5n^2+n+6
\end{equation}
provided $n\ge 1$ so $n_0=1$ here. This could also be succinctly demonstrated using the limit
\begin{equation}
\lim_{n\rightarrow \infty} \frac{5n^2+n+6}{n^2}=\lim_{n\rightarrow \infty}5+\lim_{n\rightarrow \infty}\frac{1}{n}+\lim_{n\rightarrow \infty}\frac{6}{n^2}=5<\infty
\end{equation}

However, 
\begin{equation}
T(n)\not\in O(n)
\end{equation}
Say we chosen some value $c$ then
\begin{equation}
5n^2+n+6>cn
\end{equation}
for large enough $n$, to check this divide both sides by $n$ so we need to show that $n$ can be chosen so that
\begin{equation}
5n+1+\frac{c}{n}>c
\end{equation}
Since 
\begin{equation}
5n+1+\frac{c}{n}>5n+1
\end{equation}
then, if $n>c/5$
\begin{equation}
5n+1+\frac{c}{n}>5\frac{c}{5}+1=c+1>c
\end{equation}
so, no matter what value of $c$ is chosen, making $n>5/n$ implies
\begin{equation}
5n^2+n+6>cn
\end{equation}
so $5n^2+n+6\not\in O(n)$. Again, the limit does the same job
\begin{equation}
\lim_{n\rightarrow \infty} \frac{5n^2+n+6}{n}=\lim_{n\rightarrow \infty}5n+\lim_{n\rightarrow \infty}1+\lim_{n\rightarrow \infty}\frac{6}{n}=\infty
\end{equation}


In practice, if
\begin{equation}
T(n)=a_rn^r+a_{r-1}n^{r-1}+\ldots+a_1n+a_0
\end{equation}
then $T(n)\in O(n^r)$. 

The logarithm has the funny property that it goes to infinity, but it
does so slower than $n$:
\begin{equation}
\lim_{n\rightarrow \infty}\frac{\log_2{n}}{n}=0
\end{equation}
Here, in line with standard practice in computer science, we are using
the log to the base two, in fact changing bases only causes a change
of an overall constant, see Table~\ref{math_logs} for a reminder of
the properties of the log. The limit of $\log_2{n}/n$ can be calculated
using l'H\^{o}pital's rule, here we'll just look at a plot,
Fig.~\ref{fig_log}. Now, just as $\log_2{n}$ grows very slowly, $2^n$ grows very fast, 
\begin{equation}
\lim_{n\rightarrow \infty}\frac{n^r}{2^n}=0
\end{equation}
for any finite value of $r$, worse still is $n!$, pronounced $n$-factorial
\begin{equation}
n!=n(n-1)(n-2) . . . 1
\end{equation}
If you algorithm is in $O(n!)$ you will probably need a different
algorithm. A table of different values is given as
Table~\ref{table_n_values}, mostly to emphasis how quickly $n!$ gets
big.

Now, in mathematics we call something \lq{}an abuse of notation\rq{}
if it is common to write something that doesn't quite make sense but
acts as a shorthand for something that does. Now $O(g(n))$ is a set of
functions whose large $n$ behavior is bounded by $g(n)$ so in
algorithms, being in $O(g(n))$ is a property of $T(n)$, the formula
for the running time of the algorithm. However, it is a standard abuse
of notation to say an algorithm is $O(g(n))$ for some $g(n)$ is
$T(n)\in O(g(n))$ for all cases. This is another way of saying that
the worst behavior of the algorithm isn't any worse than the behavior
of $g(n)$ for large $n$ so all possible $T(n)$ are elements of
$O(g(n))$.

\begin{table}
The logarithm is the opposite of the exponent: if
\begin{equation}
a^b=c
\end{equation}
then 
\begin{equation}
\log_a{c}=b
\end{equation}
or, written in one line
\begin{equation}
a^{\log_a{c}}=c
\end{equation}
All the laws of logs can be worked out from the laws of
exponents. Hence, since $a^0=1$ we have $\log_a{1}=0$. In a similar
way, other rules of logs can be deduced like
\begin{eqnarray}
\log_a{c_1c_2}&=&\log_a{c_1}+\log_a{c_2}\cr
\log_a{\frac{c_1}{c_2}}&=&\log_a{c_1}-\log_a{c_2}\cr
\log_a c^d&=&d\log_a c
\end{eqnarray}
and so on.

As for the change of base, let $b=\log_{a_1}{c}$ so $a_1^{b}=c$. Now take the log to the base $a_2$ of both sides
\begin{equation}
b\log_{a_2}a_1=\log_{a_2}{c}
\end{equation}
and then solve for $b$
\begin{equation}
b=\frac{\log_{a_2}c}{\log_{a_2}a_1}
\end{equation}
and, substituting back the formula for $b$
\begin{equation}
\log_{a_1}{c}=\frac{\log_{a_2}c}{\log_{a_2}a_1}
\end{equation}
Thus we see, that changing bases is just a matter of a
multiplicative factor. For example, to change from base $e$ to base two
\begin{equation}
\log_{2}{x}=\frac{\log_e{x}}{\log_e{2}}\approx\frac{\log_e{x}}{0.6931}
\end{equation}
Common bases are $\log_2{x}$ used in computer science, $\log_e{x}$
sometimes written $\ln{x}$ used in mathematics and $\log_{10}{x}$ used
in chemistry. The base two is used because of its link to bits and
also, as we will see, because of its relationship with algorithms that
divide data into two piles. The natural log $\ln{x}$ is used where
differential equations are common since
\begin{equation}
\frac{d}{dx}\ln{x}=\frac{1}{x}
\end{equation}
\caption{A reminder about logarithms. This is a quick summary of some of the laws of logs.\label{math_logs}}
\end{table}

\begin{figure}
\documentclass[11pt,a4paper]{scrartcl}
\typearea{12}
\usepackage{graphicx}
\usepackage{pstricks}
\usepackage{listings}
\lstset{language=C}
\pagestyle{headings}
\markright{COMS11600 - Principles of programming I.2}
\begin{document}

\subsection*{I.2 Big Oh notation}

The \lq{}Big Oh\rq{} notation has already been used to describe the
behavior of the running time of insert sort, we said
\begin{equation}
T(n)\in O(n^2)
\end{equation}
Here we want to formalize this notation. Basically $O(n^2)$ is a set
of function, it is all the function which, for large values of $n$ go
to infinity like $n^2$ at the fastest. By saying $T(n)\in O(n^2)$ we
are saying that $T(n)$ is one of these functions, its large $n$
behavior is, as worst, like $n^2$. 

Specifically, the definition for $g(n)$ for all $n$
\begin{equation}
O(g(n))=\{f(n)| \exists n_0>0\in {\bf N}\mbox{ and }c>0\in {\bf R}\mbox{ with }|f(n)|\le c|g(n)|\,\forall n\ge n_0\}
\end{equation}
This definition is quite dense, but we can break it down: it says that
$O(g(n))$ is a set of functions, the curly brackets mean
\lq{}set\rq{}. $f(n)$ is in the set if it has a particular
property: the \lq$|$\rq{} can be read as \lq{}such that\rq{} or
\lq{}with the property that\rq{} and so this is the set of $f(n)$'s
where $f(n)$ has the property on the right of the $|$. Now,
\lq{}$\exists$\rq{} means \lq{}there exists\rq{} and
\lq{}$\forall$\rq{} means \lq{}for all\rq{}, so the defining property
says it is possible to find a positive natural number $n_0$ and a
positive real number $c$ for that if you choose a value of $n$ at
least as big as $n_0$ then $f(n)$ is no bigger than $cg(n)$, ${\bf N}$
and ${\bf R}$ stand for the natural and real numbers. Notice the
absolute value signs, this is about $|f(n)|$ and $|g(n)|$, in fact,
here we are interested in run times, so we will deal with functions
that are non-negative, or are non-negative provided $n$ is larger than
some threshold, for example, $\log_2{n}$ will be important,
$\log_2{n}$ is positive provided $n>1$.

In short, $f(n)$ can do all sorts of crazy stuff for small values of
$n$ but, if you take $n$ large enough, its behavior is bounded by the
behavior of $g(n)$. Now it doesn't say it is bounded by $g(n)$, it is
a statement about the behavior, that's the role of the $c$. If you
know the formal definition of limits you can see that the definition
of $O(g(n))$ has this wrapped up in it, if says
\begin{equation}
f(n)\in O(g(n))\iff \lim_{n\rightarrow \infty}\frac{f(n)}{g(n)}<\infty
\end{equation}

Here are some examples, say 
\begin{equation}
T(n)=5n^2+n+6
\end{equation}
then 
\begin{equation}
T(n)\in O(n^2)
\end{equation}
by, for example, taking $c=5+1+6=12$ then 
\begin{equation}
12n^2\ge 5n^2+n+6
\end{equation}
provided $n\ge 1$ so $n_0=1$ here. This could also be succinctly demonstrated using the limit
\begin{equation}
\lim_{n\rightarrow \infty} \frac{5n^2+n+6}{n^2}=\lim_{n\rightarrow \infty}5+\lim_{n\rightarrow \infty}\frac{1}{n}+\lim_{n\rightarrow \infty}\frac{6}{n^2}=5<\infty
\end{equation}

However, 
\begin{equation}
T(n)\not\in O(n)
\end{equation}
Say we chosen some value $c$ then
\begin{equation}
5n^2+n+6>cn
\end{equation}
for large enough $n$, to check this divide both sides by $n$ so we need to show that $n$ can be chosen so that
\begin{equation}
5n+1+\frac{c}{n}>c
\end{equation}
Since 
\begin{equation}
5n+1+\frac{c}{n}>5n+1
\end{equation}
then, if $n>c/5$
\begin{equation}
5n+1+\frac{c}{n}>5\frac{c}{5}+1=c+1>c
\end{equation}
so, no matter what value of $c$ is chosen, making $n>5/n$ implies
\begin{equation}
5n^2+n+6>cn
\end{equation}
so $5n^2+n+6\not\in O(n)$. Again, the limit does the same job
\begin{equation}
\lim_{n\rightarrow \infty} \frac{5n^2+n+6}{n}=\lim_{n\rightarrow \infty}5n+\lim_{n\rightarrow \infty}1+\lim_{n\rightarrow \infty}\frac{6}{n}=\infty
\end{equation}


In practice, if
\begin{equation}
T(n)=a_rn^r+a_{r-1}n^{r-1}+\ldots+a_1n+a_0
\end{equation}
then $T(n)\in O(n^r)$. 

The logarithm has the funny property that it goes to infinity, but it
does so slower than $n$:
\begin{equation}
\lim_{n\rightarrow \infty}\frac{\log_2{n}}{n}=0
\end{equation}
Here, in line with standard practice in computer science, we are using
the log to the base two, in fact changing bases only causes a change
of an overall constant, see Table~\ref{math_logs} for a reminder of
the properties of the log. The limit of $\log_2{n}/n$ can be calculated
using l'H\^{o}pital's rule, here we'll just look at a plot,
Fig.~\ref{fig_log}. Now, just as $\log_2{n}$ grows very slowly, $2^n$ grows very fast, 
\begin{equation}
\lim_{n\rightarrow \infty}\frac{n^r}{2^n}=0
\end{equation}
for any finite value of $r$, worse still is $n!$, pronounced $n$-factorial
\begin{equation}
n!=n(n-1)(n-2) . . . 1
\end{equation}
If you algorithm is in $O(n!)$ you will probably need a different
algorithm. A table of different values is given as
Table~\ref{table_n_values}, mostly to emphasis how quickly $n!$ gets
big.

Now, in mathematics we call something \lq{}an abuse of notation\rq{}
if it is common to write something that doesn't quite make sense but
acts as a shorthand for something that does. Now $O(g(n))$ is a set of
functions whose large $n$ behavior is bounded by $g(n)$ so in
algorithms, being in $O(g(n))$ is a property of $T(n)$, the formula
for the running time of the algorithm. However, it is a standard abuse
of notation to say an algorithm is $O(g(n))$ for some $g(n)$ is
$T(n)\in O(g(n))$ for all cases. This is another way of saying that
the worst behavior of the algorithm isn't any worse than the behavior
of $g(n)$ for large $n$ so all possible $T(n)$ are elements of
$O(g(n))$.

\begin{table}
The logarithm is the opposite of the exponent: if
\begin{equation}
a^b=c
\end{equation}
then 
\begin{equation}
\log_a{c}=b
\end{equation}
or, written in one line
\begin{equation}
a^{\log_a{c}}=c
\end{equation}
All the laws of logs can be worked out from the laws of
exponents. Hence, since $a^0=1$ we have $\log_a{1}=0$. In a similar
way, other rules of logs can be deduced like
\begin{eqnarray}
\log_a{c_1c_2}&=&\log_a{c_1}+\log_a{c_2}\cr
\log_a{\frac{c_1}{c_2}}&=&\log_a{c_1}-\log_a{c_2}\cr
\log_a c^d&=&d\log_a c
\end{eqnarray}
and so on.

As for the change of base, let $b=\log_{a_1}{c}$ so $a_1^{b}=c$. Now take the log to the base $a_2$ of both sides
\begin{equation}
b\log_{a_2}a_1=\log_{a_2}{c}
\end{equation}
and then solve for $b$
\begin{equation}
b=\frac{\log_{a_2}c}{\log_{a_2}a_1}
\end{equation}
and, substituting back the formula for $b$
\begin{equation}
\log_{a_1}{c}=\frac{\log_{a_2}c}{\log_{a_2}a_1}
\end{equation}
Thus we see, that changing bases is just a matter of a
multiplicative factor. For example, to change from base $e$ to base two
\begin{equation}
\log_{2}{x}=\frac{\log_e{x}}{\log_e{2}}\approx\frac{\log_e{x}}{0.6931}
\end{equation}
Common bases are $\log_2{x}$ used in computer science, $\log_e{x}$
sometimes written $\ln{x}$ used in mathematics and $\log_{10}{x}$ used
in chemistry. The base two is used because of its link to bits and
also, as we will see, because of its relationship with algorithms that
divide data into two piles. The natural log $\ln{x}$ is used where
differential equations are common since
\begin{equation}
\frac{d}{dx}\ln{x}=\frac{1}{x}
\end{equation}
\caption{A reminder about logarithms. This is a quick summary of some of the laws of logs.\label{math_logs}}
\end{table}

\begin{figure}
\include{I.2.log}
\caption{This shows $\log_2{x}$ and $x-1$ plots for $x\in[1,4]$, the one has been taken from $x$ to make them easier to compare, the key point is that the $x$ grows faster.\label{fig_log}}
\end{figure}


\begin{figure}
\include{I.2.exp}
\caption{This shows $2^x$ and $x^2-1$ plots for $x\in[0,6]$, clearly $2^x$ quickly overtakes $x^2+1$, this will happen for any power of $x$. \label{fig_log}}
\end{figure}

\begin{table}
\begin{tabular}{l|cccccc}
        $n$    &1   &2&4   &16  &128&1024\\
$\log{n}$      &0   &1&2   &4   &7  &10\\
$n\log{n}$     &0   &2&8   &64  &896&10240\\
$n^2$     &1   &4&16&256&16384&1048576\\
$2^n$     &2   &4&16&65536&$3.4\times 10^{38}$&$1.8\times 10^{307}$\\
$n!$      &1   &2&24&$2.1\times 10^{13}$&$3.85\times 10^{305}$&$5.4\times10^{2369}$
\end{tabular}
\vskip 1cm The website\\ {\tt
  http://markknowsnothing.com/cgi-bin/calculator.php}\\ was used for
the $2^n$ calculations and\\ {\tt
  http://www.calculatorsoup.com/calculators/discretemathematics/factorials.php}\\ for
the $n!$ calculations; these give answers even when the answer is very
large.

\caption{Different values of $n$ for some functions.  \label{table_n_values}
}
\end{table}

\subsubsection*{I.2.1 Examples: linear and binary search}

Lets do a quick example; searching for the index of an element in a
sorted list. A completely terrible way to do this is linear search,
this is terrible because it doesn't make use of the fact that the list
is sorted. A code listing is given in Table~\ref{c_linear_search}. We
can see straight away that the code between lines 6 and 10 is run $n$
times in the worst case, everything else is run once and so this
algorithm is $O(n)$.

\begin{table}
\begin{lstlisting}[numbers=left]
int search(int a[],int n, int val)
{

  int i;

  for(i=0;i<n;i++)
    {
      if(a[i]==val)
	return i;
    }

  return -1;
}
\end{lstlisting}
\caption{Linear search. This function searches the entries in the array a and returns the index when it finds val, if it doesn't find val it returns -1. The program {\tt linear\_search.c} implements this.\label{c_linear_search}.}
\end{table}

A much better way to search a sorted array is binary search. This is
an example of a \lq{}divide and conquer\rq{} algorithm, many of the
fastest algorithms use divide and conquer. It will be clear to you
that this algorithm would be better written using recursion, this is
typical of divide and conquer, but we haven't looked at analysing
recursion yet. The idea is to divide the array in half and check which
half, the half with bigger numbers or the half with smaller numbers,
the value we are searching for belongs to and to keep doing this,
dividing the remaining part of the array into two parts again and again
until the remaining part of the array that is being search has only
one element. A code listing is given in Table~\ref{c_binary_search}.

\begin{table}
\begin{lstlisting}[numbers=left] 
int search(int a[],int n, int val)
{
  int mid, low=0, high=n-1;

  while(low<=high)
    {
      mid=(low+high)/2;
      if(a[mid]==val)
	return mid;
      else if(val>a[mid])
	low=mid+1;
      else
	high=mid -1;
    }

  return -1;
}
\end{lstlisting}
\caption{Binary search. This function starts in the middle of the array
  and checks if the value there is bigger or smaller than val, if it
  is bigger then it does the same in the top half of the array, if it
  is smaller, in the bottom half and then repeats until there are no
  elements left. The program {\tt binary\_search.c} implements
  this.\label{c_binary_search}.}
\end{table}

This search is extremely fast. There is a chance that a[mid]==val,
after a small number of iterations, indeed, if the middle value of the
array is the search value it will halt after only one
iteration. However, as usual, we assume the worst case, in which case
the algorithm runs to end, dividing the number of elements in half
each time. Ignoring the integer rounding effects, it goes like
Table~\ref{tab_binary_search}. Starting with $n$ states each
subsequent iteration halves the number of states until the last one
when there is one state left. Thus
\begin{equation}
1=\frac{n}{2^{T(n)-1}}
\end{equation}
and taking the log of both sides
\begin{equation}
0=\log_2{n}-(T(n)-1)
\end{equation}
using $\log_2{1}=1$, $\log_2{a^b}=b\log_2{a}$ and $\log_2{2}=1$. Hence
\begin{equation}
T(n)=\log_2{n}+1
\end{equation}
and this algorithm is $O(\log_2{n})$.

\begin{table}
\begin{center}
\begin{tabular}{llllllll}
1&2&3&4&\ldots&k&\ldots&T(n)\\
\hline
$n$&$\frac{n}{2}$&$\frac{n}{4}$&$\frac{n}{8}$&$\ldots$&$\frac{n}{2^{k-1}}$&$\ldots$&1
\end{tabular}
\end{center}
\caption{The number of elements left for binary search.\label{tab_binary_search}}
\end{table}

So, to reiterate; the usual way to examine the behavior of an
algorithm is to look at the worst case run time. This is because the
best case run time is often exceptional, like the one for binary
search if the first guess happens to be correct. The average run time
is often hard to calculate, both because it is often difficult to do
the mathematics and because it would often mean having some
description of how the initial data is distributed. Typically the
worst run time is also \lq{}of the same order\rq{} as the average run
time. We will see an exception to this later on in the case of quick
sort in which the worst case behavior is unusual. The big-Oh notation
is used for describing an algorithm, if the algorithm is said to be
$O(g(n))$ we mean $T(n)\in O(g(n))$ no matter what the initial
condition. Since $O(g(n))$ involves an upper bound $T(n)<cg(n)$ this
makes sense.


\subsubsection*{Other big Letter notations, small oh notation.}

There is another set, $\Omega(g(n))$ with a definition similar to
$O(g(n))$ that is used for describing the best case behavior. This
requires a lower bound rather than an upper bound, so the obvious
definition is
\begin{equation}
\Omega(g(n))=\{f(n)| \exists n_0>0\in {\bf N}\mbox{ and }c>0\in {\bf R}\mbox{ with }|f(n)|\ge c|g(n)|\,\forall n\ge n_0\}
\end{equation}
in other words, the same thing, but with the $\le$ symbol replaced by
a $\ge$. In fact, there is some ambiguity about this definition,
number theorist use a slightly different one. Either way, it isn't
used very often in computer science because algorithms are very
frequently $\Omega(1)$; in the best case scenario the problem is in
some sense already solve, the array already sorted for example, and
the algorithm finishs in one step. 

There is also a set of function that are both bounded above and below
by the same $g(n)$
\begin{equation}
\Theta(g(n))=\Omega(g(n))\cap O(g(n))
\end{equation}
This works because it is possible for
\begin{equation}
c_1 g(n)\le f(n)\le c_2g(n)
\end{equation}
for different $c_1$ and $c_2$. It would be very unusual for this to
apply to an algorithm, it would mean that $T(n)$ has the same behavior
for large $n$ no matter whether it is the best case or the worst case
scenario. There is a na\"ive largest element function in
Table~\ref{c_largest_linear_search} which is $\Theta(n)$. It searches
for the largest value in an unsorted array by looking at each element
in turn. In fact, for a completely unsorted array this is the best
algorithm, but, in practice, if finding the largest element in a set
is an important and frequent procedure, a special data structure,
called a heap, is used to keep track of which element is largest.

\begin{table}
\begin{lstlisting}[numbers=left]
int search(int a[],int n)
{

  int i;
  int best_val=a[0];

  for(i=1;i<n;i++)
    {
      if(a[i]>best_val)
	best_val= a[i];
    }

  return best_val;
}
\end{lstlisting}
\caption{Search for the largest element in an unsorted list. This
  function searches all the elements to see which is the largest, the
  inner loop always runs $n-1$ times since it doesn't know until it
  has looked at every element which is going to be the largest. This
  program is implemented as {\tt
    find\_largest.c}.\label{c_largest_linear_search}.}
\end{table}

Finally, little oh notation is a stricter version of big Oh notation
that is used in some more mathematical context, basically $f(n)\in
o(g(n))$ is $f(n)$ is less than $cg(n)$ for any choice of $c$, if $n$
is large enough:
\begin{equation}
\o(g(n))=\{f(n)| \exists n_0>0\in {\bf N}\mbox{ so that }|f(n)|\ge
c|g(n)|\,\forall n\ge n_0 \mbox{ and }\forall\,c>0\in {\bf R}\}
\end{equation}

\end{document}

\caption{This shows $\log_2{x}$ and $x-1$ plots for $x\in[1,4]$, the one has been taken from $x$ to make them easier to compare, the key point is that the $x$ grows faster.\label{fig_log}}
\end{figure}


\begin{figure}
\documentclass[11pt,a4paper]{scrartcl}
\typearea{12}
\usepackage{graphicx}
\usepackage{pstricks}
\usepackage{listings}
\lstset{language=C}
\pagestyle{headings}
\markright{COMS11600 - Principles of programming I.2}
\begin{document}

\subsection*{I.2 Big Oh notation}

The \lq{}Big Oh\rq{} notation has already been used to describe the
behavior of the running time of insert sort, we said
\begin{equation}
T(n)\in O(n^2)
\end{equation}
Here we want to formalize this notation. Basically $O(n^2)$ is a set
of function, it is all the function which, for large values of $n$ go
to infinity like $n^2$ at the fastest. By saying $T(n)\in O(n^2)$ we
are saying that $T(n)$ is one of these functions, its large $n$
behavior is, as worst, like $n^2$. 

Specifically, the definition for $g(n)$ for all $n$
\begin{equation}
O(g(n))=\{f(n)| \exists n_0>0\in {\bf N}\mbox{ and }c>0\in {\bf R}\mbox{ with }|f(n)|\le c|g(n)|\,\forall n\ge n_0\}
\end{equation}
This definition is quite dense, but we can break it down: it says that
$O(g(n))$ is a set of functions, the curly brackets mean
\lq{}set\rq{}. $f(n)$ is in the set if it has a particular
property: the \lq$|$\rq{} can be read as \lq{}such that\rq{} or
\lq{}with the property that\rq{} and so this is the set of $f(n)$'s
where $f(n)$ has the property on the right of the $|$. Now,
\lq{}$\exists$\rq{} means \lq{}there exists\rq{} and
\lq{}$\forall$\rq{} means \lq{}for all\rq{}, so the defining property
says it is possible to find a positive natural number $n_0$ and a
positive real number $c$ for that if you choose a value of $n$ at
least as big as $n_0$ then $f(n)$ is no bigger than $cg(n)$, ${\bf N}$
and ${\bf R}$ stand for the natural and real numbers. Notice the
absolute value signs, this is about $|f(n)|$ and $|g(n)|$, in fact,
here we are interested in run times, so we will deal with functions
that are non-negative, or are non-negative provided $n$ is larger than
some threshold, for example, $\log_2{n}$ will be important,
$\log_2{n}$ is positive provided $n>1$.

In short, $f(n)$ can do all sorts of crazy stuff for small values of
$n$ but, if you take $n$ large enough, its behavior is bounded by the
behavior of $g(n)$. Now it doesn't say it is bounded by $g(n)$, it is
a statement about the behavior, that's the role of the $c$. If you
know the formal definition of limits you can see that the definition
of $O(g(n))$ has this wrapped up in it, if says
\begin{equation}
f(n)\in O(g(n))\iff \lim_{n\rightarrow \infty}\frac{f(n)}{g(n)}<\infty
\end{equation}

Here are some examples, say 
\begin{equation}
T(n)=5n^2+n+6
\end{equation}
then 
\begin{equation}
T(n)\in O(n^2)
\end{equation}
by, for example, taking $c=5+1+6=12$ then 
\begin{equation}
12n^2\ge 5n^2+n+6
\end{equation}
provided $n\ge 1$ so $n_0=1$ here. This could also be succinctly demonstrated using the limit
\begin{equation}
\lim_{n\rightarrow \infty} \frac{5n^2+n+6}{n^2}=\lim_{n\rightarrow \infty}5+\lim_{n\rightarrow \infty}\frac{1}{n}+\lim_{n\rightarrow \infty}\frac{6}{n^2}=5<\infty
\end{equation}

However, 
\begin{equation}
T(n)\not\in O(n)
\end{equation}
Say we chosen some value $c$ then
\begin{equation}
5n^2+n+6>cn
\end{equation}
for large enough $n$, to check this divide both sides by $n$ so we need to show that $n$ can be chosen so that
\begin{equation}
5n+1+\frac{c}{n}>c
\end{equation}
Since 
\begin{equation}
5n+1+\frac{c}{n}>5n+1
\end{equation}
then, if $n>c/5$
\begin{equation}
5n+1+\frac{c}{n}>5\frac{c}{5}+1=c+1>c
\end{equation}
so, no matter what value of $c$ is chosen, making $n>5/n$ implies
\begin{equation}
5n^2+n+6>cn
\end{equation}
so $5n^2+n+6\not\in O(n)$. Again, the limit does the same job
\begin{equation}
\lim_{n\rightarrow \infty} \frac{5n^2+n+6}{n}=\lim_{n\rightarrow \infty}5n+\lim_{n\rightarrow \infty}1+\lim_{n\rightarrow \infty}\frac{6}{n}=\infty
\end{equation}


In practice, if
\begin{equation}
T(n)=a_rn^r+a_{r-1}n^{r-1}+\ldots+a_1n+a_0
\end{equation}
then $T(n)\in O(n^r)$. 

The logarithm has the funny property that it goes to infinity, but it
does so slower than $n$:
\begin{equation}
\lim_{n\rightarrow \infty}\frac{\log_2{n}}{n}=0
\end{equation}
Here, in line with standard practice in computer science, we are using
the log to the base two, in fact changing bases only causes a change
of an overall constant, see Table~\ref{math_logs} for a reminder of
the properties of the log. The limit of $\log_2{n}/n$ can be calculated
using l'H\^{o}pital's rule, here we'll just look at a plot,
Fig.~\ref{fig_log}. Now, just as $\log_2{n}$ grows very slowly, $2^n$ grows very fast, 
\begin{equation}
\lim_{n\rightarrow \infty}\frac{n^r}{2^n}=0
\end{equation}
for any finite value of $r$, worse still is $n!$, pronounced $n$-factorial
\begin{equation}
n!=n(n-1)(n-2) . . . 1
\end{equation}
If you algorithm is in $O(n!)$ you will probably need a different
algorithm. A table of different values is given as
Table~\ref{table_n_values}, mostly to emphasis how quickly $n!$ gets
big.

Now, in mathematics we call something \lq{}an abuse of notation\rq{}
if it is common to write something that doesn't quite make sense but
acts as a shorthand for something that does. Now $O(g(n))$ is a set of
functions whose large $n$ behavior is bounded by $g(n)$ so in
algorithms, being in $O(g(n))$ is a property of $T(n)$, the formula
for the running time of the algorithm. However, it is a standard abuse
of notation to say an algorithm is $O(g(n))$ for some $g(n)$ is
$T(n)\in O(g(n))$ for all cases. This is another way of saying that
the worst behavior of the algorithm isn't any worse than the behavior
of $g(n)$ for large $n$ so all possible $T(n)$ are elements of
$O(g(n))$.

\begin{table}
The logarithm is the opposite of the exponent: if
\begin{equation}
a^b=c
\end{equation}
then 
\begin{equation}
\log_a{c}=b
\end{equation}
or, written in one line
\begin{equation}
a^{\log_a{c}}=c
\end{equation}
All the laws of logs can be worked out from the laws of
exponents. Hence, since $a^0=1$ we have $\log_a{1}=0$. In a similar
way, other rules of logs can be deduced like
\begin{eqnarray}
\log_a{c_1c_2}&=&\log_a{c_1}+\log_a{c_2}\cr
\log_a{\frac{c_1}{c_2}}&=&\log_a{c_1}-\log_a{c_2}\cr
\log_a c^d&=&d\log_a c
\end{eqnarray}
and so on.

As for the change of base, let $b=\log_{a_1}{c}$ so $a_1^{b}=c$. Now take the log to the base $a_2$ of both sides
\begin{equation}
b\log_{a_2}a_1=\log_{a_2}{c}
\end{equation}
and then solve for $b$
\begin{equation}
b=\frac{\log_{a_2}c}{\log_{a_2}a_1}
\end{equation}
and, substituting back the formula for $b$
\begin{equation}
\log_{a_1}{c}=\frac{\log_{a_2}c}{\log_{a_2}a_1}
\end{equation}
Thus we see, that changing bases is just a matter of a
multiplicative factor. For example, to change from base $e$ to base two
\begin{equation}
\log_{2}{x}=\frac{\log_e{x}}{\log_e{2}}\approx\frac{\log_e{x}}{0.6931}
\end{equation}
Common bases are $\log_2{x}$ used in computer science, $\log_e{x}$
sometimes written $\ln{x}$ used in mathematics and $\log_{10}{x}$ used
in chemistry. The base two is used because of its link to bits and
also, as we will see, because of its relationship with algorithms that
divide data into two piles. The natural log $\ln{x}$ is used where
differential equations are common since
\begin{equation}
\frac{d}{dx}\ln{x}=\frac{1}{x}
\end{equation}
\caption{A reminder about logarithms. This is a quick summary of some of the laws of logs.\label{math_logs}}
\end{table}

\begin{figure}
\include{I.2.log}
\caption{This shows $\log_2{x}$ and $x-1$ plots for $x\in[1,4]$, the one has been taken from $x$ to make them easier to compare, the key point is that the $x$ grows faster.\label{fig_log}}
\end{figure}


\begin{figure}
\include{I.2.exp}
\caption{This shows $2^x$ and $x^2-1$ plots for $x\in[0,6]$, clearly $2^x$ quickly overtakes $x^2+1$, this will happen for any power of $x$. \label{fig_log}}
\end{figure}

\begin{table}
\begin{tabular}{l|cccccc}
        $n$    &1   &2&4   &16  &128&1024\\
$\log{n}$      &0   &1&2   &4   &7  &10\\
$n\log{n}$     &0   &2&8   &64  &896&10240\\
$n^2$     &1   &4&16&256&16384&1048576\\
$2^n$     &2   &4&16&65536&$3.4\times 10^{38}$&$1.8\times 10^{307}$\\
$n!$      &1   &2&24&$2.1\times 10^{13}$&$3.85\times 10^{305}$&$5.4\times10^{2369}$
\end{tabular}
\vskip 1cm The website\\ {\tt
  http://markknowsnothing.com/cgi-bin/calculator.php}\\ was used for
the $2^n$ calculations and\\ {\tt
  http://www.calculatorsoup.com/calculators/discretemathematics/factorials.php}\\ for
the $n!$ calculations; these give answers even when the answer is very
large.

\caption{Different values of $n$ for some functions.  \label{table_n_values}
}
\end{table}

\subsubsection*{I.2.1 Examples: linear and binary search}

Lets do a quick example; searching for the index of an element in a
sorted list. A completely terrible way to do this is linear search,
this is terrible because it doesn't make use of the fact that the list
is sorted. A code listing is given in Table~\ref{c_linear_search}. We
can see straight away that the code between lines 6 and 10 is run $n$
times in the worst case, everything else is run once and so this
algorithm is $O(n)$.

\begin{table}
\begin{lstlisting}[numbers=left]
int search(int a[],int n, int val)
{

  int i;

  for(i=0;i<n;i++)
    {
      if(a[i]==val)
	return i;
    }

  return -1;
}
\end{lstlisting}
\caption{Linear search. This function searches the entries in the array a and returns the index when it finds val, if it doesn't find val it returns -1. The program {\tt linear\_search.c} implements this.\label{c_linear_search}.}
\end{table}

A much better way to search a sorted array is binary search. This is
an example of a \lq{}divide and conquer\rq{} algorithm, many of the
fastest algorithms use divide and conquer. It will be clear to you
that this algorithm would be better written using recursion, this is
typical of divide and conquer, but we haven't looked at analysing
recursion yet. The idea is to divide the array in half and check which
half, the half with bigger numbers or the half with smaller numbers,
the value we are searching for belongs to and to keep doing this,
dividing the remaining part of the array into two parts again and again
until the remaining part of the array that is being search has only
one element. A code listing is given in Table~\ref{c_binary_search}.

\begin{table}
\begin{lstlisting}[numbers=left] 
int search(int a[],int n, int val)
{
  int mid, low=0, high=n-1;

  while(low<=high)
    {
      mid=(low+high)/2;
      if(a[mid]==val)
	return mid;
      else if(val>a[mid])
	low=mid+1;
      else
	high=mid -1;
    }

  return -1;
}
\end{lstlisting}
\caption{Binary search. This function starts in the middle of the array
  and checks if the value there is bigger or smaller than val, if it
  is bigger then it does the same in the top half of the array, if it
  is smaller, in the bottom half and then repeats until there are no
  elements left. The program {\tt binary\_search.c} implements
  this.\label{c_binary_search}.}
\end{table}

This search is extremely fast. There is a chance that a[mid]==val,
after a small number of iterations, indeed, if the middle value of the
array is the search value it will halt after only one
iteration. However, as usual, we assume the worst case, in which case
the algorithm runs to end, dividing the number of elements in half
each time. Ignoring the integer rounding effects, it goes like
Table~\ref{tab_binary_search}. Starting with $n$ states each
subsequent iteration halves the number of states until the last one
when there is one state left. Thus
\begin{equation}
1=\frac{n}{2^{T(n)-1}}
\end{equation}
and taking the log of both sides
\begin{equation}
0=\log_2{n}-(T(n)-1)
\end{equation}
using $\log_2{1}=1$, $\log_2{a^b}=b\log_2{a}$ and $\log_2{2}=1$. Hence
\begin{equation}
T(n)=\log_2{n}+1
\end{equation}
and this algorithm is $O(\log_2{n})$.

\begin{table}
\begin{center}
\begin{tabular}{llllllll}
1&2&3&4&\ldots&k&\ldots&T(n)\\
\hline
$n$&$\frac{n}{2}$&$\frac{n}{4}$&$\frac{n}{8}$&$\ldots$&$\frac{n}{2^{k-1}}$&$\ldots$&1
\end{tabular}
\end{center}
\caption{The number of elements left for binary search.\label{tab_binary_search}}
\end{table}

So, to reiterate; the usual way to examine the behavior of an
algorithm is to look at the worst case run time. This is because the
best case run time is often exceptional, like the one for binary
search if the first guess happens to be correct. The average run time
is often hard to calculate, both because it is often difficult to do
the mathematics and because it would often mean having some
description of how the initial data is distributed. Typically the
worst run time is also \lq{}of the same order\rq{} as the average run
time. We will see an exception to this later on in the case of quick
sort in which the worst case behavior is unusual. The big-Oh notation
is used for describing an algorithm, if the algorithm is said to be
$O(g(n))$ we mean $T(n)\in O(g(n))$ no matter what the initial
condition. Since $O(g(n))$ involves an upper bound $T(n)<cg(n)$ this
makes sense.


\subsubsection*{Other big Letter notations, small oh notation.}

There is another set, $\Omega(g(n))$ with a definition similar to
$O(g(n))$ that is used for describing the best case behavior. This
requires a lower bound rather than an upper bound, so the obvious
definition is
\begin{equation}
\Omega(g(n))=\{f(n)| \exists n_0>0\in {\bf N}\mbox{ and }c>0\in {\bf R}\mbox{ with }|f(n)|\ge c|g(n)|\,\forall n\ge n_0\}
\end{equation}
in other words, the same thing, but with the $\le$ symbol replaced by
a $\ge$. In fact, there is some ambiguity about this definition,
number theorist use a slightly different one. Either way, it isn't
used very often in computer science because algorithms are very
frequently $\Omega(1)$; in the best case scenario the problem is in
some sense already solve, the array already sorted for example, and
the algorithm finishs in one step. 

There is also a set of function that are both bounded above and below
by the same $g(n)$
\begin{equation}
\Theta(g(n))=\Omega(g(n))\cap O(g(n))
\end{equation}
This works because it is possible for
\begin{equation}
c_1 g(n)\le f(n)\le c_2g(n)
\end{equation}
for different $c_1$ and $c_2$. It would be very unusual for this to
apply to an algorithm, it would mean that $T(n)$ has the same behavior
for large $n$ no matter whether it is the best case or the worst case
scenario. There is a na\"ive largest element function in
Table~\ref{c_largest_linear_search} which is $\Theta(n)$. It searches
for the largest value in an unsorted array by looking at each element
in turn. In fact, for a completely unsorted array this is the best
algorithm, but, in practice, if finding the largest element in a set
is an important and frequent procedure, a special data structure,
called a heap, is used to keep track of which element is largest.

\begin{table}
\begin{lstlisting}[numbers=left]
int search(int a[],int n)
{

  int i;
  int best_val=a[0];

  for(i=1;i<n;i++)
    {
      if(a[i]>best_val)
	best_val= a[i];
    }

  return best_val;
}
\end{lstlisting}
\caption{Search for the largest element in an unsorted list. This
  function searches all the elements to see which is the largest, the
  inner loop always runs $n-1$ times since it doesn't know until it
  has looked at every element which is going to be the largest. This
  program is implemented as {\tt
    find\_largest.c}.\label{c_largest_linear_search}.}
\end{table}

Finally, little oh notation is a stricter version of big Oh notation
that is used in some more mathematical context, basically $f(n)\in
o(g(n))$ is $f(n)$ is less than $cg(n)$ for any choice of $c$, if $n$
is large enough:
\begin{equation}
\o(g(n))=\{f(n)| \exists n_0>0\in {\bf N}\mbox{ so that }|f(n)|\ge
c|g(n)|\,\forall n\ge n_0 \mbox{ and }\forall\,c>0\in {\bf R}\}
\end{equation}

\end{document}

\caption{This shows $2^x$ and $x^2-1$ plots for $x\in[0,6]$, clearly $2^x$ quickly overtakes $x^2+1$, this will happen for any power of $x$. \label{fig_log}}
\end{figure}

\begin{table}
\begin{tabular}{l|cccccc}
        $n$    &1   &2&4   &16  &128&1024\\
$\log{n}$      &0   &1&2   &4   &7  &10\\
$n\log{n}$     &0   &2&8   &64  &896&10240\\
$n^2$     &1   &4&16&256&16384&1048576\\
$2^n$     &2   &4&16&65536&$3.4\times 10^{38}$&$1.8\times 10^{307}$\\
$n!$      &1   &2&24&$2.1\times 10^{13}$&$3.85\times 10^{305}$&$5.4\times10^{2369}$
\end{tabular}
\vskip 1cm The website\\ {\tt
  http://markknowsnothing.com/cgi-bin/calculator.php}\\ was used for
the $2^n$ calculations and\\ {\tt
  http://www.calculatorsoup.com/calculators/discretemathematics/factorials.php}\\ for
the $n!$ calculations; these give answers even when the answer is very
large.

\caption{Different values of $n$ for some functions.  \label{table_n_values}
}
\end{table}

\subsubsection*{I.2.1 Examples: linear and binary search}

Lets do a quick example; searching for the index of an element in a
sorted list. A completely terrible way to do this is linear search,
this is terrible because it doesn't make use of the fact that the list
is sorted. A code listing is given in Table~\ref{c_linear_search}. We
can see straight away that the code between lines 6 and 10 is run $n$
times in the worst case, everything else is run once and so this
algorithm is $O(n)$.

\begin{table}
\begin{lstlisting}[numbers=left]
int search(int a[],int n, int val)
{

  int i;

  for(i=0;i<n;i++)
    {
      if(a[i]==val)
	return i;
    }

  return -1;
}
\end{lstlisting}
\caption{Linear search. This function searches the entries in the array a and returns the index when it finds val, if it doesn't find val it returns -1. The program {\tt linear\_search.c} implements this.\label{c_linear_search}.}
\end{table}

A much better way to search a sorted array is binary search. This is
an example of a \lq{}divide and conquer\rq{} algorithm, many of the
fastest algorithms use divide and conquer. It will be clear to you
that this algorithm would be better written using recursion, this is
typical of divide and conquer, but we haven't looked at analysing
recursion yet. The idea is to divide the array in half and check which
half, the half with bigger numbers or the half with smaller numbers,
the value we are searching for belongs to and to keep doing this,
dividing the remaining part of the array into two parts again and again
until the remaining part of the array that is being search has only
one element. A code listing is given in Table~\ref{c_binary_search}.

\begin{table}
\begin{lstlisting}[numbers=left] 
int search(int a[],int n, int val)
{
  int mid, low=0, high=n-1;

  while(low<=high)
    {
      mid=(low+high)/2;
      if(a[mid]==val)
	return mid;
      else if(val>a[mid])
	low=mid+1;
      else
	high=mid -1;
    }

  return -1;
}
\end{lstlisting}
\caption{Binary search. This function starts in the middle of the array
  and checks if the value there is bigger or smaller than val, if it
  is bigger then it does the same in the top half of the array, if it
  is smaller, in the bottom half and then repeats until there are no
  elements left. The program {\tt binary\_search.c} implements
  this.\label{c_binary_search}.}
\end{table}

This search is extremely fast. There is a chance that a[mid]==val,
after a small number of iterations, indeed, if the middle value of the
array is the search value it will halt after only one
iteration. However, as usual, we assume the worst case, in which case
the algorithm runs to end, dividing the number of elements in half
each time. Ignoring the integer rounding effects, it goes like
Table~\ref{tab_binary_search}. Starting with $n$ states each
subsequent iteration halves the number of states until the last one
when there is one state left. Thus
\begin{equation}
1=\frac{n}{2^{T(n)-1}}
\end{equation}
and taking the log of both sides
\begin{equation}
0=\log_2{n}-(T(n)-1)
\end{equation}
using $\log_2{1}=1$, $\log_2{a^b}=b\log_2{a}$ and $\log_2{2}=1$. Hence
\begin{equation}
T(n)=\log_2{n}+1
\end{equation}
and this algorithm is $O(\log_2{n})$.

\begin{table}
\begin{center}
\begin{tabular}{llllllll}
1&2&3&4&\ldots&k&\ldots&T(n)\\
\hline
$n$&$\frac{n}{2}$&$\frac{n}{4}$&$\frac{n}{8}$&$\ldots$&$\frac{n}{2^{k-1}}$&$\ldots$&1
\end{tabular}
\end{center}
\caption{The number of elements left for binary search.\label{tab_binary_search}}
\end{table}

So, to reiterate; the usual way to examine the behavior of an
algorithm is to look at the worst case run time. This is because the
best case run time is often exceptional, like the one for binary
search if the first guess happens to be correct. The average run time
is often hard to calculate, both because it is often difficult to do
the mathematics and because it would often mean having some
description of how the initial data is distributed. Typically the
worst run time is also \lq{}of the same order\rq{} as the average run
time. We will see an exception to this later on in the case of quick
sort in which the worst case behavior is unusual. The big-Oh notation
is used for describing an algorithm, if the algorithm is said to be
$O(g(n))$ we mean $T(n)\in O(g(n))$ no matter what the initial
condition. Since $O(g(n))$ involves an upper bound $T(n)<cg(n)$ this
makes sense.


\subsubsection*{Other big Letter notations, small oh notation.}

There is another set, $\Omega(g(n))$ with a definition similar to
$O(g(n))$ that is used for describing the best case behavior. This
requires a lower bound rather than an upper bound, so the obvious
definition is
\begin{equation}
\Omega(g(n))=\{f(n)| \exists n_0>0\in {\bf N}\mbox{ and }c>0\in {\bf R}\mbox{ with }|f(n)|\ge c|g(n)|\,\forall n\ge n_0\}
\end{equation}
in other words, the same thing, but with the $\le$ symbol replaced by
a $\ge$. In fact, there is some ambiguity about this definition,
number theorist use a slightly different one. Either way, it isn't
used very often in computer science because algorithms are very
frequently $\Omega(1)$; in the best case scenario the problem is in
some sense already solve, the array already sorted for example, and
the algorithm finishs in one step. 

There is also a set of function that are both bounded above and below
by the same $g(n)$
\begin{equation}
\Theta(g(n))=\Omega(g(n))\cap O(g(n))
\end{equation}
This works because it is possible for
\begin{equation}
c_1 g(n)\le f(n)\le c_2g(n)
\end{equation}
for different $c_1$ and $c_2$. It would be very unusual for this to
apply to an algorithm, it would mean that $T(n)$ has the same behavior
for large $n$ no matter whether it is the best case or the worst case
scenario. There is a na\"ive largest element function in
Table~\ref{c_largest_linear_search} which is $\Theta(n)$. It searches
for the largest value in an unsorted array by looking at each element
in turn. In fact, for a completely unsorted array this is the best
algorithm, but, in practice, if finding the largest element in a set
is an important and frequent procedure, a special data structure,
called a heap, is used to keep track of which element is largest.

\begin{table}
\begin{lstlisting}[numbers=left]
int search(int a[],int n)
{

  int i;
  int best_val=a[0];

  for(i=1;i<n;i++)
    {
      if(a[i]>best_val)
	best_val= a[i];
    }

  return best_val;
}
\end{lstlisting}
\caption{Search for the largest element in an unsorted list. This
  function searches all the elements to see which is the largest, the
  inner loop always runs $n-1$ times since it doesn't know until it
  has looked at every element which is going to be the largest. This
  program is implemented as {\tt
    find\_largest.c}.\label{c_largest_linear_search}.}
\end{table}

Finally, little oh notation is a stricter version of big Oh notation
that is used in some more mathematical context, basically $f(n)\in
o(g(n))$ is $f(n)$ is less than $cg(n)$ for any choice of $c$, if $n$
is large enough:
\begin{equation}
\o(g(n))=\{f(n)| \exists n_0>0\in {\bf N}\mbox{ so that }|f(n)|\ge
c|g(n)|\,\forall n\ge n_0 \mbox{ and }\forall\,c>0\in {\bf R}\}
\end{equation}

\end{document}

\caption{This shows $2^x$ and $x^2-1$ plots for $x\in[0,6]$, clearly $2^x$ quickly overtakes $x^2+1$, this will happen for any power of $x$. \label{fig_log}}
\end{figure}

\begin{table}
\begin{tabular}{l|cccccc}
        $n$    &1   &2&4   &16  &128&1024\\
$\log{n}$      &0   &1&2   &4   &7  &10\\
$n\log{n}$     &0   &2&8   &64  &896&10240\\
$n^2$     &1   &4&16&256&16384&1048576\\
$2^n$     &2   &4&16&65536&$3.4\times 10^{38}$&$1.8\times 10^{307}$\\
$n!$      &1   &2&24&$2.1\times 10^{13}$&$3.85\times 10^{305}$&$5.4\times10^{2369}$
\end{tabular}
\vskip 1cm The website\\ {\tt
  http://markknowsnothing.com/cgi-bin/calculator.php}\\ was used for
the $2^n$ calculations and\\ {\tt
  http://www.calculatorsoup.com/calculators/discretemathematics/factorials.php}\\ for
the $n!$ calculations; these give answers even when the answer is very
large.

\caption{Different values of $n$ for some functions.  \label{table_n_values}
}
\end{table}

\subsubsection*{I.2.1 Examples: linear and binary search}

Lets do a quick example; searching for the index of an element in a
sorted list. A completely terrible way to do this is linear search,
this is terrible because it doesn't make use of the fact that the list
is sorted. A code listing is given in Table~\ref{c_linear_search}. We
can see straight away that the code between lines 6 and 10 is run $n$
times in the worst case, everything else is run once and so this
algorithm is $O(n)$.

\begin{table}
\begin{lstlisting}[numbers=left]
int search(int a[],int n, int val)
{

  int i;

  for(i=0;i<n;i++)
    {
      if(a[i]==val)
	return i;
    }

  return -1;
}
\end{lstlisting}
\caption{Linear search. This function searches the entries in the array a and returns the index when it finds val, if it doesn't find val it returns -1. The program {\tt linear\_search.c} implements this.\label{c_linear_search}.}
\end{table}

A much better way to search a sorted array is binary search. This is
an example of a \lq{}divide and conquer\rq{} algorithm, many of the
fastest algorithms use divide and conquer. It will be clear to you
that this algorithm would be better written using recursion, this is
typical of divide and conquer, but we haven't looked at analysing
recursion yet. The idea is to divide the array in half and check which
half, the half with bigger numbers or the half with smaller numbers,
the value we are searching for belongs to and to keep doing this,
dividing the remaining part of the array into two parts again and again
until the remaining part of the array that is being search has only
one element. A code listing is given in Table~\ref{c_binary_search}.

\begin{table}
\begin{lstlisting}[numbers=left] 
int search(int a[],int n, int val)
{
  int mid, low=0, high=n-1;

  while(low<=high)
    {
      mid=(low+high)/2;
      if(a[mid]==val)
	return mid;
      else if(val>a[mid])
	low=mid+1;
      else
	high=mid -1;
    }

  return -1;
}
\end{lstlisting}
\caption{Binary search. This function starts in the middle of the array
  and checks if the value there is bigger or smaller than val, if it
  is bigger then it does the same in the top half of the array, if it
  is smaller, in the bottom half and then repeats until there are no
  elements left. The program {\tt binary\_search.c} implements
  this.\label{c_binary_search}.}
\end{table}

This search is extremely fast. There is a chance that a[mid]==val,
after a small number of iterations, indeed, if the middle value of the
array is the search value it will halt after only one
iteration. However, as usual, we assume the worst case, in which case
the algorithm runs to end, dividing the number of elements in half
each time. Ignoring the integer rounding effects, it goes like
Table~\ref{tab_binary_search}. Starting with $n$ states each
subsequent iteration halves the number of states until the last one
when there is one state left. Thus
\begin{equation}
1=\frac{n}{2^{T(n)-1}}
\end{equation}
and taking the log of both sides
\begin{equation}
0=\log_2{n}-(T(n)-1)
\end{equation}
using $\log_2{1}=1$, $\log_2{a^b}=b\log_2{a}$ and $\log_2{2}=1$. Hence
\begin{equation}
T(n)=\log_2{n}+1
\end{equation}
and this algorithm is $O(\log_2{n})$.

\begin{table}
\begin{center}
\begin{tabular}{llllllll}
1&2&3&4&\ldots&k&\ldots&T(n)\\
\hline
$n$&$\frac{n}{2}$&$\frac{n}{4}$&$\frac{n}{8}$&$\ldots$&$\frac{n}{2^{k-1}}$&$\ldots$&1
\end{tabular}
\end{center}
\caption{The number of elements left for binary search.\label{tab_binary_search}}
\end{table}

So, to reiterate; the usual way to examine the behavior of an
algorithm is to look at the worst case run time. This is because the
best case run time is often exceptional, like the one for binary
search if the first guess happens to be correct. The average run time
is often hard to calculate, both because it is often difficult to do
the mathematics and because it would often mean having some
description of how the initial data is distributed. Typically the
worst run time is also \lq{}of the same order\rq{} as the average run
time. We will see an exception to this later on in the case of quick
sort in which the worst case behavior is unusual. The big-Oh notation
is used for describing an algorithm, if the algorithm is said to be
$O(g(n))$ we mean $T(n)\in O(g(n))$ no matter what the initial
condition. Since $O(g(n))$ involves an upper bound $T(n)<cg(n)$ this
makes sense.


\subsubsection*{Other big Letter notations, small oh notation.}

There is another set, $\Omega(g(n))$ with a definition similar to
$O(g(n))$ that is used for describing the best case behavior. This
requires a lower bound rather than an upper bound, so the obvious
definition is
\begin{equation}
\Omega(g(n))=\{f(n)| \exists n_0>0\in {\bf N}\mbox{ and }c>0\in {\bf R}\mbox{ with }|f(n)|\ge c|g(n)|\,\forall n\ge n_0\}
\end{equation}
in other words, the same thing, but with the $\le$ symbol replaced by
a $\ge$. In fact, there is some ambiguity about this definition,
number theorist use a slightly different one. Either way, it isn't
used very often in computer science because algorithms are very
frequently $\Omega(1)$; in the best case scenario the problem is in
some sense already solve, the array already sorted for example, and
the algorithm finishs in one step. 

There is also a set of function that are both bounded above and below
by the same $g(n)$
\begin{equation}
\Theta(g(n))=\Omega(g(n))\cap O(g(n))
\end{equation}
This works because it is possible for
\begin{equation}
c_1 g(n)\le f(n)\le c_2g(n)
\end{equation}
for different $c_1$ and $c_2$. It would be very unusual for this to
apply to an algorithm, it would mean that $T(n)$ has the same behavior
for large $n$ no matter whether it is the best case or the worst case
scenario. There is a na\"ive largest element function in
Table~\ref{c_largest_linear_search} which is $\Theta(n)$. It searches
for the largest value in an unsorted array by looking at each element
in turn. In fact, for a completely unsorted array this is the best
algorithm, but, in practice, if finding the largest element in a set
is an important and frequent procedure, a special data structure,
called a heap, is used to keep track of which element is largest.

\begin{table}
\begin{lstlisting}[numbers=left]
int search(int a[],int n)
{

  int i;
  int best_val=a[0];

  for(i=1;i<n;i++)
    {
      if(a[i]>best_val)
	best_val= a[i];
    }

  return best_val;
}
\end{lstlisting}
\caption{Search for the largest element in an unsorted list. This
  function searches all the elements to see which is the largest, the
  inner loop always runs $n-1$ times since it doesn't know until it
  has looked at every element which is going to be the largest. This
  program is implemented as {\tt
    find\_largest.c}.\label{c_largest_linear_search}.}
\end{table}

Finally, little oh notation is a stricter version of big Oh notation
that is used in some more mathematical context, basically $f(n)\in
o(g(n))$ is $f(n)$ is less than $cg(n)$ for any choice of $c$, if $n$
is large enough:
\begin{equation}
\o(g(n))=\{f(n)| \exists n_0>0\in {\bf N}\mbox{ so that }|f(n)|\ge
c|g(n)|\,\forall n\ge n_0 \mbox{ and }\forall\,c>0\in {\bf R}\}
\end{equation}

\end{document}

\caption{This shows $\log_2{x}$ and $x-1$ plots for $x\in[1,4]$, the one has been taken from $x$ to make them easier to compare, the key point is that the $x$ grows faster.\label{fig_log}}
\end{figure}


\begin{figure}
\documentclass[11pt,a4paper]{scrartcl}
\typearea{12}
\usepackage{graphicx}
\usepackage{pstricks}
\usepackage{listings}
\lstset{language=C}
\pagestyle{headings}
\markright{COMS11600 - Principles of programming I.2}
\begin{document}

\subsection*{I.2 Big Oh notation}

The \lq{}Big Oh\rq{} notation has already been used to describe the
behavior of the running time of insert sort, we said
\begin{equation}
T(n)\in O(n^2)
\end{equation}
Here we want to formalize this notation. Basically $O(n^2)$ is a set
of function, it is all the function which, for large values of $n$ go
to infinity like $n^2$ at the fastest. By saying $T(n)\in O(n^2)$ we
are saying that $T(n)$ is one of these functions, its large $n$
behavior is, as worst, like $n^2$. 

Specifically, the definition for $g(n)$ for all $n$
\begin{equation}
O(g(n))=\{f(n)| \exists n_0>0\in {\bf N}\mbox{ and }c>0\in {\bf R}\mbox{ with }|f(n)|\le c|g(n)|\,\forall n\ge n_0\}
\end{equation}
This definition is quite dense, but we can break it down: it says that
$O(g(n))$ is a set of functions, the curly brackets mean
\lq{}set\rq{}. $f(n)$ is in the set if it has a particular
property: the \lq$|$\rq{} can be read as \lq{}such that\rq{} or
\lq{}with the property that\rq{} and so this is the set of $f(n)$'s
where $f(n)$ has the property on the right of the $|$. Now,
\lq{}$\exists$\rq{} means \lq{}there exists\rq{} and
\lq{}$\forall$\rq{} means \lq{}for all\rq{}, so the defining property
says it is possible to find a positive natural number $n_0$ and a
positive real number $c$ for that if you choose a value of $n$ at
least as big as $n_0$ then $f(n)$ is no bigger than $cg(n)$, ${\bf N}$
and ${\bf R}$ stand for the natural and real numbers. Notice the
absolute value signs, this is about $|f(n)|$ and $|g(n)|$, in fact,
here we are interested in run times, so we will deal with functions
that are non-negative, or are non-negative provided $n$ is larger than
some threshold, for example, $\log_2{n}$ will be important,
$\log_2{n}$ is positive provided $n>1$.

In short, $f(n)$ can do all sorts of crazy stuff for small values of
$n$ but, if you take $n$ large enough, its behavior is bounded by the
behavior of $g(n)$. Now it doesn't say it is bounded by $g(n)$, it is
a statement about the behavior, that's the role of the $c$. If you
know the formal definition of limits you can see that the definition
of $O(g(n))$ has this wrapped up in it, if says
\begin{equation}
f(n)\in O(g(n))\iff \lim_{n\rightarrow \infty}\frac{f(n)}{g(n)}<\infty
\end{equation}

Here are some examples, say 
\begin{equation}
T(n)=5n^2+n+6
\end{equation}
then 
\begin{equation}
T(n)\in O(n^2)
\end{equation}
by, for example, taking $c=5+1+6=12$ then 
\begin{equation}
12n^2\ge 5n^2+n+6
\end{equation}
provided $n\ge 1$ so $n_0=1$ here. This could also be succinctly demonstrated using the limit
\begin{equation}
\lim_{n\rightarrow \infty} \frac{5n^2+n+6}{n^2}=\lim_{n\rightarrow \infty}5+\lim_{n\rightarrow \infty}\frac{1}{n}+\lim_{n\rightarrow \infty}\frac{6}{n^2}=5<\infty
\end{equation}

However, 
\begin{equation}
T(n)\not\in O(n)
\end{equation}
Say we chosen some value $c$ then
\begin{equation}
5n^2+n+6>cn
\end{equation}
for large enough $n$, to check this divide both sides by $n$ so we need to show that $n$ can be chosen so that
\begin{equation}
5n+1+\frac{c}{n}>c
\end{equation}
Since 
\begin{equation}
5n+1+\frac{c}{n}>5n+1
\end{equation}
then, if $n>c/5$
\begin{equation}
5n+1+\frac{c}{n}>5\frac{c}{5}+1=c+1>c
\end{equation}
so, no matter what value of $c$ is chosen, making $n>5/n$ implies
\begin{equation}
5n^2+n+6>cn
\end{equation}
so $5n^2+n+6\not\in O(n)$. Again, the limit does the same job
\begin{equation}
\lim_{n\rightarrow \infty} \frac{5n^2+n+6}{n}=\lim_{n\rightarrow \infty}5n+\lim_{n\rightarrow \infty}1+\lim_{n\rightarrow \infty}\frac{6}{n}=\infty
\end{equation}


In practice, if
\begin{equation}
T(n)=a_rn^r+a_{r-1}n^{r-1}+\ldots+a_1n+a_0
\end{equation}
then $T(n)\in O(n^r)$. 

The logarithm has the funny property that it goes to infinity, but it
does so slower than $n$:
\begin{equation}
\lim_{n\rightarrow \infty}\frac{\log_2{n}}{n}=0
\end{equation}
Here, in line with standard practice in computer science, we are using
the log to the base two, in fact changing bases only causes a change
of an overall constant, see Table~\ref{math_logs} for a reminder of
the properties of the log. The limit of $\log_2{n}/n$ can be calculated
using l'H\^{o}pital's rule, here we'll just look at a plot,
Fig.~\ref{fig_log}. Now, just as $\log_2{n}$ grows very slowly, $2^n$ grows very fast, 
\begin{equation}
\lim_{n\rightarrow \infty}\frac{n^r}{2^n}=0
\end{equation}
for any finite value of $r$, worse still is $n!$, pronounced $n$-factorial
\begin{equation}
n!=n(n-1)(n-2) . . . 1
\end{equation}
If you algorithm is in $O(n!)$ you will probably need a different
algorithm. A table of different values is given as
Table~\ref{table_n_values}, mostly to emphasis how quickly $n!$ gets
big.

Now, in mathematics we call something \lq{}an abuse of notation\rq{}
if it is common to write something that doesn't quite make sense but
acts as a shorthand for something that does. Now $O(g(n))$ is a set of
functions whose large $n$ behavior is bounded by $g(n)$ so in
algorithms, being in $O(g(n))$ is a property of $T(n)$, the formula
for the running time of the algorithm. However, it is a standard abuse
of notation to say an algorithm is $O(g(n))$ for some $g(n)$ is
$T(n)\in O(g(n))$ for all cases. This is another way of saying that
the worst behavior of the algorithm isn't any worse than the behavior
of $g(n)$ for large $n$ so all possible $T(n)$ are elements of
$O(g(n))$.

\begin{table}
The logarithm is the opposite of the exponent: if
\begin{equation}
a^b=c
\end{equation}
then 
\begin{equation}
\log_a{c}=b
\end{equation}
or, written in one line
\begin{equation}
a^{\log_a{c}}=c
\end{equation}
All the laws of logs can be worked out from the laws of
exponents. Hence, since $a^0=1$ we have $\log_a{1}=0$. In a similar
way, other rules of logs can be deduced like
\begin{eqnarray}
\log_a{c_1c_2}&=&\log_a{c_1}+\log_a{c_2}\cr
\log_a{\frac{c_1}{c_2}}&=&\log_a{c_1}-\log_a{c_2}\cr
\log_a c^d&=&d\log_a c
\end{eqnarray}
and so on.

As for the change of base, let $b=\log_{a_1}{c}$ so $a_1^{b}=c$. Now take the log to the base $a_2$ of both sides
\begin{equation}
b\log_{a_2}a_1=\log_{a_2}{c}
\end{equation}
and then solve for $b$
\begin{equation}
b=\frac{\log_{a_2}c}{\log_{a_2}a_1}
\end{equation}
and, substituting back the formula for $b$
\begin{equation}
\log_{a_1}{c}=\frac{\log_{a_2}c}{\log_{a_2}a_1}
\end{equation}
Thus we see, that changing bases is just a matter of a
multiplicative factor. For example, to change from base $e$ to base two
\begin{equation}
\log_{2}{x}=\frac{\log_e{x}}{\log_e{2}}\approx\frac{\log_e{x}}{0.6931}
\end{equation}
Common bases are $\log_2{x}$ used in computer science, $\log_e{x}$
sometimes written $\ln{x}$ used in mathematics and $\log_{10}{x}$ used
in chemistry. The base two is used because of its link to bits and
also, as we will see, because of its relationship with algorithms that
divide data into two piles. The natural log $\ln{x}$ is used where
differential equations are common since
\begin{equation}
\frac{d}{dx}\ln{x}=\frac{1}{x}
\end{equation}
\caption{A reminder about logarithms. This is a quick summary of some of the laws of logs.\label{math_logs}}
\end{table}

\begin{figure}
\documentclass[11pt,a4paper]{scrartcl}
\typearea{12}
\usepackage{graphicx}
\usepackage{pstricks}
\usepackage{listings}
\lstset{language=C}
\pagestyle{headings}
\markright{COMS11600 - Principles of programming I.2}
\begin{document}

\subsection*{I.2 Big Oh notation}

The \lq{}Big Oh\rq{} notation has already been used to describe the
behavior of the running time of insert sort, we said
\begin{equation}
T(n)\in O(n^2)
\end{equation}
Here we want to formalize this notation. Basically $O(n^2)$ is a set
of function, it is all the function which, for large values of $n$ go
to infinity like $n^2$ at the fastest. By saying $T(n)\in O(n^2)$ we
are saying that $T(n)$ is one of these functions, its large $n$
behavior is, as worst, like $n^2$. 

Specifically, the definition for $g(n)$ for all $n$
\begin{equation}
O(g(n))=\{f(n)| \exists n_0>0\in {\bf N}\mbox{ and }c>0\in {\bf R}\mbox{ with }|f(n)|\le c|g(n)|\,\forall n\ge n_0\}
\end{equation}
This definition is quite dense, but we can break it down: it says that
$O(g(n))$ is a set of functions, the curly brackets mean
\lq{}set\rq{}. $f(n)$ is in the set if it has a particular
property: the \lq$|$\rq{} can be read as \lq{}such that\rq{} or
\lq{}with the property that\rq{} and so this is the set of $f(n)$'s
where $f(n)$ has the property on the right of the $|$. Now,
\lq{}$\exists$\rq{} means \lq{}there exists\rq{} and
\lq{}$\forall$\rq{} means \lq{}for all\rq{}, so the defining property
says it is possible to find a positive natural number $n_0$ and a
positive real number $c$ for that if you choose a value of $n$ at
least as big as $n_0$ then $f(n)$ is no bigger than $cg(n)$, ${\bf N}$
and ${\bf R}$ stand for the natural and real numbers. Notice the
absolute value signs, this is about $|f(n)|$ and $|g(n)|$, in fact,
here we are interested in run times, so we will deal with functions
that are non-negative, or are non-negative provided $n$ is larger than
some threshold, for example, $\log_2{n}$ will be important,
$\log_2{n}$ is positive provided $n>1$.

In short, $f(n)$ can do all sorts of crazy stuff for small values of
$n$ but, if you take $n$ large enough, its behavior is bounded by the
behavior of $g(n)$. Now it doesn't say it is bounded by $g(n)$, it is
a statement about the behavior, that's the role of the $c$. If you
know the formal definition of limits you can see that the definition
of $O(g(n))$ has this wrapped up in it, if says
\begin{equation}
f(n)\in O(g(n))\iff \lim_{n\rightarrow \infty}\frac{f(n)}{g(n)}<\infty
\end{equation}

Here are some examples, say 
\begin{equation}
T(n)=5n^2+n+6
\end{equation}
then 
\begin{equation}
T(n)\in O(n^2)
\end{equation}
by, for example, taking $c=5+1+6=12$ then 
\begin{equation}
12n^2\ge 5n^2+n+6
\end{equation}
provided $n\ge 1$ so $n_0=1$ here. This could also be succinctly demonstrated using the limit
\begin{equation}
\lim_{n\rightarrow \infty} \frac{5n^2+n+6}{n^2}=\lim_{n\rightarrow \infty}5+\lim_{n\rightarrow \infty}\frac{1}{n}+\lim_{n\rightarrow \infty}\frac{6}{n^2}=5<\infty
\end{equation}

However, 
\begin{equation}
T(n)\not\in O(n)
\end{equation}
Say we chosen some value $c$ then
\begin{equation}
5n^2+n+6>cn
\end{equation}
for large enough $n$, to check this divide both sides by $n$ so we need to show that $n$ can be chosen so that
\begin{equation}
5n+1+\frac{c}{n}>c
\end{equation}
Since 
\begin{equation}
5n+1+\frac{c}{n}>5n+1
\end{equation}
then, if $n>c/5$
\begin{equation}
5n+1+\frac{c}{n}>5\frac{c}{5}+1=c+1>c
\end{equation}
so, no matter what value of $c$ is chosen, making $n>5/n$ implies
\begin{equation}
5n^2+n+6>cn
\end{equation}
so $5n^2+n+6\not\in O(n)$. Again, the limit does the same job
\begin{equation}
\lim_{n\rightarrow \infty} \frac{5n^2+n+6}{n}=\lim_{n\rightarrow \infty}5n+\lim_{n\rightarrow \infty}1+\lim_{n\rightarrow \infty}\frac{6}{n}=\infty
\end{equation}


In practice, if
\begin{equation}
T(n)=a_rn^r+a_{r-1}n^{r-1}+\ldots+a_1n+a_0
\end{equation}
then $T(n)\in O(n^r)$. 

The logarithm has the funny property that it goes to infinity, but it
does so slower than $n$:
\begin{equation}
\lim_{n\rightarrow \infty}\frac{\log_2{n}}{n}=0
\end{equation}
Here, in line with standard practice in computer science, we are using
the log to the base two, in fact changing bases only causes a change
of an overall constant, see Table~\ref{math_logs} for a reminder of
the properties of the log. The limit of $\log_2{n}/n$ can be calculated
using l'H\^{o}pital's rule, here we'll just look at a plot,
Fig.~\ref{fig_log}. Now, just as $\log_2{n}$ grows very slowly, $2^n$ grows very fast, 
\begin{equation}
\lim_{n\rightarrow \infty}\frac{n^r}{2^n}=0
\end{equation}
for any finite value of $r$, worse still is $n!$, pronounced $n$-factorial
\begin{equation}
n!=n(n-1)(n-2) . . . 1
\end{equation}
If you algorithm is in $O(n!)$ you will probably need a different
algorithm. A table of different values is given as
Table~\ref{table_n_values}, mostly to emphasis how quickly $n!$ gets
big.

Now, in mathematics we call something \lq{}an abuse of notation\rq{}
if it is common to write something that doesn't quite make sense but
acts as a shorthand for something that does. Now $O(g(n))$ is a set of
functions whose large $n$ behavior is bounded by $g(n)$ so in
algorithms, being in $O(g(n))$ is a property of $T(n)$, the formula
for the running time of the algorithm. However, it is a standard abuse
of notation to say an algorithm is $O(g(n))$ for some $g(n)$ is
$T(n)\in O(g(n))$ for all cases. This is another way of saying that
the worst behavior of the algorithm isn't any worse than the behavior
of $g(n)$ for large $n$ so all possible $T(n)$ are elements of
$O(g(n))$.

\begin{table}
The logarithm is the opposite of the exponent: if
\begin{equation}
a^b=c
\end{equation}
then 
\begin{equation}
\log_a{c}=b
\end{equation}
or, written in one line
\begin{equation}
a^{\log_a{c}}=c
\end{equation}
All the laws of logs can be worked out from the laws of
exponents. Hence, since $a^0=1$ we have $\log_a{1}=0$. In a similar
way, other rules of logs can be deduced like
\begin{eqnarray}
\log_a{c_1c_2}&=&\log_a{c_1}+\log_a{c_2}\cr
\log_a{\frac{c_1}{c_2}}&=&\log_a{c_1}-\log_a{c_2}\cr
\log_a c^d&=&d\log_a c
\end{eqnarray}
and so on.

As for the change of base, let $b=\log_{a_1}{c}$ so $a_1^{b}=c$. Now take the log to the base $a_2$ of both sides
\begin{equation}
b\log_{a_2}a_1=\log_{a_2}{c}
\end{equation}
and then solve for $b$
\begin{equation}
b=\frac{\log_{a_2}c}{\log_{a_2}a_1}
\end{equation}
and, substituting back the formula for $b$
\begin{equation}
\log_{a_1}{c}=\frac{\log_{a_2}c}{\log_{a_2}a_1}
\end{equation}
Thus we see, that changing bases is just a matter of a
multiplicative factor. For example, to change from base $e$ to base two
\begin{equation}
\log_{2}{x}=\frac{\log_e{x}}{\log_e{2}}\approx\frac{\log_e{x}}{0.6931}
\end{equation}
Common bases are $\log_2{x}$ used in computer science, $\log_e{x}$
sometimes written $\ln{x}$ used in mathematics and $\log_{10}{x}$ used
in chemistry. The base two is used because of its link to bits and
also, as we will see, because of its relationship with algorithms that
divide data into two piles. The natural log $\ln{x}$ is used where
differential equations are common since
\begin{equation}
\frac{d}{dx}\ln{x}=\frac{1}{x}
\end{equation}
\caption{A reminder about logarithms. This is a quick summary of some of the laws of logs.\label{math_logs}}
\end{table}

\begin{figure}
\documentclass[11pt,a4paper]{scrartcl}
\typearea{12}
\usepackage{graphicx}
\usepackage{pstricks}
\usepackage{listings}
\lstset{language=C}
\pagestyle{headings}
\markright{COMS11600 - Principles of programming I.2}
\begin{document}

\subsection*{I.2 Big Oh notation}

The \lq{}Big Oh\rq{} notation has already been used to describe the
behavior of the running time of insert sort, we said
\begin{equation}
T(n)\in O(n^2)
\end{equation}
Here we want to formalize this notation. Basically $O(n^2)$ is a set
of function, it is all the function which, for large values of $n$ go
to infinity like $n^2$ at the fastest. By saying $T(n)\in O(n^2)$ we
are saying that $T(n)$ is one of these functions, its large $n$
behavior is, as worst, like $n^2$. 

Specifically, the definition for $g(n)$ for all $n$
\begin{equation}
O(g(n))=\{f(n)| \exists n_0>0\in {\bf N}\mbox{ and }c>0\in {\bf R}\mbox{ with }|f(n)|\le c|g(n)|\,\forall n\ge n_0\}
\end{equation}
This definition is quite dense, but we can break it down: it says that
$O(g(n))$ is a set of functions, the curly brackets mean
\lq{}set\rq{}. $f(n)$ is in the set if it has a particular
property: the \lq$|$\rq{} can be read as \lq{}such that\rq{} or
\lq{}with the property that\rq{} and so this is the set of $f(n)$'s
where $f(n)$ has the property on the right of the $|$. Now,
\lq{}$\exists$\rq{} means \lq{}there exists\rq{} and
\lq{}$\forall$\rq{} means \lq{}for all\rq{}, so the defining property
says it is possible to find a positive natural number $n_0$ and a
positive real number $c$ for that if you choose a value of $n$ at
least as big as $n_0$ then $f(n)$ is no bigger than $cg(n)$, ${\bf N}$
and ${\bf R}$ stand for the natural and real numbers. Notice the
absolute value signs, this is about $|f(n)|$ and $|g(n)|$, in fact,
here we are interested in run times, so we will deal with functions
that are non-negative, or are non-negative provided $n$ is larger than
some threshold, for example, $\log_2{n}$ will be important,
$\log_2{n}$ is positive provided $n>1$.

In short, $f(n)$ can do all sorts of crazy stuff for small values of
$n$ but, if you take $n$ large enough, its behavior is bounded by the
behavior of $g(n)$. Now it doesn't say it is bounded by $g(n)$, it is
a statement about the behavior, that's the role of the $c$. If you
know the formal definition of limits you can see that the definition
of $O(g(n))$ has this wrapped up in it, if says
\begin{equation}
f(n)\in O(g(n))\iff \lim_{n\rightarrow \infty}\frac{f(n)}{g(n)}<\infty
\end{equation}

Here are some examples, say 
\begin{equation}
T(n)=5n^2+n+6
\end{equation}
then 
\begin{equation}
T(n)\in O(n^2)
\end{equation}
by, for example, taking $c=5+1+6=12$ then 
\begin{equation}
12n^2\ge 5n^2+n+6
\end{equation}
provided $n\ge 1$ so $n_0=1$ here. This could also be succinctly demonstrated using the limit
\begin{equation}
\lim_{n\rightarrow \infty} \frac{5n^2+n+6}{n^2}=\lim_{n\rightarrow \infty}5+\lim_{n\rightarrow \infty}\frac{1}{n}+\lim_{n\rightarrow \infty}\frac{6}{n^2}=5<\infty
\end{equation}

However, 
\begin{equation}
T(n)\not\in O(n)
\end{equation}
Say we chosen some value $c$ then
\begin{equation}
5n^2+n+6>cn
\end{equation}
for large enough $n$, to check this divide both sides by $n$ so we need to show that $n$ can be chosen so that
\begin{equation}
5n+1+\frac{c}{n}>c
\end{equation}
Since 
\begin{equation}
5n+1+\frac{c}{n}>5n+1
\end{equation}
then, if $n>c/5$
\begin{equation}
5n+1+\frac{c}{n}>5\frac{c}{5}+1=c+1>c
\end{equation}
so, no matter what value of $c$ is chosen, making $n>5/n$ implies
\begin{equation}
5n^2+n+6>cn
\end{equation}
so $5n^2+n+6\not\in O(n)$. Again, the limit does the same job
\begin{equation}
\lim_{n\rightarrow \infty} \frac{5n^2+n+6}{n}=\lim_{n\rightarrow \infty}5n+\lim_{n\rightarrow \infty}1+\lim_{n\rightarrow \infty}\frac{6}{n}=\infty
\end{equation}


In practice, if
\begin{equation}
T(n)=a_rn^r+a_{r-1}n^{r-1}+\ldots+a_1n+a_0
\end{equation}
then $T(n)\in O(n^r)$. 

The logarithm has the funny property that it goes to infinity, but it
does so slower than $n$:
\begin{equation}
\lim_{n\rightarrow \infty}\frac{\log_2{n}}{n}=0
\end{equation}
Here, in line with standard practice in computer science, we are using
the log to the base two, in fact changing bases only causes a change
of an overall constant, see Table~\ref{math_logs} for a reminder of
the properties of the log. The limit of $\log_2{n}/n$ can be calculated
using l'H\^{o}pital's rule, here we'll just look at a plot,
Fig.~\ref{fig_log}. Now, just as $\log_2{n}$ grows very slowly, $2^n$ grows very fast, 
\begin{equation}
\lim_{n\rightarrow \infty}\frac{n^r}{2^n}=0
\end{equation}
for any finite value of $r$, worse still is $n!$, pronounced $n$-factorial
\begin{equation}
n!=n(n-1)(n-2) . . . 1
\end{equation}
If you algorithm is in $O(n!)$ you will probably need a different
algorithm. A table of different values is given as
Table~\ref{table_n_values}, mostly to emphasis how quickly $n!$ gets
big.

Now, in mathematics we call something \lq{}an abuse of notation\rq{}
if it is common to write something that doesn't quite make sense but
acts as a shorthand for something that does. Now $O(g(n))$ is a set of
functions whose large $n$ behavior is bounded by $g(n)$ so in
algorithms, being in $O(g(n))$ is a property of $T(n)$, the formula
for the running time of the algorithm. However, it is a standard abuse
of notation to say an algorithm is $O(g(n))$ for some $g(n)$ is
$T(n)\in O(g(n))$ for all cases. This is another way of saying that
the worst behavior of the algorithm isn't any worse than the behavior
of $g(n)$ for large $n$ so all possible $T(n)$ are elements of
$O(g(n))$.

\begin{table}
The logarithm is the opposite of the exponent: if
\begin{equation}
a^b=c
\end{equation}
then 
\begin{equation}
\log_a{c}=b
\end{equation}
or, written in one line
\begin{equation}
a^{\log_a{c}}=c
\end{equation}
All the laws of logs can be worked out from the laws of
exponents. Hence, since $a^0=1$ we have $\log_a{1}=0$. In a similar
way, other rules of logs can be deduced like
\begin{eqnarray}
\log_a{c_1c_2}&=&\log_a{c_1}+\log_a{c_2}\cr
\log_a{\frac{c_1}{c_2}}&=&\log_a{c_1}-\log_a{c_2}\cr
\log_a c^d&=&d\log_a c
\end{eqnarray}
and so on.

As for the change of base, let $b=\log_{a_1}{c}$ so $a_1^{b}=c$. Now take the log to the base $a_2$ of both sides
\begin{equation}
b\log_{a_2}a_1=\log_{a_2}{c}
\end{equation}
and then solve for $b$
\begin{equation}
b=\frac{\log_{a_2}c}{\log_{a_2}a_1}
\end{equation}
and, substituting back the formula for $b$
\begin{equation}
\log_{a_1}{c}=\frac{\log_{a_2}c}{\log_{a_2}a_1}
\end{equation}
Thus we see, that changing bases is just a matter of a
multiplicative factor. For example, to change from base $e$ to base two
\begin{equation}
\log_{2}{x}=\frac{\log_e{x}}{\log_e{2}}\approx\frac{\log_e{x}}{0.6931}
\end{equation}
Common bases are $\log_2{x}$ used in computer science, $\log_e{x}$
sometimes written $\ln{x}$ used in mathematics and $\log_{10}{x}$ used
in chemistry. The base two is used because of its link to bits and
also, as we will see, because of its relationship with algorithms that
divide data into two piles. The natural log $\ln{x}$ is used where
differential equations are common since
\begin{equation}
\frac{d}{dx}\ln{x}=\frac{1}{x}
\end{equation}
\caption{A reminder about logarithms. This is a quick summary of some of the laws of logs.\label{math_logs}}
\end{table}

\begin{figure}
\include{I.2.log}
\caption{This shows $\log_2{x}$ and $x-1$ plots for $x\in[1,4]$, the one has been taken from $x$ to make them easier to compare, the key point is that the $x$ grows faster.\label{fig_log}}
\end{figure}


\begin{figure}
\include{I.2.exp}
\caption{This shows $2^x$ and $x^2-1$ plots for $x\in[0,6]$, clearly $2^x$ quickly overtakes $x^2+1$, this will happen for any power of $x$. \label{fig_log}}
\end{figure}

\begin{table}
\begin{tabular}{l|cccccc}
        $n$    &1   &2&4   &16  &128&1024\\
$\log{n}$      &0   &1&2   &4   &7  &10\\
$n\log{n}$     &0   &2&8   &64  &896&10240\\
$n^2$     &1   &4&16&256&16384&1048576\\
$2^n$     &2   &4&16&65536&$3.4\times 10^{38}$&$1.8\times 10^{307}$\\
$n!$      &1   &2&24&$2.1\times 10^{13}$&$3.85\times 10^{305}$&$5.4\times10^{2369}$
\end{tabular}
\vskip 1cm The website\\ {\tt
  http://markknowsnothing.com/cgi-bin/calculator.php}\\ was used for
the $2^n$ calculations and\\ {\tt
  http://www.calculatorsoup.com/calculators/discretemathematics/factorials.php}\\ for
the $n!$ calculations; these give answers even when the answer is very
large.

\caption{Different values of $n$ for some functions.  \label{table_n_values}
}
\end{table}

\subsubsection*{I.2.1 Examples: linear and binary search}

Lets do a quick example; searching for the index of an element in a
sorted list. A completely terrible way to do this is linear search,
this is terrible because it doesn't make use of the fact that the list
is sorted. A code listing is given in Table~\ref{c_linear_search}. We
can see straight away that the code between lines 6 and 10 is run $n$
times in the worst case, everything else is run once and so this
algorithm is $O(n)$.

\begin{table}
\begin{lstlisting}[numbers=left]
int search(int a[],int n, int val)
{

  int i;

  for(i=0;i<n;i++)
    {
      if(a[i]==val)
	return i;
    }

  return -1;
}
\end{lstlisting}
\caption{Linear search. This function searches the entries in the array a and returns the index when it finds val, if it doesn't find val it returns -1. The program {\tt linear\_search.c} implements this.\label{c_linear_search}.}
\end{table}

A much better way to search a sorted array is binary search. This is
an example of a \lq{}divide and conquer\rq{} algorithm, many of the
fastest algorithms use divide and conquer. It will be clear to you
that this algorithm would be better written using recursion, this is
typical of divide and conquer, but we haven't looked at analysing
recursion yet. The idea is to divide the array in half and check which
half, the half with bigger numbers or the half with smaller numbers,
the value we are searching for belongs to and to keep doing this,
dividing the remaining part of the array into two parts again and again
until the remaining part of the array that is being search has only
one element. A code listing is given in Table~\ref{c_binary_search}.

\begin{table}
\begin{lstlisting}[numbers=left] 
int search(int a[],int n, int val)
{
  int mid, low=0, high=n-1;

  while(low<=high)
    {
      mid=(low+high)/2;
      if(a[mid]==val)
	return mid;
      else if(val>a[mid])
	low=mid+1;
      else
	high=mid -1;
    }

  return -1;
}
\end{lstlisting}
\caption{Binary search. This function starts in the middle of the array
  and checks if the value there is bigger or smaller than val, if it
  is bigger then it does the same in the top half of the array, if it
  is smaller, in the bottom half and then repeats until there are no
  elements left. The program {\tt binary\_search.c} implements
  this.\label{c_binary_search}.}
\end{table}

This search is extremely fast. There is a chance that a[mid]==val,
after a small number of iterations, indeed, if the middle value of the
array is the search value it will halt after only one
iteration. However, as usual, we assume the worst case, in which case
the algorithm runs to end, dividing the number of elements in half
each time. Ignoring the integer rounding effects, it goes like
Table~\ref{tab_binary_search}. Starting with $n$ states each
subsequent iteration halves the number of states until the last one
when there is one state left. Thus
\begin{equation}
1=\frac{n}{2^{T(n)-1}}
\end{equation}
and taking the log of both sides
\begin{equation}
0=\log_2{n}-(T(n)-1)
\end{equation}
using $\log_2{1}=1$, $\log_2{a^b}=b\log_2{a}$ and $\log_2{2}=1$. Hence
\begin{equation}
T(n)=\log_2{n}+1
\end{equation}
and this algorithm is $O(\log_2{n})$.

\begin{table}
\begin{center}
\begin{tabular}{llllllll}
1&2&3&4&\ldots&k&\ldots&T(n)\\
\hline
$n$&$\frac{n}{2}$&$\frac{n}{4}$&$\frac{n}{8}$&$\ldots$&$\frac{n}{2^{k-1}}$&$\ldots$&1
\end{tabular}
\end{center}
\caption{The number of elements left for binary search.\label{tab_binary_search}}
\end{table}

So, to reiterate; the usual way to examine the behavior of an
algorithm is to look at the worst case run time. This is because the
best case run time is often exceptional, like the one for binary
search if the first guess happens to be correct. The average run time
is often hard to calculate, both because it is often difficult to do
the mathematics and because it would often mean having some
description of how the initial data is distributed. Typically the
worst run time is also \lq{}of the same order\rq{} as the average run
time. We will see an exception to this later on in the case of quick
sort in which the worst case behavior is unusual. The big-Oh notation
is used for describing an algorithm, if the algorithm is said to be
$O(g(n))$ we mean $T(n)\in O(g(n))$ no matter what the initial
condition. Since $O(g(n))$ involves an upper bound $T(n)<cg(n)$ this
makes sense.


\subsubsection*{Other big Letter notations, small oh notation.}

There is another set, $\Omega(g(n))$ with a definition similar to
$O(g(n))$ that is used for describing the best case behavior. This
requires a lower bound rather than an upper bound, so the obvious
definition is
\begin{equation}
\Omega(g(n))=\{f(n)| \exists n_0>0\in {\bf N}\mbox{ and }c>0\in {\bf R}\mbox{ with }|f(n)|\ge c|g(n)|\,\forall n\ge n_0\}
\end{equation}
in other words, the same thing, but with the $\le$ symbol replaced by
a $\ge$. In fact, there is some ambiguity about this definition,
number theorist use a slightly different one. Either way, it isn't
used very often in computer science because algorithms are very
frequently $\Omega(1)$; in the best case scenario the problem is in
some sense already solve, the array already sorted for example, and
the algorithm finishs in one step. 

There is also a set of function that are both bounded above and below
by the same $g(n)$
\begin{equation}
\Theta(g(n))=\Omega(g(n))\cap O(g(n))
\end{equation}
This works because it is possible for
\begin{equation}
c_1 g(n)\le f(n)\le c_2g(n)
\end{equation}
for different $c_1$ and $c_2$. It would be very unusual for this to
apply to an algorithm, it would mean that $T(n)$ has the same behavior
for large $n$ no matter whether it is the best case or the worst case
scenario. There is a na\"ive largest element function in
Table~\ref{c_largest_linear_search} which is $\Theta(n)$. It searches
for the largest value in an unsorted array by looking at each element
in turn. In fact, for a completely unsorted array this is the best
algorithm, but, in practice, if finding the largest element in a set
is an important and frequent procedure, a special data structure,
called a heap, is used to keep track of which element is largest.

\begin{table}
\begin{lstlisting}[numbers=left]
int search(int a[],int n)
{

  int i;
  int best_val=a[0];

  for(i=1;i<n;i++)
    {
      if(a[i]>best_val)
	best_val= a[i];
    }

  return best_val;
}
\end{lstlisting}
\caption{Search for the largest element in an unsorted list. This
  function searches all the elements to see which is the largest, the
  inner loop always runs $n-1$ times since it doesn't know until it
  has looked at every element which is going to be the largest. This
  program is implemented as {\tt
    find\_largest.c}.\label{c_largest_linear_search}.}
\end{table}

Finally, little oh notation is a stricter version of big Oh notation
that is used in some more mathematical context, basically $f(n)\in
o(g(n))$ is $f(n)$ is less than $cg(n)$ for any choice of $c$, if $n$
is large enough:
\begin{equation}
\o(g(n))=\{f(n)| \exists n_0>0\in {\bf N}\mbox{ so that }|f(n)|\ge
c|g(n)|\,\forall n\ge n_0 \mbox{ and }\forall\,c>0\in {\bf R}\}
\end{equation}

\end{document}

\caption{This shows $\log_2{x}$ and $x-1$ plots for $x\in[1,4]$, the one has been taken from $x$ to make them easier to compare, the key point is that the $x$ grows faster.\label{fig_log}}
\end{figure}


\begin{figure}
\documentclass[11pt,a4paper]{scrartcl}
\typearea{12}
\usepackage{graphicx}
\usepackage{pstricks}
\usepackage{listings}
\lstset{language=C}
\pagestyle{headings}
\markright{COMS11600 - Principles of programming I.2}
\begin{document}

\subsection*{I.2 Big Oh notation}

The \lq{}Big Oh\rq{} notation has already been used to describe the
behavior of the running time of insert sort, we said
\begin{equation}
T(n)\in O(n^2)
\end{equation}
Here we want to formalize this notation. Basically $O(n^2)$ is a set
of function, it is all the function which, for large values of $n$ go
to infinity like $n^2$ at the fastest. By saying $T(n)\in O(n^2)$ we
are saying that $T(n)$ is one of these functions, its large $n$
behavior is, as worst, like $n^2$. 

Specifically, the definition for $g(n)$ for all $n$
\begin{equation}
O(g(n))=\{f(n)| \exists n_0>0\in {\bf N}\mbox{ and }c>0\in {\bf R}\mbox{ with }|f(n)|\le c|g(n)|\,\forall n\ge n_0\}
\end{equation}
This definition is quite dense, but we can break it down: it says that
$O(g(n))$ is a set of functions, the curly brackets mean
\lq{}set\rq{}. $f(n)$ is in the set if it has a particular
property: the \lq$|$\rq{} can be read as \lq{}such that\rq{} or
\lq{}with the property that\rq{} and so this is the set of $f(n)$'s
where $f(n)$ has the property on the right of the $|$. Now,
\lq{}$\exists$\rq{} means \lq{}there exists\rq{} and
\lq{}$\forall$\rq{} means \lq{}for all\rq{}, so the defining property
says it is possible to find a positive natural number $n_0$ and a
positive real number $c$ for that if you choose a value of $n$ at
least as big as $n_0$ then $f(n)$ is no bigger than $cg(n)$, ${\bf N}$
and ${\bf R}$ stand for the natural and real numbers. Notice the
absolute value signs, this is about $|f(n)|$ and $|g(n)|$, in fact,
here we are interested in run times, so we will deal with functions
that are non-negative, or are non-negative provided $n$ is larger than
some threshold, for example, $\log_2{n}$ will be important,
$\log_2{n}$ is positive provided $n>1$.

In short, $f(n)$ can do all sorts of crazy stuff for small values of
$n$ but, if you take $n$ large enough, its behavior is bounded by the
behavior of $g(n)$. Now it doesn't say it is bounded by $g(n)$, it is
a statement about the behavior, that's the role of the $c$. If you
know the formal definition of limits you can see that the definition
of $O(g(n))$ has this wrapped up in it, if says
\begin{equation}
f(n)\in O(g(n))\iff \lim_{n\rightarrow \infty}\frac{f(n)}{g(n)}<\infty
\end{equation}

Here are some examples, say 
\begin{equation}
T(n)=5n^2+n+6
\end{equation}
then 
\begin{equation}
T(n)\in O(n^2)
\end{equation}
by, for example, taking $c=5+1+6=12$ then 
\begin{equation}
12n^2\ge 5n^2+n+6
\end{equation}
provided $n\ge 1$ so $n_0=1$ here. This could also be succinctly demonstrated using the limit
\begin{equation}
\lim_{n\rightarrow \infty} \frac{5n^2+n+6}{n^2}=\lim_{n\rightarrow \infty}5+\lim_{n\rightarrow \infty}\frac{1}{n}+\lim_{n\rightarrow \infty}\frac{6}{n^2}=5<\infty
\end{equation}

However, 
\begin{equation}
T(n)\not\in O(n)
\end{equation}
Say we chosen some value $c$ then
\begin{equation}
5n^2+n+6>cn
\end{equation}
for large enough $n$, to check this divide both sides by $n$ so we need to show that $n$ can be chosen so that
\begin{equation}
5n+1+\frac{c}{n}>c
\end{equation}
Since 
\begin{equation}
5n+1+\frac{c}{n}>5n+1
\end{equation}
then, if $n>c/5$
\begin{equation}
5n+1+\frac{c}{n}>5\frac{c}{5}+1=c+1>c
\end{equation}
so, no matter what value of $c$ is chosen, making $n>5/n$ implies
\begin{equation}
5n^2+n+6>cn
\end{equation}
so $5n^2+n+6\not\in O(n)$. Again, the limit does the same job
\begin{equation}
\lim_{n\rightarrow \infty} \frac{5n^2+n+6}{n}=\lim_{n\rightarrow \infty}5n+\lim_{n\rightarrow \infty}1+\lim_{n\rightarrow \infty}\frac{6}{n}=\infty
\end{equation}


In practice, if
\begin{equation}
T(n)=a_rn^r+a_{r-1}n^{r-1}+\ldots+a_1n+a_0
\end{equation}
then $T(n)\in O(n^r)$. 

The logarithm has the funny property that it goes to infinity, but it
does so slower than $n$:
\begin{equation}
\lim_{n\rightarrow \infty}\frac{\log_2{n}}{n}=0
\end{equation}
Here, in line with standard practice in computer science, we are using
the log to the base two, in fact changing bases only causes a change
of an overall constant, see Table~\ref{math_logs} for a reminder of
the properties of the log. The limit of $\log_2{n}/n$ can be calculated
using l'H\^{o}pital's rule, here we'll just look at a plot,
Fig.~\ref{fig_log}. Now, just as $\log_2{n}$ grows very slowly, $2^n$ grows very fast, 
\begin{equation}
\lim_{n\rightarrow \infty}\frac{n^r}{2^n}=0
\end{equation}
for any finite value of $r$, worse still is $n!$, pronounced $n$-factorial
\begin{equation}
n!=n(n-1)(n-2) . . . 1
\end{equation}
If you algorithm is in $O(n!)$ you will probably need a different
algorithm. A table of different values is given as
Table~\ref{table_n_values}, mostly to emphasis how quickly $n!$ gets
big.

Now, in mathematics we call something \lq{}an abuse of notation\rq{}
if it is common to write something that doesn't quite make sense but
acts as a shorthand for something that does. Now $O(g(n))$ is a set of
functions whose large $n$ behavior is bounded by $g(n)$ so in
algorithms, being in $O(g(n))$ is a property of $T(n)$, the formula
for the running time of the algorithm. However, it is a standard abuse
of notation to say an algorithm is $O(g(n))$ for some $g(n)$ is
$T(n)\in O(g(n))$ for all cases. This is another way of saying that
the worst behavior of the algorithm isn't any worse than the behavior
of $g(n)$ for large $n$ so all possible $T(n)$ are elements of
$O(g(n))$.

\begin{table}
The logarithm is the opposite of the exponent: if
\begin{equation}
a^b=c
\end{equation}
then 
\begin{equation}
\log_a{c}=b
\end{equation}
or, written in one line
\begin{equation}
a^{\log_a{c}}=c
\end{equation}
All the laws of logs can be worked out from the laws of
exponents. Hence, since $a^0=1$ we have $\log_a{1}=0$. In a similar
way, other rules of logs can be deduced like
\begin{eqnarray}
\log_a{c_1c_2}&=&\log_a{c_1}+\log_a{c_2}\cr
\log_a{\frac{c_1}{c_2}}&=&\log_a{c_1}-\log_a{c_2}\cr
\log_a c^d&=&d\log_a c
\end{eqnarray}
and so on.

As for the change of base, let $b=\log_{a_1}{c}$ so $a_1^{b}=c$. Now take the log to the base $a_2$ of both sides
\begin{equation}
b\log_{a_2}a_1=\log_{a_2}{c}
\end{equation}
and then solve for $b$
\begin{equation}
b=\frac{\log_{a_2}c}{\log_{a_2}a_1}
\end{equation}
and, substituting back the formula for $b$
\begin{equation}
\log_{a_1}{c}=\frac{\log_{a_2}c}{\log_{a_2}a_1}
\end{equation}
Thus we see, that changing bases is just a matter of a
multiplicative factor. For example, to change from base $e$ to base two
\begin{equation}
\log_{2}{x}=\frac{\log_e{x}}{\log_e{2}}\approx\frac{\log_e{x}}{0.6931}
\end{equation}
Common bases are $\log_2{x}$ used in computer science, $\log_e{x}$
sometimes written $\ln{x}$ used in mathematics and $\log_{10}{x}$ used
in chemistry. The base two is used because of its link to bits and
also, as we will see, because of its relationship with algorithms that
divide data into two piles. The natural log $\ln{x}$ is used where
differential equations are common since
\begin{equation}
\frac{d}{dx}\ln{x}=\frac{1}{x}
\end{equation}
\caption{A reminder about logarithms. This is a quick summary of some of the laws of logs.\label{math_logs}}
\end{table}

\begin{figure}
\include{I.2.log}
\caption{This shows $\log_2{x}$ and $x-1$ plots for $x\in[1,4]$, the one has been taken from $x$ to make them easier to compare, the key point is that the $x$ grows faster.\label{fig_log}}
\end{figure}


\begin{figure}
\include{I.2.exp}
\caption{This shows $2^x$ and $x^2-1$ plots for $x\in[0,6]$, clearly $2^x$ quickly overtakes $x^2+1$, this will happen for any power of $x$. \label{fig_log}}
\end{figure}

\begin{table}
\begin{tabular}{l|cccccc}
        $n$    &1   &2&4   &16  &128&1024\\
$\log{n}$      &0   &1&2   &4   &7  &10\\
$n\log{n}$     &0   &2&8   &64  &896&10240\\
$n^2$     &1   &4&16&256&16384&1048576\\
$2^n$     &2   &4&16&65536&$3.4\times 10^{38}$&$1.8\times 10^{307}$\\
$n!$      &1   &2&24&$2.1\times 10^{13}$&$3.85\times 10^{305}$&$5.4\times10^{2369}$
\end{tabular}
\vskip 1cm The website\\ {\tt
  http://markknowsnothing.com/cgi-bin/calculator.php}\\ was used for
the $2^n$ calculations and\\ {\tt
  http://www.calculatorsoup.com/calculators/discretemathematics/factorials.php}\\ for
the $n!$ calculations; these give answers even when the answer is very
large.

\caption{Different values of $n$ for some functions.  \label{table_n_values}
}
\end{table}

\subsubsection*{I.2.1 Examples: linear and binary search}

Lets do a quick example; searching for the index of an element in a
sorted list. A completely terrible way to do this is linear search,
this is terrible because it doesn't make use of the fact that the list
is sorted. A code listing is given in Table~\ref{c_linear_search}. We
can see straight away that the code between lines 6 and 10 is run $n$
times in the worst case, everything else is run once and so this
algorithm is $O(n)$.

\begin{table}
\begin{lstlisting}[numbers=left]
int search(int a[],int n, int val)
{

  int i;

  for(i=0;i<n;i++)
    {
      if(a[i]==val)
	return i;
    }

  return -1;
}
\end{lstlisting}
\caption{Linear search. This function searches the entries in the array a and returns the index when it finds val, if it doesn't find val it returns -1. The program {\tt linear\_search.c} implements this.\label{c_linear_search}.}
\end{table}

A much better way to search a sorted array is binary search. This is
an example of a \lq{}divide and conquer\rq{} algorithm, many of the
fastest algorithms use divide and conquer. It will be clear to you
that this algorithm would be better written using recursion, this is
typical of divide and conquer, but we haven't looked at analysing
recursion yet. The idea is to divide the array in half and check which
half, the half with bigger numbers or the half with smaller numbers,
the value we are searching for belongs to and to keep doing this,
dividing the remaining part of the array into two parts again and again
until the remaining part of the array that is being search has only
one element. A code listing is given in Table~\ref{c_binary_search}.

\begin{table}
\begin{lstlisting}[numbers=left] 
int search(int a[],int n, int val)
{
  int mid, low=0, high=n-1;

  while(low<=high)
    {
      mid=(low+high)/2;
      if(a[mid]==val)
	return mid;
      else if(val>a[mid])
	low=mid+1;
      else
	high=mid -1;
    }

  return -1;
}
\end{lstlisting}
\caption{Binary search. This function starts in the middle of the array
  and checks if the value there is bigger or smaller than val, if it
  is bigger then it does the same in the top half of the array, if it
  is smaller, in the bottom half and then repeats until there are no
  elements left. The program {\tt binary\_search.c} implements
  this.\label{c_binary_search}.}
\end{table}

This search is extremely fast. There is a chance that a[mid]==val,
after a small number of iterations, indeed, if the middle value of the
array is the search value it will halt after only one
iteration. However, as usual, we assume the worst case, in which case
the algorithm runs to end, dividing the number of elements in half
each time. Ignoring the integer rounding effects, it goes like
Table~\ref{tab_binary_search}. Starting with $n$ states each
subsequent iteration halves the number of states until the last one
when there is one state left. Thus
\begin{equation}
1=\frac{n}{2^{T(n)-1}}
\end{equation}
and taking the log of both sides
\begin{equation}
0=\log_2{n}-(T(n)-1)
\end{equation}
using $\log_2{1}=1$, $\log_2{a^b}=b\log_2{a}$ and $\log_2{2}=1$. Hence
\begin{equation}
T(n)=\log_2{n}+1
\end{equation}
and this algorithm is $O(\log_2{n})$.

\begin{table}
\begin{center}
\begin{tabular}{llllllll}
1&2&3&4&\ldots&k&\ldots&T(n)\\
\hline
$n$&$\frac{n}{2}$&$\frac{n}{4}$&$\frac{n}{8}$&$\ldots$&$\frac{n}{2^{k-1}}$&$\ldots$&1
\end{tabular}
\end{center}
\caption{The number of elements left for binary search.\label{tab_binary_search}}
\end{table}

So, to reiterate; the usual way to examine the behavior of an
algorithm is to look at the worst case run time. This is because the
best case run time is often exceptional, like the one for binary
search if the first guess happens to be correct. The average run time
is often hard to calculate, both because it is often difficult to do
the mathematics and because it would often mean having some
description of how the initial data is distributed. Typically the
worst run time is also \lq{}of the same order\rq{} as the average run
time. We will see an exception to this later on in the case of quick
sort in which the worst case behavior is unusual. The big-Oh notation
is used for describing an algorithm, if the algorithm is said to be
$O(g(n))$ we mean $T(n)\in O(g(n))$ no matter what the initial
condition. Since $O(g(n))$ involves an upper bound $T(n)<cg(n)$ this
makes sense.


\subsubsection*{Other big Letter notations, small oh notation.}

There is another set, $\Omega(g(n))$ with a definition similar to
$O(g(n))$ that is used for describing the best case behavior. This
requires a lower bound rather than an upper bound, so the obvious
definition is
\begin{equation}
\Omega(g(n))=\{f(n)| \exists n_0>0\in {\bf N}\mbox{ and }c>0\in {\bf R}\mbox{ with }|f(n)|\ge c|g(n)|\,\forall n\ge n_0\}
\end{equation}
in other words, the same thing, but with the $\le$ symbol replaced by
a $\ge$. In fact, there is some ambiguity about this definition,
number theorist use a slightly different one. Either way, it isn't
used very often in computer science because algorithms are very
frequently $\Omega(1)$; in the best case scenario the problem is in
some sense already solve, the array already sorted for example, and
the algorithm finishs in one step. 

There is also a set of function that are both bounded above and below
by the same $g(n)$
\begin{equation}
\Theta(g(n))=\Omega(g(n))\cap O(g(n))
\end{equation}
This works because it is possible for
\begin{equation}
c_1 g(n)\le f(n)\le c_2g(n)
\end{equation}
for different $c_1$ and $c_2$. It would be very unusual for this to
apply to an algorithm, it would mean that $T(n)$ has the same behavior
for large $n$ no matter whether it is the best case or the worst case
scenario. There is a na\"ive largest element function in
Table~\ref{c_largest_linear_search} which is $\Theta(n)$. It searches
for the largest value in an unsorted array by looking at each element
in turn. In fact, for a completely unsorted array this is the best
algorithm, but, in practice, if finding the largest element in a set
is an important and frequent procedure, a special data structure,
called a heap, is used to keep track of which element is largest.

\begin{table}
\begin{lstlisting}[numbers=left]
int search(int a[],int n)
{

  int i;
  int best_val=a[0];

  for(i=1;i<n;i++)
    {
      if(a[i]>best_val)
	best_val= a[i];
    }

  return best_val;
}
\end{lstlisting}
\caption{Search for the largest element in an unsorted list. This
  function searches all the elements to see which is the largest, the
  inner loop always runs $n-1$ times since it doesn't know until it
  has looked at every element which is going to be the largest. This
  program is implemented as {\tt
    find\_largest.c}.\label{c_largest_linear_search}.}
\end{table}

Finally, little oh notation is a stricter version of big Oh notation
that is used in some more mathematical context, basically $f(n)\in
o(g(n))$ is $f(n)$ is less than $cg(n)$ for any choice of $c$, if $n$
is large enough:
\begin{equation}
\o(g(n))=\{f(n)| \exists n_0>0\in {\bf N}\mbox{ so that }|f(n)|\ge
c|g(n)|\,\forall n\ge n_0 \mbox{ and }\forall\,c>0\in {\bf R}\}
\end{equation}

\end{document}

\caption{This shows $2^x$ and $x^2-1$ plots for $x\in[0,6]$, clearly $2^x$ quickly overtakes $x^2+1$, this will happen for any power of $x$. \label{fig_log}}
\end{figure}

\begin{table}
\begin{tabular}{l|cccccc}
        $n$    &1   &2&4   &16  &128&1024\\
$\log{n}$      &0   &1&2   &4   &7  &10\\
$n\log{n}$     &0   &2&8   &64  &896&10240\\
$n^2$     &1   &4&16&256&16384&1048576\\
$2^n$     &2   &4&16&65536&$3.4\times 10^{38}$&$1.8\times 10^{307}$\\
$n!$      &1   &2&24&$2.1\times 10^{13}$&$3.85\times 10^{305}$&$5.4\times10^{2369}$
\end{tabular}
\vskip 1cm The website\\ {\tt
  http://markknowsnothing.com/cgi-bin/calculator.php}\\ was used for
the $2^n$ calculations and\\ {\tt
  http://www.calculatorsoup.com/calculators/discretemathematics/factorials.php}\\ for
the $n!$ calculations; these give answers even when the answer is very
large.

\caption{Different values of $n$ for some functions.  \label{table_n_values}
}
\end{table}

\subsubsection*{I.2.1 Examples: linear and binary search}

Lets do a quick example; searching for the index of an element in a
sorted list. A completely terrible way to do this is linear search,
this is terrible because it doesn't make use of the fact that the list
is sorted. A code listing is given in Table~\ref{c_linear_search}. We
can see straight away that the code between lines 6 and 10 is run $n$
times in the worst case, everything else is run once and so this
algorithm is $O(n)$.

\begin{table}
\begin{lstlisting}[numbers=left]
int search(int a[],int n, int val)
{

  int i;

  for(i=0;i<n;i++)
    {
      if(a[i]==val)
	return i;
    }

  return -1;
}
\end{lstlisting}
\caption{Linear search. This function searches the entries in the array a and returns the index when it finds val, if it doesn't find val it returns -1. The program {\tt linear\_search.c} implements this.\label{c_linear_search}.}
\end{table}

A much better way to search a sorted array is binary search. This is
an example of a \lq{}divide and conquer\rq{} algorithm, many of the
fastest algorithms use divide and conquer. It will be clear to you
that this algorithm would be better written using recursion, this is
typical of divide and conquer, but we haven't looked at analysing
recursion yet. The idea is to divide the array in half and check which
half, the half with bigger numbers or the half with smaller numbers,
the value we are searching for belongs to and to keep doing this,
dividing the remaining part of the array into two parts again and again
until the remaining part of the array that is being search has only
one element. A code listing is given in Table~\ref{c_binary_search}.

\begin{table}
\begin{lstlisting}[numbers=left] 
int search(int a[],int n, int val)
{
  int mid, low=0, high=n-1;

  while(low<=high)
    {
      mid=(low+high)/2;
      if(a[mid]==val)
	return mid;
      else if(val>a[mid])
	low=mid+1;
      else
	high=mid -1;
    }

  return -1;
}
\end{lstlisting}
\caption{Binary search. This function starts in the middle of the array
  and checks if the value there is bigger or smaller than val, if it
  is bigger then it does the same in the top half of the array, if it
  is smaller, in the bottom half and then repeats until there are no
  elements left. The program {\tt binary\_search.c} implements
  this.\label{c_binary_search}.}
\end{table}

This search is extremely fast. There is a chance that a[mid]==val,
after a small number of iterations, indeed, if the middle value of the
array is the search value it will halt after only one
iteration. However, as usual, we assume the worst case, in which case
the algorithm runs to end, dividing the number of elements in half
each time. Ignoring the integer rounding effects, it goes like
Table~\ref{tab_binary_search}. Starting with $n$ states each
subsequent iteration halves the number of states until the last one
when there is one state left. Thus
\begin{equation}
1=\frac{n}{2^{T(n)-1}}
\end{equation}
and taking the log of both sides
\begin{equation}
0=\log_2{n}-(T(n)-1)
\end{equation}
using $\log_2{1}=1$, $\log_2{a^b}=b\log_2{a}$ and $\log_2{2}=1$. Hence
\begin{equation}
T(n)=\log_2{n}+1
\end{equation}
and this algorithm is $O(\log_2{n})$.

\begin{table}
\begin{center}
\begin{tabular}{llllllll}
1&2&3&4&\ldots&k&\ldots&T(n)\\
\hline
$n$&$\frac{n}{2}$&$\frac{n}{4}$&$\frac{n}{8}$&$\ldots$&$\frac{n}{2^{k-1}}$&$\ldots$&1
\end{tabular}
\end{center}
\caption{The number of elements left for binary search.\label{tab_binary_search}}
\end{table}

So, to reiterate; the usual way to examine the behavior of an
algorithm is to look at the worst case run time. This is because the
best case run time is often exceptional, like the one for binary
search if the first guess happens to be correct. The average run time
is often hard to calculate, both because it is often difficult to do
the mathematics and because it would often mean having some
description of how the initial data is distributed. Typically the
worst run time is also \lq{}of the same order\rq{} as the average run
time. We will see an exception to this later on in the case of quick
sort in which the worst case behavior is unusual. The big-Oh notation
is used for describing an algorithm, if the algorithm is said to be
$O(g(n))$ we mean $T(n)\in O(g(n))$ no matter what the initial
condition. Since $O(g(n))$ involves an upper bound $T(n)<cg(n)$ this
makes sense.


\subsubsection*{Other big Letter notations, small oh notation.}

There is another set, $\Omega(g(n))$ with a definition similar to
$O(g(n))$ that is used for describing the best case behavior. This
requires a lower bound rather than an upper bound, so the obvious
definition is
\begin{equation}
\Omega(g(n))=\{f(n)| \exists n_0>0\in {\bf N}\mbox{ and }c>0\in {\bf R}\mbox{ with }|f(n)|\ge c|g(n)|\,\forall n\ge n_0\}
\end{equation}
in other words, the same thing, but with the $\le$ symbol replaced by
a $\ge$. In fact, there is some ambiguity about this definition,
number theorist use a slightly different one. Either way, it isn't
used very often in computer science because algorithms are very
frequently $\Omega(1)$; in the best case scenario the problem is in
some sense already solve, the array already sorted for example, and
the algorithm finishs in one step. 

There is also a set of function that are both bounded above and below
by the same $g(n)$
\begin{equation}
\Theta(g(n))=\Omega(g(n))\cap O(g(n))
\end{equation}
This works because it is possible for
\begin{equation}
c_1 g(n)\le f(n)\le c_2g(n)
\end{equation}
for different $c_1$ and $c_2$. It would be very unusual for this to
apply to an algorithm, it would mean that $T(n)$ has the same behavior
for large $n$ no matter whether it is the best case or the worst case
scenario. There is a na\"ive largest element function in
Table~\ref{c_largest_linear_search} which is $\Theta(n)$. It searches
for the largest value in an unsorted array by looking at each element
in turn. In fact, for a completely unsorted array this is the best
algorithm, but, in practice, if finding the largest element in a set
is an important and frequent procedure, a special data structure,
called a heap, is used to keep track of which element is largest.

\begin{table}
\begin{lstlisting}[numbers=left]
int search(int a[],int n)
{

  int i;
  int best_val=a[0];

  for(i=1;i<n;i++)
    {
      if(a[i]>best_val)
	best_val= a[i];
    }

  return best_val;
}
\end{lstlisting}
\caption{Search for the largest element in an unsorted list. This
  function searches all the elements to see which is the largest, the
  inner loop always runs $n-1$ times since it doesn't know until it
  has looked at every element which is going to be the largest. This
  program is implemented as {\tt
    find\_largest.c}.\label{c_largest_linear_search}.}
\end{table}

Finally, little oh notation is a stricter version of big Oh notation
that is used in some more mathematical context, basically $f(n)\in
o(g(n))$ is $f(n)$ is less than $cg(n)$ for any choice of $c$, if $n$
is large enough:
\begin{equation}
\o(g(n))=\{f(n)| \exists n_0>0\in {\bf N}\mbox{ so that }|f(n)|\ge
c|g(n)|\,\forall n\ge n_0 \mbox{ and }\forall\,c>0\in {\bf R}\}
\end{equation}

\end{document}

\caption{This shows $\log_2{x}$ and $x-1$ plots for $x\in[1,4]$, the one has been taken from $x$ to make them easier to compare, the key point is that the $x$ grows faster.\label{fig_log}}
\end{figure}


\begin{figure}
\documentclass[11pt,a4paper]{scrartcl}
\typearea{12}
\usepackage{graphicx}
\usepackage{pstricks}
\usepackage{listings}
\lstset{language=C}
\pagestyle{headings}
\markright{COMS11600 - Principles of programming I.2}
\begin{document}

\subsection*{I.2 Big Oh notation}

The \lq{}Big Oh\rq{} notation has already been used to describe the
behavior of the running time of insert sort, we said
\begin{equation}
T(n)\in O(n^2)
\end{equation}
Here we want to formalize this notation. Basically $O(n^2)$ is a set
of function, it is all the function which, for large values of $n$ go
to infinity like $n^2$ at the fastest. By saying $T(n)\in O(n^2)$ we
are saying that $T(n)$ is one of these functions, its large $n$
behavior is, as worst, like $n^2$. 

Specifically, the definition for $g(n)$ for all $n$
\begin{equation}
O(g(n))=\{f(n)| \exists n_0>0\in {\bf N}\mbox{ and }c>0\in {\bf R}\mbox{ with }|f(n)|\le c|g(n)|\,\forall n\ge n_0\}
\end{equation}
This definition is quite dense, but we can break it down: it says that
$O(g(n))$ is a set of functions, the curly brackets mean
\lq{}set\rq{}. $f(n)$ is in the set if it has a particular
property: the \lq$|$\rq{} can be read as \lq{}such that\rq{} or
\lq{}with the property that\rq{} and so this is the set of $f(n)$'s
where $f(n)$ has the property on the right of the $|$. Now,
\lq{}$\exists$\rq{} means \lq{}there exists\rq{} and
\lq{}$\forall$\rq{} means \lq{}for all\rq{}, so the defining property
says it is possible to find a positive natural number $n_0$ and a
positive real number $c$ for that if you choose a value of $n$ at
least as big as $n_0$ then $f(n)$ is no bigger than $cg(n)$, ${\bf N}$
and ${\bf R}$ stand for the natural and real numbers. Notice the
absolute value signs, this is about $|f(n)|$ and $|g(n)|$, in fact,
here we are interested in run times, so we will deal with functions
that are non-negative, or are non-negative provided $n$ is larger than
some threshold, for example, $\log_2{n}$ will be important,
$\log_2{n}$ is positive provided $n>1$.

In short, $f(n)$ can do all sorts of crazy stuff for small values of
$n$ but, if you take $n$ large enough, its behavior is bounded by the
behavior of $g(n)$. Now it doesn't say it is bounded by $g(n)$, it is
a statement about the behavior, that's the role of the $c$. If you
know the formal definition of limits you can see that the definition
of $O(g(n))$ has this wrapped up in it, if says
\begin{equation}
f(n)\in O(g(n))\iff \lim_{n\rightarrow \infty}\frac{f(n)}{g(n)}<\infty
\end{equation}

Here are some examples, say 
\begin{equation}
T(n)=5n^2+n+6
\end{equation}
then 
\begin{equation}
T(n)\in O(n^2)
\end{equation}
by, for example, taking $c=5+1+6=12$ then 
\begin{equation}
12n^2\ge 5n^2+n+6
\end{equation}
provided $n\ge 1$ so $n_0=1$ here. This could also be succinctly demonstrated using the limit
\begin{equation}
\lim_{n\rightarrow \infty} \frac{5n^2+n+6}{n^2}=\lim_{n\rightarrow \infty}5+\lim_{n\rightarrow \infty}\frac{1}{n}+\lim_{n\rightarrow \infty}\frac{6}{n^2}=5<\infty
\end{equation}

However, 
\begin{equation}
T(n)\not\in O(n)
\end{equation}
Say we chosen some value $c$ then
\begin{equation}
5n^2+n+6>cn
\end{equation}
for large enough $n$, to check this divide both sides by $n$ so we need to show that $n$ can be chosen so that
\begin{equation}
5n+1+\frac{c}{n}>c
\end{equation}
Since 
\begin{equation}
5n+1+\frac{c}{n}>5n+1
\end{equation}
then, if $n>c/5$
\begin{equation}
5n+1+\frac{c}{n}>5\frac{c}{5}+1=c+1>c
\end{equation}
so, no matter what value of $c$ is chosen, making $n>5/n$ implies
\begin{equation}
5n^2+n+6>cn
\end{equation}
so $5n^2+n+6\not\in O(n)$. Again, the limit does the same job
\begin{equation}
\lim_{n\rightarrow \infty} \frac{5n^2+n+6}{n}=\lim_{n\rightarrow \infty}5n+\lim_{n\rightarrow \infty}1+\lim_{n\rightarrow \infty}\frac{6}{n}=\infty
\end{equation}


In practice, if
\begin{equation}
T(n)=a_rn^r+a_{r-1}n^{r-1}+\ldots+a_1n+a_0
\end{equation}
then $T(n)\in O(n^r)$. 

The logarithm has the funny property that it goes to infinity, but it
does so slower than $n$:
\begin{equation}
\lim_{n\rightarrow \infty}\frac{\log_2{n}}{n}=0
\end{equation}
Here, in line with standard practice in computer science, we are using
the log to the base two, in fact changing bases only causes a change
of an overall constant, see Table~\ref{math_logs} for a reminder of
the properties of the log. The limit of $\log_2{n}/n$ can be calculated
using l'H\^{o}pital's rule, here we'll just look at a plot,
Fig.~\ref{fig_log}. Now, just as $\log_2{n}$ grows very slowly, $2^n$ grows very fast, 
\begin{equation}
\lim_{n\rightarrow \infty}\frac{n^r}{2^n}=0
\end{equation}
for any finite value of $r$, worse still is $n!$, pronounced $n$-factorial
\begin{equation}
n!=n(n-1)(n-2) . . . 1
\end{equation}
If you algorithm is in $O(n!)$ you will probably need a different
algorithm. A table of different values is given as
Table~\ref{table_n_values}, mostly to emphasis how quickly $n!$ gets
big.

Now, in mathematics we call something \lq{}an abuse of notation\rq{}
if it is common to write something that doesn't quite make sense but
acts as a shorthand for something that does. Now $O(g(n))$ is a set of
functions whose large $n$ behavior is bounded by $g(n)$ so in
algorithms, being in $O(g(n))$ is a property of $T(n)$, the formula
for the running time of the algorithm. However, it is a standard abuse
of notation to say an algorithm is $O(g(n))$ for some $g(n)$ is
$T(n)\in O(g(n))$ for all cases. This is another way of saying that
the worst behavior of the algorithm isn't any worse than the behavior
of $g(n)$ for large $n$ so all possible $T(n)$ are elements of
$O(g(n))$.

\begin{table}
The logarithm is the opposite of the exponent: if
\begin{equation}
a^b=c
\end{equation}
then 
\begin{equation}
\log_a{c}=b
\end{equation}
or, written in one line
\begin{equation}
a^{\log_a{c}}=c
\end{equation}
All the laws of logs can be worked out from the laws of
exponents. Hence, since $a^0=1$ we have $\log_a{1}=0$. In a similar
way, other rules of logs can be deduced like
\begin{eqnarray}
\log_a{c_1c_2}&=&\log_a{c_1}+\log_a{c_2}\cr
\log_a{\frac{c_1}{c_2}}&=&\log_a{c_1}-\log_a{c_2}\cr
\log_a c^d&=&d\log_a c
\end{eqnarray}
and so on.

As for the change of base, let $b=\log_{a_1}{c}$ so $a_1^{b}=c$. Now take the log to the base $a_2$ of both sides
\begin{equation}
b\log_{a_2}a_1=\log_{a_2}{c}
\end{equation}
and then solve for $b$
\begin{equation}
b=\frac{\log_{a_2}c}{\log_{a_2}a_1}
\end{equation}
and, substituting back the formula for $b$
\begin{equation}
\log_{a_1}{c}=\frac{\log_{a_2}c}{\log_{a_2}a_1}
\end{equation}
Thus we see, that changing bases is just a matter of a
multiplicative factor. For example, to change from base $e$ to base two
\begin{equation}
\log_{2}{x}=\frac{\log_e{x}}{\log_e{2}}\approx\frac{\log_e{x}}{0.6931}
\end{equation}
Common bases are $\log_2{x}$ used in computer science, $\log_e{x}$
sometimes written $\ln{x}$ used in mathematics and $\log_{10}{x}$ used
in chemistry. The base two is used because of its link to bits and
also, as we will see, because of its relationship with algorithms that
divide data into two piles. The natural log $\ln{x}$ is used where
differential equations are common since
\begin{equation}
\frac{d}{dx}\ln{x}=\frac{1}{x}
\end{equation}
\caption{A reminder about logarithms. This is a quick summary of some of the laws of logs.\label{math_logs}}
\end{table}

\begin{figure}
\documentclass[11pt,a4paper]{scrartcl}
\typearea{12}
\usepackage{graphicx}
\usepackage{pstricks}
\usepackage{listings}
\lstset{language=C}
\pagestyle{headings}
\markright{COMS11600 - Principles of programming I.2}
\begin{document}

\subsection*{I.2 Big Oh notation}

The \lq{}Big Oh\rq{} notation has already been used to describe the
behavior of the running time of insert sort, we said
\begin{equation}
T(n)\in O(n^2)
\end{equation}
Here we want to formalize this notation. Basically $O(n^2)$ is a set
of function, it is all the function which, for large values of $n$ go
to infinity like $n^2$ at the fastest. By saying $T(n)\in O(n^2)$ we
are saying that $T(n)$ is one of these functions, its large $n$
behavior is, as worst, like $n^2$. 

Specifically, the definition for $g(n)$ for all $n$
\begin{equation}
O(g(n))=\{f(n)| \exists n_0>0\in {\bf N}\mbox{ and }c>0\in {\bf R}\mbox{ with }|f(n)|\le c|g(n)|\,\forall n\ge n_0\}
\end{equation}
This definition is quite dense, but we can break it down: it says that
$O(g(n))$ is a set of functions, the curly brackets mean
\lq{}set\rq{}. $f(n)$ is in the set if it has a particular
property: the \lq$|$\rq{} can be read as \lq{}such that\rq{} or
\lq{}with the property that\rq{} and so this is the set of $f(n)$'s
where $f(n)$ has the property on the right of the $|$. Now,
\lq{}$\exists$\rq{} means \lq{}there exists\rq{} and
\lq{}$\forall$\rq{} means \lq{}for all\rq{}, so the defining property
says it is possible to find a positive natural number $n_0$ and a
positive real number $c$ for that if you choose a value of $n$ at
least as big as $n_0$ then $f(n)$ is no bigger than $cg(n)$, ${\bf N}$
and ${\bf R}$ stand for the natural and real numbers. Notice the
absolute value signs, this is about $|f(n)|$ and $|g(n)|$, in fact,
here we are interested in run times, so we will deal with functions
that are non-negative, or are non-negative provided $n$ is larger than
some threshold, for example, $\log_2{n}$ will be important,
$\log_2{n}$ is positive provided $n>1$.

In short, $f(n)$ can do all sorts of crazy stuff for small values of
$n$ but, if you take $n$ large enough, its behavior is bounded by the
behavior of $g(n)$. Now it doesn't say it is bounded by $g(n)$, it is
a statement about the behavior, that's the role of the $c$. If you
know the formal definition of limits you can see that the definition
of $O(g(n))$ has this wrapped up in it, if says
\begin{equation}
f(n)\in O(g(n))\iff \lim_{n\rightarrow \infty}\frac{f(n)}{g(n)}<\infty
\end{equation}

Here are some examples, say 
\begin{equation}
T(n)=5n^2+n+6
\end{equation}
then 
\begin{equation}
T(n)\in O(n^2)
\end{equation}
by, for example, taking $c=5+1+6=12$ then 
\begin{equation}
12n^2\ge 5n^2+n+6
\end{equation}
provided $n\ge 1$ so $n_0=1$ here. This could also be succinctly demonstrated using the limit
\begin{equation}
\lim_{n\rightarrow \infty} \frac{5n^2+n+6}{n^2}=\lim_{n\rightarrow \infty}5+\lim_{n\rightarrow \infty}\frac{1}{n}+\lim_{n\rightarrow \infty}\frac{6}{n^2}=5<\infty
\end{equation}

However, 
\begin{equation}
T(n)\not\in O(n)
\end{equation}
Say we chosen some value $c$ then
\begin{equation}
5n^2+n+6>cn
\end{equation}
for large enough $n$, to check this divide both sides by $n$ so we need to show that $n$ can be chosen so that
\begin{equation}
5n+1+\frac{c}{n}>c
\end{equation}
Since 
\begin{equation}
5n+1+\frac{c}{n}>5n+1
\end{equation}
then, if $n>c/5$
\begin{equation}
5n+1+\frac{c}{n}>5\frac{c}{5}+1=c+1>c
\end{equation}
so, no matter what value of $c$ is chosen, making $n>5/n$ implies
\begin{equation}
5n^2+n+6>cn
\end{equation}
so $5n^2+n+6\not\in O(n)$. Again, the limit does the same job
\begin{equation}
\lim_{n\rightarrow \infty} \frac{5n^2+n+6}{n}=\lim_{n\rightarrow \infty}5n+\lim_{n\rightarrow \infty}1+\lim_{n\rightarrow \infty}\frac{6}{n}=\infty
\end{equation}


In practice, if
\begin{equation}
T(n)=a_rn^r+a_{r-1}n^{r-1}+\ldots+a_1n+a_0
\end{equation}
then $T(n)\in O(n^r)$. 

The logarithm has the funny property that it goes to infinity, but it
does so slower than $n$:
\begin{equation}
\lim_{n\rightarrow \infty}\frac{\log_2{n}}{n}=0
\end{equation}
Here, in line with standard practice in computer science, we are using
the log to the base two, in fact changing bases only causes a change
of an overall constant, see Table~\ref{math_logs} for a reminder of
the properties of the log. The limit of $\log_2{n}/n$ can be calculated
using l'H\^{o}pital's rule, here we'll just look at a plot,
Fig.~\ref{fig_log}. Now, just as $\log_2{n}$ grows very slowly, $2^n$ grows very fast, 
\begin{equation}
\lim_{n\rightarrow \infty}\frac{n^r}{2^n}=0
\end{equation}
for any finite value of $r$, worse still is $n!$, pronounced $n$-factorial
\begin{equation}
n!=n(n-1)(n-2) . . . 1
\end{equation}
If you algorithm is in $O(n!)$ you will probably need a different
algorithm. A table of different values is given as
Table~\ref{table_n_values}, mostly to emphasis how quickly $n!$ gets
big.

Now, in mathematics we call something \lq{}an abuse of notation\rq{}
if it is common to write something that doesn't quite make sense but
acts as a shorthand for something that does. Now $O(g(n))$ is a set of
functions whose large $n$ behavior is bounded by $g(n)$ so in
algorithms, being in $O(g(n))$ is a property of $T(n)$, the formula
for the running time of the algorithm. However, it is a standard abuse
of notation to say an algorithm is $O(g(n))$ for some $g(n)$ is
$T(n)\in O(g(n))$ for all cases. This is another way of saying that
the worst behavior of the algorithm isn't any worse than the behavior
of $g(n)$ for large $n$ so all possible $T(n)$ are elements of
$O(g(n))$.

\begin{table}
The logarithm is the opposite of the exponent: if
\begin{equation}
a^b=c
\end{equation}
then 
\begin{equation}
\log_a{c}=b
\end{equation}
or, written in one line
\begin{equation}
a^{\log_a{c}}=c
\end{equation}
All the laws of logs can be worked out from the laws of
exponents. Hence, since $a^0=1$ we have $\log_a{1}=0$. In a similar
way, other rules of logs can be deduced like
\begin{eqnarray}
\log_a{c_1c_2}&=&\log_a{c_1}+\log_a{c_2}\cr
\log_a{\frac{c_1}{c_2}}&=&\log_a{c_1}-\log_a{c_2}\cr
\log_a c^d&=&d\log_a c
\end{eqnarray}
and so on.

As for the change of base, let $b=\log_{a_1}{c}$ so $a_1^{b}=c$. Now take the log to the base $a_2$ of both sides
\begin{equation}
b\log_{a_2}a_1=\log_{a_2}{c}
\end{equation}
and then solve for $b$
\begin{equation}
b=\frac{\log_{a_2}c}{\log_{a_2}a_1}
\end{equation}
and, substituting back the formula for $b$
\begin{equation}
\log_{a_1}{c}=\frac{\log_{a_2}c}{\log_{a_2}a_1}
\end{equation}
Thus we see, that changing bases is just a matter of a
multiplicative factor. For example, to change from base $e$ to base two
\begin{equation}
\log_{2}{x}=\frac{\log_e{x}}{\log_e{2}}\approx\frac{\log_e{x}}{0.6931}
\end{equation}
Common bases are $\log_2{x}$ used in computer science, $\log_e{x}$
sometimes written $\ln{x}$ used in mathematics and $\log_{10}{x}$ used
in chemistry. The base two is used because of its link to bits and
also, as we will see, because of its relationship with algorithms that
divide data into two piles. The natural log $\ln{x}$ is used where
differential equations are common since
\begin{equation}
\frac{d}{dx}\ln{x}=\frac{1}{x}
\end{equation}
\caption{A reminder about logarithms. This is a quick summary of some of the laws of logs.\label{math_logs}}
\end{table}

\begin{figure}
\include{I.2.log}
\caption{This shows $\log_2{x}$ and $x-1$ plots for $x\in[1,4]$, the one has been taken from $x$ to make them easier to compare, the key point is that the $x$ grows faster.\label{fig_log}}
\end{figure}


\begin{figure}
\include{I.2.exp}
\caption{This shows $2^x$ and $x^2-1$ plots for $x\in[0,6]$, clearly $2^x$ quickly overtakes $x^2+1$, this will happen for any power of $x$. \label{fig_log}}
\end{figure}

\begin{table}
\begin{tabular}{l|cccccc}
        $n$    &1   &2&4   &16  &128&1024\\
$\log{n}$      &0   &1&2   &4   &7  &10\\
$n\log{n}$     &0   &2&8   &64  &896&10240\\
$n^2$     &1   &4&16&256&16384&1048576\\
$2^n$     &2   &4&16&65536&$3.4\times 10^{38}$&$1.8\times 10^{307}$\\
$n!$      &1   &2&24&$2.1\times 10^{13}$&$3.85\times 10^{305}$&$5.4\times10^{2369}$
\end{tabular}
\vskip 1cm The website\\ {\tt
  http://markknowsnothing.com/cgi-bin/calculator.php}\\ was used for
the $2^n$ calculations and\\ {\tt
  http://www.calculatorsoup.com/calculators/discretemathematics/factorials.php}\\ for
the $n!$ calculations; these give answers even when the answer is very
large.

\caption{Different values of $n$ for some functions.  \label{table_n_values}
}
\end{table}

\subsubsection*{I.2.1 Examples: linear and binary search}

Lets do a quick example; searching for the index of an element in a
sorted list. A completely terrible way to do this is linear search,
this is terrible because it doesn't make use of the fact that the list
is sorted. A code listing is given in Table~\ref{c_linear_search}. We
can see straight away that the code between lines 6 and 10 is run $n$
times in the worst case, everything else is run once and so this
algorithm is $O(n)$.

\begin{table}
\begin{lstlisting}[numbers=left]
int search(int a[],int n, int val)
{

  int i;

  for(i=0;i<n;i++)
    {
      if(a[i]==val)
	return i;
    }

  return -1;
}
\end{lstlisting}
\caption{Linear search. This function searches the entries in the array a and returns the index when it finds val, if it doesn't find val it returns -1. The program {\tt linear\_search.c} implements this.\label{c_linear_search}.}
\end{table}

A much better way to search a sorted array is binary search. This is
an example of a \lq{}divide and conquer\rq{} algorithm, many of the
fastest algorithms use divide and conquer. It will be clear to you
that this algorithm would be better written using recursion, this is
typical of divide and conquer, but we haven't looked at analysing
recursion yet. The idea is to divide the array in half and check which
half, the half with bigger numbers or the half with smaller numbers,
the value we are searching for belongs to and to keep doing this,
dividing the remaining part of the array into two parts again and again
until the remaining part of the array that is being search has only
one element. A code listing is given in Table~\ref{c_binary_search}.

\begin{table}
\begin{lstlisting}[numbers=left] 
int search(int a[],int n, int val)
{
  int mid, low=0, high=n-1;

  while(low<=high)
    {
      mid=(low+high)/2;
      if(a[mid]==val)
	return mid;
      else if(val>a[mid])
	low=mid+1;
      else
	high=mid -1;
    }

  return -1;
}
\end{lstlisting}
\caption{Binary search. This function starts in the middle of the array
  and checks if the value there is bigger or smaller than val, if it
  is bigger then it does the same in the top half of the array, if it
  is smaller, in the bottom half and then repeats until there are no
  elements left. The program {\tt binary\_search.c} implements
  this.\label{c_binary_search}.}
\end{table}

This search is extremely fast. There is a chance that a[mid]==val,
after a small number of iterations, indeed, if the middle value of the
array is the search value it will halt after only one
iteration. However, as usual, we assume the worst case, in which case
the algorithm runs to end, dividing the number of elements in half
each time. Ignoring the integer rounding effects, it goes like
Table~\ref{tab_binary_search}. Starting with $n$ states each
subsequent iteration halves the number of states until the last one
when there is one state left. Thus
\begin{equation}
1=\frac{n}{2^{T(n)-1}}
\end{equation}
and taking the log of both sides
\begin{equation}
0=\log_2{n}-(T(n)-1)
\end{equation}
using $\log_2{1}=1$, $\log_2{a^b}=b\log_2{a}$ and $\log_2{2}=1$. Hence
\begin{equation}
T(n)=\log_2{n}+1
\end{equation}
and this algorithm is $O(\log_2{n})$.

\begin{table}
\begin{center}
\begin{tabular}{llllllll}
1&2&3&4&\ldots&k&\ldots&T(n)\\
\hline
$n$&$\frac{n}{2}$&$\frac{n}{4}$&$\frac{n}{8}$&$\ldots$&$\frac{n}{2^{k-1}}$&$\ldots$&1
\end{tabular}
\end{center}
\caption{The number of elements left for binary search.\label{tab_binary_search}}
\end{table}

So, to reiterate; the usual way to examine the behavior of an
algorithm is to look at the worst case run time. This is because the
best case run time is often exceptional, like the one for binary
search if the first guess happens to be correct. The average run time
is often hard to calculate, both because it is often difficult to do
the mathematics and because it would often mean having some
description of how the initial data is distributed. Typically the
worst run time is also \lq{}of the same order\rq{} as the average run
time. We will see an exception to this later on in the case of quick
sort in which the worst case behavior is unusual. The big-Oh notation
is used for describing an algorithm, if the algorithm is said to be
$O(g(n))$ we mean $T(n)\in O(g(n))$ no matter what the initial
condition. Since $O(g(n))$ involves an upper bound $T(n)<cg(n)$ this
makes sense.


\subsubsection*{Other big Letter notations, small oh notation.}

There is another set, $\Omega(g(n))$ with a definition similar to
$O(g(n))$ that is used for describing the best case behavior. This
requires a lower bound rather than an upper bound, so the obvious
definition is
\begin{equation}
\Omega(g(n))=\{f(n)| \exists n_0>0\in {\bf N}\mbox{ and }c>0\in {\bf R}\mbox{ with }|f(n)|\ge c|g(n)|\,\forall n\ge n_0\}
\end{equation}
in other words, the same thing, but with the $\le$ symbol replaced by
a $\ge$. In fact, there is some ambiguity about this definition,
number theorist use a slightly different one. Either way, it isn't
used very often in computer science because algorithms are very
frequently $\Omega(1)$; in the best case scenario the problem is in
some sense already solve, the array already sorted for example, and
the algorithm finishs in one step. 

There is also a set of function that are both bounded above and below
by the same $g(n)$
\begin{equation}
\Theta(g(n))=\Omega(g(n))\cap O(g(n))
\end{equation}
This works because it is possible for
\begin{equation}
c_1 g(n)\le f(n)\le c_2g(n)
\end{equation}
for different $c_1$ and $c_2$. It would be very unusual for this to
apply to an algorithm, it would mean that $T(n)$ has the same behavior
for large $n$ no matter whether it is the best case or the worst case
scenario. There is a na\"ive largest element function in
Table~\ref{c_largest_linear_search} which is $\Theta(n)$. It searches
for the largest value in an unsorted array by looking at each element
in turn. In fact, for a completely unsorted array this is the best
algorithm, but, in practice, if finding the largest element in a set
is an important and frequent procedure, a special data structure,
called a heap, is used to keep track of which element is largest.

\begin{table}
\begin{lstlisting}[numbers=left]
int search(int a[],int n)
{

  int i;
  int best_val=a[0];

  for(i=1;i<n;i++)
    {
      if(a[i]>best_val)
	best_val= a[i];
    }

  return best_val;
}
\end{lstlisting}
\caption{Search for the largest element in an unsorted list. This
  function searches all the elements to see which is the largest, the
  inner loop always runs $n-1$ times since it doesn't know until it
  has looked at every element which is going to be the largest. This
  program is implemented as {\tt
    find\_largest.c}.\label{c_largest_linear_search}.}
\end{table}

Finally, little oh notation is a stricter version of big Oh notation
that is used in some more mathematical context, basically $f(n)\in
o(g(n))$ is $f(n)$ is less than $cg(n)$ for any choice of $c$, if $n$
is large enough:
\begin{equation}
\o(g(n))=\{f(n)| \exists n_0>0\in {\bf N}\mbox{ so that }|f(n)|\ge
c|g(n)|\,\forall n\ge n_0 \mbox{ and }\forall\,c>0\in {\bf R}\}
\end{equation}

\end{document}

\caption{This shows $\log_2{x}$ and $x-1$ plots for $x\in[1,4]$, the one has been taken from $x$ to make them easier to compare, the key point is that the $x$ grows faster.\label{fig_log}}
\end{figure}


\begin{figure}
\documentclass[11pt,a4paper]{scrartcl}
\typearea{12}
\usepackage{graphicx}
\usepackage{pstricks}
\usepackage{listings}
\lstset{language=C}
\pagestyle{headings}
\markright{COMS11600 - Principles of programming I.2}
\begin{document}

\subsection*{I.2 Big Oh notation}

The \lq{}Big Oh\rq{} notation has already been used to describe the
behavior of the running time of insert sort, we said
\begin{equation}
T(n)\in O(n^2)
\end{equation}
Here we want to formalize this notation. Basically $O(n^2)$ is a set
of function, it is all the function which, for large values of $n$ go
to infinity like $n^2$ at the fastest. By saying $T(n)\in O(n^2)$ we
are saying that $T(n)$ is one of these functions, its large $n$
behavior is, as worst, like $n^2$. 

Specifically, the definition for $g(n)$ for all $n$
\begin{equation}
O(g(n))=\{f(n)| \exists n_0>0\in {\bf N}\mbox{ and }c>0\in {\bf R}\mbox{ with }|f(n)|\le c|g(n)|\,\forall n\ge n_0\}
\end{equation}
This definition is quite dense, but we can break it down: it says that
$O(g(n))$ is a set of functions, the curly brackets mean
\lq{}set\rq{}. $f(n)$ is in the set if it has a particular
property: the \lq$|$\rq{} can be read as \lq{}such that\rq{} or
\lq{}with the property that\rq{} and so this is the set of $f(n)$'s
where $f(n)$ has the property on the right of the $|$. Now,
\lq{}$\exists$\rq{} means \lq{}there exists\rq{} and
\lq{}$\forall$\rq{} means \lq{}for all\rq{}, so the defining property
says it is possible to find a positive natural number $n_0$ and a
positive real number $c$ for that if you choose a value of $n$ at
least as big as $n_0$ then $f(n)$ is no bigger than $cg(n)$, ${\bf N}$
and ${\bf R}$ stand for the natural and real numbers. Notice the
absolute value signs, this is about $|f(n)|$ and $|g(n)|$, in fact,
here we are interested in run times, so we will deal with functions
that are non-negative, or are non-negative provided $n$ is larger than
some threshold, for example, $\log_2{n}$ will be important,
$\log_2{n}$ is positive provided $n>1$.

In short, $f(n)$ can do all sorts of crazy stuff for small values of
$n$ but, if you take $n$ large enough, its behavior is bounded by the
behavior of $g(n)$. Now it doesn't say it is bounded by $g(n)$, it is
a statement about the behavior, that's the role of the $c$. If you
know the formal definition of limits you can see that the definition
of $O(g(n))$ has this wrapped up in it, if says
\begin{equation}
f(n)\in O(g(n))\iff \lim_{n\rightarrow \infty}\frac{f(n)}{g(n)}<\infty
\end{equation}

Here are some examples, say 
\begin{equation}
T(n)=5n^2+n+6
\end{equation}
then 
\begin{equation}
T(n)\in O(n^2)
\end{equation}
by, for example, taking $c=5+1+6=12$ then 
\begin{equation}
12n^2\ge 5n^2+n+6
\end{equation}
provided $n\ge 1$ so $n_0=1$ here. This could also be succinctly demonstrated using the limit
\begin{equation}
\lim_{n\rightarrow \infty} \frac{5n^2+n+6}{n^2}=\lim_{n\rightarrow \infty}5+\lim_{n\rightarrow \infty}\frac{1}{n}+\lim_{n\rightarrow \infty}\frac{6}{n^2}=5<\infty
\end{equation}

However, 
\begin{equation}
T(n)\not\in O(n)
\end{equation}
Say we chosen some value $c$ then
\begin{equation}
5n^2+n+6>cn
\end{equation}
for large enough $n$, to check this divide both sides by $n$ so we need to show that $n$ can be chosen so that
\begin{equation}
5n+1+\frac{c}{n}>c
\end{equation}
Since 
\begin{equation}
5n+1+\frac{c}{n}>5n+1
\end{equation}
then, if $n>c/5$
\begin{equation}
5n+1+\frac{c}{n}>5\frac{c}{5}+1=c+1>c
\end{equation}
so, no matter what value of $c$ is chosen, making $n>5/n$ implies
\begin{equation}
5n^2+n+6>cn
\end{equation}
so $5n^2+n+6\not\in O(n)$. Again, the limit does the same job
\begin{equation}
\lim_{n\rightarrow \infty} \frac{5n^2+n+6}{n}=\lim_{n\rightarrow \infty}5n+\lim_{n\rightarrow \infty}1+\lim_{n\rightarrow \infty}\frac{6}{n}=\infty
\end{equation}


In practice, if
\begin{equation}
T(n)=a_rn^r+a_{r-1}n^{r-1}+\ldots+a_1n+a_0
\end{equation}
then $T(n)\in O(n^r)$. 

The logarithm has the funny property that it goes to infinity, but it
does so slower than $n$:
\begin{equation}
\lim_{n\rightarrow \infty}\frac{\log_2{n}}{n}=0
\end{equation}
Here, in line with standard practice in computer science, we are using
the log to the base two, in fact changing bases only causes a change
of an overall constant, see Table~\ref{math_logs} for a reminder of
the properties of the log. The limit of $\log_2{n}/n$ can be calculated
using l'H\^{o}pital's rule, here we'll just look at a plot,
Fig.~\ref{fig_log}. Now, just as $\log_2{n}$ grows very slowly, $2^n$ grows very fast, 
\begin{equation}
\lim_{n\rightarrow \infty}\frac{n^r}{2^n}=0
\end{equation}
for any finite value of $r$, worse still is $n!$, pronounced $n$-factorial
\begin{equation}
n!=n(n-1)(n-2) . . . 1
\end{equation}
If you algorithm is in $O(n!)$ you will probably need a different
algorithm. A table of different values is given as
Table~\ref{table_n_values}, mostly to emphasis how quickly $n!$ gets
big.

Now, in mathematics we call something \lq{}an abuse of notation\rq{}
if it is common to write something that doesn't quite make sense but
acts as a shorthand for something that does. Now $O(g(n))$ is a set of
functions whose large $n$ behavior is bounded by $g(n)$ so in
algorithms, being in $O(g(n))$ is a property of $T(n)$, the formula
for the running time of the algorithm. However, it is a standard abuse
of notation to say an algorithm is $O(g(n))$ for some $g(n)$ is
$T(n)\in O(g(n))$ for all cases. This is another way of saying that
the worst behavior of the algorithm isn't any worse than the behavior
of $g(n)$ for large $n$ so all possible $T(n)$ are elements of
$O(g(n))$.

\begin{table}
The logarithm is the opposite of the exponent: if
\begin{equation}
a^b=c
\end{equation}
then 
\begin{equation}
\log_a{c}=b
\end{equation}
or, written in one line
\begin{equation}
a^{\log_a{c}}=c
\end{equation}
All the laws of logs can be worked out from the laws of
exponents. Hence, since $a^0=1$ we have $\log_a{1}=0$. In a similar
way, other rules of logs can be deduced like
\begin{eqnarray}
\log_a{c_1c_2}&=&\log_a{c_1}+\log_a{c_2}\cr
\log_a{\frac{c_1}{c_2}}&=&\log_a{c_1}-\log_a{c_2}\cr
\log_a c^d&=&d\log_a c
\end{eqnarray}
and so on.

As for the change of base, let $b=\log_{a_1}{c}$ so $a_1^{b}=c$. Now take the log to the base $a_2$ of both sides
\begin{equation}
b\log_{a_2}a_1=\log_{a_2}{c}
\end{equation}
and then solve for $b$
\begin{equation}
b=\frac{\log_{a_2}c}{\log_{a_2}a_1}
\end{equation}
and, substituting back the formula for $b$
\begin{equation}
\log_{a_1}{c}=\frac{\log_{a_2}c}{\log_{a_2}a_1}
\end{equation}
Thus we see, that changing bases is just a matter of a
multiplicative factor. For example, to change from base $e$ to base two
\begin{equation}
\log_{2}{x}=\frac{\log_e{x}}{\log_e{2}}\approx\frac{\log_e{x}}{0.6931}
\end{equation}
Common bases are $\log_2{x}$ used in computer science, $\log_e{x}$
sometimes written $\ln{x}$ used in mathematics and $\log_{10}{x}$ used
in chemistry. The base two is used because of its link to bits and
also, as we will see, because of its relationship with algorithms that
divide data into two piles. The natural log $\ln{x}$ is used where
differential equations are common since
\begin{equation}
\frac{d}{dx}\ln{x}=\frac{1}{x}
\end{equation}
\caption{A reminder about logarithms. This is a quick summary of some of the laws of logs.\label{math_logs}}
\end{table}

\begin{figure}
\include{I.2.log}
\caption{This shows $\log_2{x}$ and $x-1$ plots for $x\in[1,4]$, the one has been taken from $x$ to make them easier to compare, the key point is that the $x$ grows faster.\label{fig_log}}
\end{figure}


\begin{figure}
\include{I.2.exp}
\caption{This shows $2^x$ and $x^2-1$ plots for $x\in[0,6]$, clearly $2^x$ quickly overtakes $x^2+1$, this will happen for any power of $x$. \label{fig_log}}
\end{figure}

\begin{table}
\begin{tabular}{l|cccccc}
        $n$    &1   &2&4   &16  &128&1024\\
$\log{n}$      &0   &1&2   &4   &7  &10\\
$n\log{n}$     &0   &2&8   &64  &896&10240\\
$n^2$     &1   &4&16&256&16384&1048576\\
$2^n$     &2   &4&16&65536&$3.4\times 10^{38}$&$1.8\times 10^{307}$\\
$n!$      &1   &2&24&$2.1\times 10^{13}$&$3.85\times 10^{305}$&$5.4\times10^{2369}$
\end{tabular}
\vskip 1cm The website\\ {\tt
  http://markknowsnothing.com/cgi-bin/calculator.php}\\ was used for
the $2^n$ calculations and\\ {\tt
  http://www.calculatorsoup.com/calculators/discretemathematics/factorials.php}\\ for
the $n!$ calculations; these give answers even when the answer is very
large.

\caption{Different values of $n$ for some functions.  \label{table_n_values}
}
\end{table}

\subsubsection*{I.2.1 Examples: linear and binary search}

Lets do a quick example; searching for the index of an element in a
sorted list. A completely terrible way to do this is linear search,
this is terrible because it doesn't make use of the fact that the list
is sorted. A code listing is given in Table~\ref{c_linear_search}. We
can see straight away that the code between lines 6 and 10 is run $n$
times in the worst case, everything else is run once and so this
algorithm is $O(n)$.

\begin{table}
\begin{lstlisting}[numbers=left]
int search(int a[],int n, int val)
{

  int i;

  for(i=0;i<n;i++)
    {
      if(a[i]==val)
	return i;
    }

  return -1;
}
\end{lstlisting}
\caption{Linear search. This function searches the entries in the array a and returns the index when it finds val, if it doesn't find val it returns -1. The program {\tt linear\_search.c} implements this.\label{c_linear_search}.}
\end{table}

A much better way to search a sorted array is binary search. This is
an example of a \lq{}divide and conquer\rq{} algorithm, many of the
fastest algorithms use divide and conquer. It will be clear to you
that this algorithm would be better written using recursion, this is
typical of divide and conquer, but we haven't looked at analysing
recursion yet. The idea is to divide the array in half and check which
half, the half with bigger numbers or the half with smaller numbers,
the value we are searching for belongs to and to keep doing this,
dividing the remaining part of the array into two parts again and again
until the remaining part of the array that is being search has only
one element. A code listing is given in Table~\ref{c_binary_search}.

\begin{table}
\begin{lstlisting}[numbers=left] 
int search(int a[],int n, int val)
{
  int mid, low=0, high=n-1;

  while(low<=high)
    {
      mid=(low+high)/2;
      if(a[mid]==val)
	return mid;
      else if(val>a[mid])
	low=mid+1;
      else
	high=mid -1;
    }

  return -1;
}
\end{lstlisting}
\caption{Binary search. This function starts in the middle of the array
  and checks if the value there is bigger or smaller than val, if it
  is bigger then it does the same in the top half of the array, if it
  is smaller, in the bottom half and then repeats until there are no
  elements left. The program {\tt binary\_search.c} implements
  this.\label{c_binary_search}.}
\end{table}

This search is extremely fast. There is a chance that a[mid]==val,
after a small number of iterations, indeed, if the middle value of the
array is the search value it will halt after only one
iteration. However, as usual, we assume the worst case, in which case
the algorithm runs to end, dividing the number of elements in half
each time. Ignoring the integer rounding effects, it goes like
Table~\ref{tab_binary_search}. Starting with $n$ states each
subsequent iteration halves the number of states until the last one
when there is one state left. Thus
\begin{equation}
1=\frac{n}{2^{T(n)-1}}
\end{equation}
and taking the log of both sides
\begin{equation}
0=\log_2{n}-(T(n)-1)
\end{equation}
using $\log_2{1}=1$, $\log_2{a^b}=b\log_2{a}$ and $\log_2{2}=1$. Hence
\begin{equation}
T(n)=\log_2{n}+1
\end{equation}
and this algorithm is $O(\log_2{n})$.

\begin{table}
\begin{center}
\begin{tabular}{llllllll}
1&2&3&4&\ldots&k&\ldots&T(n)\\
\hline
$n$&$\frac{n}{2}$&$\frac{n}{4}$&$\frac{n}{8}$&$\ldots$&$\frac{n}{2^{k-1}}$&$\ldots$&1
\end{tabular}
\end{center}
\caption{The number of elements left for binary search.\label{tab_binary_search}}
\end{table}

So, to reiterate; the usual way to examine the behavior of an
algorithm is to look at the worst case run time. This is because the
best case run time is often exceptional, like the one for binary
search if the first guess happens to be correct. The average run time
is often hard to calculate, both because it is often difficult to do
the mathematics and because it would often mean having some
description of how the initial data is distributed. Typically the
worst run time is also \lq{}of the same order\rq{} as the average run
time. We will see an exception to this later on in the case of quick
sort in which the worst case behavior is unusual. The big-Oh notation
is used for describing an algorithm, if the algorithm is said to be
$O(g(n))$ we mean $T(n)\in O(g(n))$ no matter what the initial
condition. Since $O(g(n))$ involves an upper bound $T(n)<cg(n)$ this
makes sense.


\subsubsection*{Other big Letter notations, small oh notation.}

There is another set, $\Omega(g(n))$ with a definition similar to
$O(g(n))$ that is used for describing the best case behavior. This
requires a lower bound rather than an upper bound, so the obvious
definition is
\begin{equation}
\Omega(g(n))=\{f(n)| \exists n_0>0\in {\bf N}\mbox{ and }c>0\in {\bf R}\mbox{ with }|f(n)|\ge c|g(n)|\,\forall n\ge n_0\}
\end{equation}
in other words, the same thing, but with the $\le$ symbol replaced by
a $\ge$. In fact, there is some ambiguity about this definition,
number theorist use a slightly different one. Either way, it isn't
used very often in computer science because algorithms are very
frequently $\Omega(1)$; in the best case scenario the problem is in
some sense already solve, the array already sorted for example, and
the algorithm finishs in one step. 

There is also a set of function that are both bounded above and below
by the same $g(n)$
\begin{equation}
\Theta(g(n))=\Omega(g(n))\cap O(g(n))
\end{equation}
This works because it is possible for
\begin{equation}
c_1 g(n)\le f(n)\le c_2g(n)
\end{equation}
for different $c_1$ and $c_2$. It would be very unusual for this to
apply to an algorithm, it would mean that $T(n)$ has the same behavior
for large $n$ no matter whether it is the best case or the worst case
scenario. There is a na\"ive largest element function in
Table~\ref{c_largest_linear_search} which is $\Theta(n)$. It searches
for the largest value in an unsorted array by looking at each element
in turn. In fact, for a completely unsorted array this is the best
algorithm, but, in practice, if finding the largest element in a set
is an important and frequent procedure, a special data structure,
called a heap, is used to keep track of which element is largest.

\begin{table}
\begin{lstlisting}[numbers=left]
int search(int a[],int n)
{

  int i;
  int best_val=a[0];

  for(i=1;i<n;i++)
    {
      if(a[i]>best_val)
	best_val= a[i];
    }

  return best_val;
}
\end{lstlisting}
\caption{Search for the largest element in an unsorted list. This
  function searches all the elements to see which is the largest, the
  inner loop always runs $n-1$ times since it doesn't know until it
  has looked at every element which is going to be the largest. This
  program is implemented as {\tt
    find\_largest.c}.\label{c_largest_linear_search}.}
\end{table}

Finally, little oh notation is a stricter version of big Oh notation
that is used in some more mathematical context, basically $f(n)\in
o(g(n))$ is $f(n)$ is less than $cg(n)$ for any choice of $c$, if $n$
is large enough:
\begin{equation}
\o(g(n))=\{f(n)| \exists n_0>0\in {\bf N}\mbox{ so that }|f(n)|\ge
c|g(n)|\,\forall n\ge n_0 \mbox{ and }\forall\,c>0\in {\bf R}\}
\end{equation}

\end{document}

\caption{This shows $2^x$ and $x^2-1$ plots for $x\in[0,6]$, clearly $2^x$ quickly overtakes $x^2+1$, this will happen for any power of $x$. \label{fig_log}}
\end{figure}

\begin{table}
\begin{tabular}{l|cccccc}
        $n$    &1   &2&4   &16  &128&1024\\
$\log{n}$      &0   &1&2   &4   &7  &10\\
$n\log{n}$     &0   &2&8   &64  &896&10240\\
$n^2$     &1   &4&16&256&16384&1048576\\
$2^n$     &2   &4&16&65536&$3.4\times 10^{38}$&$1.8\times 10^{307}$\\
$n!$      &1   &2&24&$2.1\times 10^{13}$&$3.85\times 10^{305}$&$5.4\times10^{2369}$
\end{tabular}
\vskip 1cm The website\\ {\tt
  http://markknowsnothing.com/cgi-bin/calculator.php}\\ was used for
the $2^n$ calculations and\\ {\tt
  http://www.calculatorsoup.com/calculators/discretemathematics/factorials.php}\\ for
the $n!$ calculations; these give answers even when the answer is very
large.

\caption{Different values of $n$ for some functions.  \label{table_n_values}
}
\end{table}

\subsubsection*{I.2.1 Examples: linear and binary search}

Lets do a quick example; searching for the index of an element in a
sorted list. A completely terrible way to do this is linear search,
this is terrible because it doesn't make use of the fact that the list
is sorted. A code listing is given in Table~\ref{c_linear_search}. We
can see straight away that the code between lines 6 and 10 is run $n$
times in the worst case, everything else is run once and so this
algorithm is $O(n)$.

\begin{table}
\begin{lstlisting}[numbers=left]
int search(int a[],int n, int val)
{

  int i;

  for(i=0;i<n;i++)
    {
      if(a[i]==val)
	return i;
    }

  return -1;
}
\end{lstlisting}
\caption{Linear search. This function searches the entries in the array a and returns the index when it finds val, if it doesn't find val it returns -1. The program {\tt linear\_search.c} implements this.\label{c_linear_search}.}
\end{table}

A much better way to search a sorted array is binary search. This is
an example of a \lq{}divide and conquer\rq{} algorithm, many of the
fastest algorithms use divide and conquer. It will be clear to you
that this algorithm would be better written using recursion, this is
typical of divide and conquer, but we haven't looked at analysing
recursion yet. The idea is to divide the array in half and check which
half, the half with bigger numbers or the half with smaller numbers,
the value we are searching for belongs to and to keep doing this,
dividing the remaining part of the array into two parts again and again
until the remaining part of the array that is being search has only
one element. A code listing is given in Table~\ref{c_binary_search}.

\begin{table}
\begin{lstlisting}[numbers=left] 
int search(int a[],int n, int val)
{
  int mid, low=0, high=n-1;

  while(low<=high)
    {
      mid=(low+high)/2;
      if(a[mid]==val)
	return mid;
      else if(val>a[mid])
	low=mid+1;
      else
	high=mid -1;
    }

  return -1;
}
\end{lstlisting}
\caption{Binary search. This function starts in the middle of the array
  and checks if the value there is bigger or smaller than val, if it
  is bigger then it does the same in the top half of the array, if it
  is smaller, in the bottom half and then repeats until there are no
  elements left. The program {\tt binary\_search.c} implements
  this.\label{c_binary_search}.}
\end{table}

This search is extremely fast. There is a chance that a[mid]==val,
after a small number of iterations, indeed, if the middle value of the
array is the search value it will halt after only one
iteration. However, as usual, we assume the worst case, in which case
the algorithm runs to end, dividing the number of elements in half
each time. Ignoring the integer rounding effects, it goes like
Table~\ref{tab_binary_search}. Starting with $n$ states each
subsequent iteration halves the number of states until the last one
when there is one state left. Thus
\begin{equation}
1=\frac{n}{2^{T(n)-1}}
\end{equation}
and taking the log of both sides
\begin{equation}
0=\log_2{n}-(T(n)-1)
\end{equation}
using $\log_2{1}=1$, $\log_2{a^b}=b\log_2{a}$ and $\log_2{2}=1$. Hence
\begin{equation}
T(n)=\log_2{n}+1
\end{equation}
and this algorithm is $O(\log_2{n})$.

\begin{table}
\begin{center}
\begin{tabular}{llllllll}
1&2&3&4&\ldots&k&\ldots&T(n)\\
\hline
$n$&$\frac{n}{2}$&$\frac{n}{4}$&$\frac{n}{8}$&$\ldots$&$\frac{n}{2^{k-1}}$&$\ldots$&1
\end{tabular}
\end{center}
\caption{The number of elements left for binary search.\label{tab_binary_search}}
\end{table}

So, to reiterate; the usual way to examine the behavior of an
algorithm is to look at the worst case run time. This is because the
best case run time is often exceptional, like the one for binary
search if the first guess happens to be correct. The average run time
is often hard to calculate, both because it is often difficult to do
the mathematics and because it would often mean having some
description of how the initial data is distributed. Typically the
worst run time is also \lq{}of the same order\rq{} as the average run
time. We will see an exception to this later on in the case of quick
sort in which the worst case behavior is unusual. The big-Oh notation
is used for describing an algorithm, if the algorithm is said to be
$O(g(n))$ we mean $T(n)\in O(g(n))$ no matter what the initial
condition. Since $O(g(n))$ involves an upper bound $T(n)<cg(n)$ this
makes sense.


\subsubsection*{Other big Letter notations, small oh notation.}

There is another set, $\Omega(g(n))$ with a definition similar to
$O(g(n))$ that is used for describing the best case behavior. This
requires a lower bound rather than an upper bound, so the obvious
definition is
\begin{equation}
\Omega(g(n))=\{f(n)| \exists n_0>0\in {\bf N}\mbox{ and }c>0\in {\bf R}\mbox{ with }|f(n)|\ge c|g(n)|\,\forall n\ge n_0\}
\end{equation}
in other words, the same thing, but with the $\le$ symbol replaced by
a $\ge$. In fact, there is some ambiguity about this definition,
number theorist use a slightly different one. Either way, it isn't
used very often in computer science because algorithms are very
frequently $\Omega(1)$; in the best case scenario the problem is in
some sense already solve, the array already sorted for example, and
the algorithm finishs in one step. 

There is also a set of function that are both bounded above and below
by the same $g(n)$
\begin{equation}
\Theta(g(n))=\Omega(g(n))\cap O(g(n))
\end{equation}
This works because it is possible for
\begin{equation}
c_1 g(n)\le f(n)\le c_2g(n)
\end{equation}
for different $c_1$ and $c_2$. It would be very unusual for this to
apply to an algorithm, it would mean that $T(n)$ has the same behavior
for large $n$ no matter whether it is the best case or the worst case
scenario. There is a na\"ive largest element function in
Table~\ref{c_largest_linear_search} which is $\Theta(n)$. It searches
for the largest value in an unsorted array by looking at each element
in turn. In fact, for a completely unsorted array this is the best
algorithm, but, in practice, if finding the largest element in a set
is an important and frequent procedure, a special data structure,
called a heap, is used to keep track of which element is largest.

\begin{table}
\begin{lstlisting}[numbers=left]
int search(int a[],int n)
{

  int i;
  int best_val=a[0];

  for(i=1;i<n;i++)
    {
      if(a[i]>best_val)
	best_val= a[i];
    }

  return best_val;
}
\end{lstlisting}
\caption{Search for the largest element in an unsorted list. This
  function searches all the elements to see which is the largest, the
  inner loop always runs $n-1$ times since it doesn't know until it
  has looked at every element which is going to be the largest. This
  program is implemented as {\tt
    find\_largest.c}.\label{c_largest_linear_search}.}
\end{table}

Finally, little oh notation is a stricter version of big Oh notation
that is used in some more mathematical context, basically $f(n)\in
o(g(n))$ is $f(n)$ is less than $cg(n)$ for any choice of $c$, if $n$
is large enough:
\begin{equation}
\o(g(n))=\{f(n)| \exists n_0>0\in {\bf N}\mbox{ so that }|f(n)|\ge
c|g(n)|\,\forall n\ge n_0 \mbox{ and }\forall\,c>0\in {\bf R}\}
\end{equation}

\end{document}

\caption{This shows $2^x$ and $x^2-1$ plots for $x\in[0,6]$, clearly $2^x$ quickly overtakes $x^2+1$, this will happen for any power of $x$. \label{fig_log}}
\end{figure}

\begin{table}
\begin{tabular}{l|cccccc}
        $n$    &1   &2&4   &16  &128&1024\\
$\log{n}$      &0   &1&2   &4   &7  &10\\
$n\log{n}$     &0   &2&8   &64  &896&10240\\
$n^2$     &1   &4&16&256&16384&1048576\\
$2^n$     &2   &4&16&65536&$3.4\times 10^{38}$&$1.8\times 10^{307}$\\
$n!$      &1   &2&24&$2.1\times 10^{13}$&$3.85\times 10^{305}$&$5.4\times10^{2369}$
\end{tabular}
\vskip 1cm The website\\ {\tt
  http://markknowsnothing.com/cgi-bin/calculator.php}\\ was used for
the $2^n$ calculations and\\ {\tt
  http://www.calculatorsoup.com/calculators/discretemathematics/factorials.php}\\ for
the $n!$ calculations; these give answers even when the answer is very
large.

\caption{Different values of $n$ for some functions.  \label{table_n_values}
}
\end{table}

\subsubsection*{I.2.1 Examples: linear and binary search}

Lets do a quick example; searching for the index of an element in a
sorted list. A completely terrible way to do this is linear search,
this is terrible because it doesn't make use of the fact that the list
is sorted. A code listing is given in Table~\ref{c_linear_search}. We
can see straight away that the code between lines 6 and 10 is run $n$
times in the worst case, everything else is run once and so this
algorithm is $O(n)$.

\begin{table}
\begin{lstlisting}[numbers=left]
int search(int a[],int n, int val)
{

  int i;

  for(i=0;i<n;i++)
    {
      if(a[i]==val)
	return i;
    }

  return -1;
}
\end{lstlisting}
\caption{Linear search. This function searches the entries in the array a and returns the index when it finds val, if it doesn't find val it returns -1. The program {\tt linear\_search.c} implements this.\label{c_linear_search}.}
\end{table}

A much better way to search a sorted array is binary search. This is
an example of a \lq{}divide and conquer\rq{} algorithm, many of the
fastest algorithms use divide and conquer. It will be clear to you
that this algorithm would be better written using recursion, this is
typical of divide and conquer, but we haven't looked at analysing
recursion yet. The idea is to divide the array in half and check which
half, the half with bigger numbers or the half with smaller numbers,
the value we are searching for belongs to and to keep doing this,
dividing the remaining part of the array into two parts again and again
until the remaining part of the array that is being search has only
one element. A code listing is given in Table~\ref{c_binary_search}.

\begin{table}
\begin{lstlisting}[numbers=left] 
int search(int a[],int n, int val)
{
  int mid, low=0, high=n-1;

  while(low<=high)
    {
      mid=(low+high)/2;
      if(a[mid]==val)
	return mid;
      else if(val>a[mid])
	low=mid+1;
      else
	high=mid -1;
    }

  return -1;
}
\end{lstlisting}
\caption{Binary search. This function starts in the middle of the array
  and checks if the value there is bigger or smaller than val, if it
  is bigger then it does the same in the top half of the array, if it
  is smaller, in the bottom half and then repeats until there are no
  elements left. The program {\tt binary\_search.c} implements
  this.\label{c_binary_search}.}
\end{table}

This search is extremely fast. There is a chance that a[mid]==val,
after a small number of iterations, indeed, if the middle value of the
array is the search value it will halt after only one
iteration. However, as usual, we assume the worst case, in which case
the algorithm runs to end, dividing the number of elements in half
each time. Ignoring the integer rounding effects, it goes like
Table~\ref{tab_binary_search}. Starting with $n$ states each
subsequent iteration halves the number of states until the last one
when there is one state left. Thus
\begin{equation}
1=\frac{n}{2^{T(n)-1}}
\end{equation}
and taking the log of both sides
\begin{equation}
0=\log_2{n}-(T(n)-1)
\end{equation}
using $\log_2{1}=1$, $\log_2{a^b}=b\log_2{a}$ and $\log_2{2}=1$. Hence
\begin{equation}
T(n)=\log_2{n}+1
\end{equation}
and this algorithm is $O(\log_2{n})$.

\begin{table}
\begin{center}
\begin{tabular}{llllllll}
1&2&3&4&\ldots&k&\ldots&T(n)\\
\hline
$n$&$\frac{n}{2}$&$\frac{n}{4}$&$\frac{n}{8}$&$\ldots$&$\frac{n}{2^{k-1}}$&$\ldots$&1
\end{tabular}
\end{center}
\caption{The number of elements left for binary search.\label{tab_binary_search}}
\end{table}

So, to reiterate; the usual way to examine the behavior of an
algorithm is to look at the worst case run time. This is because the
best case run time is often exceptional, like the one for binary
search if the first guess happens to be correct. The average run time
is often hard to calculate, both because it is often difficult to do
the mathematics and because it would often mean having some
description of how the initial data is distributed. Typically the
worst run time is also \lq{}of the same order\rq{} as the average run
time. We will see an exception to this later on in the case of quick
sort in which the worst case behavior is unusual. The big-Oh notation
is used for describing an algorithm, if the algorithm is said to be
$O(g(n))$ we mean $T(n)\in O(g(n))$ no matter what the initial
condition. Since $O(g(n))$ involves an upper bound $T(n)<cg(n)$ this
makes sense.


\subsubsection*{Other big Letter notations, small oh notation.}

There is another set, $\Omega(g(n))$ with a definition similar to
$O(g(n))$ that is used for describing the best case behavior. This
requires a lower bound rather than an upper bound, so the obvious
definition is
\begin{equation}
\Omega(g(n))=\{f(n)| \exists n_0>0\in {\bf N}\mbox{ and }c>0\in {\bf R}\mbox{ with }|f(n)|\ge c|g(n)|\,\forall n\ge n_0\}
\end{equation}
in other words, the same thing, but with the $\le$ symbol replaced by
a $\ge$. In fact, there is some ambiguity about this definition,
number theorist use a slightly different one. Either way, it isn't
used very often in computer science because algorithms are very
frequently $\Omega(1)$; in the best case scenario the problem is in
some sense already solve, the array already sorted for example, and
the algorithm finishs in one step. 

There is also a set of function that are both bounded above and below
by the same $g(n)$
\begin{equation}
\Theta(g(n))=\Omega(g(n))\cap O(g(n))
\end{equation}
This works because it is possible for
\begin{equation}
c_1 g(n)\le f(n)\le c_2g(n)
\end{equation}
for different $c_1$ and $c_2$. It would be very unusual for this to
apply to an algorithm, it would mean that $T(n)$ has the same behavior
for large $n$ no matter whether it is the best case or the worst case
scenario. There is a na\"ive largest element function in
Table~\ref{c_largest_linear_search} which is $\Theta(n)$. It searches
for the largest value in an unsorted array by looking at each element
in turn. In fact, for a completely unsorted array this is the best
algorithm, but, in practice, if finding the largest element in a set
is an important and frequent procedure, a special data structure,
called a heap, is used to keep track of which element is largest.

\begin{table}
\begin{lstlisting}[numbers=left]
int search(int a[],int n)
{

  int i;
  int best_val=a[0];

  for(i=1;i<n;i++)
    {
      if(a[i]>best_val)
	best_val= a[i];
    }

  return best_val;
}
\end{lstlisting}
\caption{Search for the largest element in an unsorted list. This
  function searches all the elements to see which is the largest, the
  inner loop always runs $n-1$ times since it doesn't know until it
  has looked at every element which is going to be the largest. This
  program is implemented as {\tt
    find\_largest.c}.\label{c_largest_linear_search}.}
\end{table}

Finally, little oh notation is a stricter version of big Oh notation
that is used in some more mathematical context, basically $f(n)\in
o(g(n))$ is $f(n)$ is less than $cg(n)$ for any choice of $c$, if $n$
is large enough:
\begin{equation}
\o(g(n))=\{f(n)| \exists n_0>0\in {\bf N}\mbox{ so that }|f(n)|\ge
c|g(n)|\,\forall n\ge n_0 \mbox{ and }\forall\,c>0\in {\bf R}\}
\end{equation}

\end{document}

\caption{This shows $2^x$ and $x^2-1$ plots for $x\in[0,6]$, clearly $2^x$ quickly overtakes $x^2+1$, this will happen for any power of $x$. \label{fig_log}}
\end{figure}

\begin{table}
\begin{tabular}{l|cccccc}
        $n$    &1   &2&4   &16  &128&1024\\
$\log{n}$      &0   &1&2   &4   &7  &10\\
$n\log{n}$     &0   &2&8   &64  &896&10240\\
$n^2$     &1   &4&16&256&16384&1048576\\
$2^n$     &2   &4&16&65536&$3.4\times 10^{38}$&$1.8\times 10^{307}$\\
$n!$      &1   &2&24&$2.1\times 10^{13}$&$3.85\times 10^{305}$&$5.4\times10^{2369}$
\end{tabular}
\vskip 1cm The website\\ {\tt
  http://markknowsnothing.com/cgi-bin/calculator.php}\\ was used for
the $2^n$ calculations and\\ {\tt
  http://www.calculatorsoup.com/calculators/discretemathematics/factorials.php}\\ for
the $n!$ calculations; these give answers even when the answer is very
large.

\caption{Different values of $n$ for some functions.  \label{table_n_values}
}
\end{table}

\subsubsection*{I.2.1 Examples: linear and binary search}

Lets do a quick example; searching for the index of an element in a
sorted list. A completely terrible way to do this is linear search,
this is terrible because it doesn't make use of the fact that the list
is sorted. A code listing is given in Table~\ref{c_linear_search}. We
can see straight away that the code between lines 6 and 10 is run $n$
times in the worst case, everything else is run once and so this
algorithm is $O(n)$.

\begin{table}
\begin{lstlisting}[numbers=left]
int search(int a[],int n, int val)
{

  int i;

  for(i=0;i<n;i++)
    {
      if(a[i]==val)
	return i;
    }

  return -1;
}
\end{lstlisting}
\caption{Linear search. This function searches the entries in the array a and returns the index when it finds val, if it doesn't find val it returns -1. The program {\tt linear\_search.c} implements this.\label{c_linear_search}.}
\end{table}

A much better way to search a sorted array is binary search. This is
an example of a \lq{}divide and conquer\rq{} algorithm, many of the
fastest algorithms use divide and conquer. It will be clear to you
that this algorithm would be better written using recursion, this is
typical of divide and conquer, but we haven't looked at analysing
recursion yet. The idea is to divide the array in half and check which
half, the half with bigger numbers or the half with smaller numbers,
the value we are searching for belongs to and to keep doing this,
dividing the remaining part of the array into two parts again and again
until the remaining part of the array that is being search has only
one element. A code listing is given in Table~\ref{c_binary_search}.

\begin{table}
\begin{lstlisting}[numbers=left] 
int search(int a[],int n, int val)
{
  int mid, low=0, high=n-1;

  while(low<=high)
    {
      mid=(low+high)/2;
      if(a[mid]==val)
	return mid;
      else if(val>a[mid])
	low=mid+1;
      else
	high=mid -1;
    }

  return -1;
}
\end{lstlisting}
\caption{Binary search. This function starts in the middle of the array
  and checks if the value there is bigger or smaller than val, if it
  is bigger then it does the same in the top half of the array, if it
  is smaller, in the bottom half and then repeats until there are no
  elements left. The program {\tt binary\_search.c} implements
  this.\label{c_binary_search}.}
\end{table}

This search is extremely fast. There is a chance that a[mid]==val,
after a small number of iterations, indeed, if the middle value of the
array is the search value it will halt after only one
iteration. However, as usual, we assume the worst case, in which case
the algorithm runs to end, dividing the number of elements in half
each time. Ignoring the integer rounding effects, it goes like
Table~\ref{tab_binary_search}. Starting with $n$ states each
subsequent iteration halves the number of states until the last one
when there is one state left. Thus
\begin{equation}
1=\frac{n}{2^{T(n)-1}}
\end{equation}
and taking the log of both sides
\begin{equation}
0=\log_2{n}-(T(n)-1)
\end{equation}
using $\log_2{1}=1$, $\log_2{a^b}=b\log_2{a}$ and $\log_2{2}=1$. Hence
\begin{equation}
T(n)=\log_2{n}+1
\end{equation}
and this algorithm is $O(\log_2{n})$.

\begin{table}
\begin{center}
\begin{tabular}{llllllll}
1&2&3&4&\ldots&k&\ldots&T(n)\\
\hline
$n$&$\frac{n}{2}$&$\frac{n}{4}$&$\frac{n}{8}$&$\ldots$&$\frac{n}{2^{k-1}}$&$\ldots$&1
\end{tabular}
\end{center}
\caption{The number of elements left for binary search.\label{tab_binary_search}}
\end{table}

So, to reiterate; the usual way to examine the behavior of an
algorithm is to look at the worst case run time. This is because the
best case run time is often exceptional, like the one for binary
search if the first guess happens to be correct. The average run time
is often hard to calculate, both because it is often difficult to do
the mathematics and because it would often mean having some
description of how the initial data is distributed. Typically the
worst run time is also \lq{}of the same order\rq{} as the average run
time. We will see an exception to this later on in the case of quick
sort in which the worst case behavior is unusual. The big-Oh notation
is used for describing an algorithm, if the algorithm is said to be
$O(g(n))$ we mean $T(n)\in O(g(n))$ no matter what the initial
condition. Since $O(g(n))$ involves an upper bound $T(n)<cg(n)$ this
makes sense.


\subsubsection*{Other big Letter notations, small oh notation.}

There is another set, $\Omega(g(n))$ with a definition similar to
$O(g(n))$ that is used for describing the best case behavior. This
requires a lower bound rather than an upper bound, so the obvious
definition is
\begin{equation}
\Omega(g(n))=\{f(n)| \exists n_0>0\in {\bf N}\mbox{ and }c>0\in {\bf R}\mbox{ with }|f(n)|\ge c|g(n)|\,\forall n\ge n_0\}
\end{equation}
in other words, the same thing, but with the $\le$ symbol replaced by
a $\ge$. In fact, there is some ambiguity about this definition,
number theorist use a slightly different one. Either way, it isn't
used very often in computer science because algorithms are very
frequently $\Omega(1)$; in the best case scenario the problem is in
some sense already solve, the array already sorted for example, and
the algorithm finishs in one step. 

There is also a set of function that are both bounded above and below
by the same $g(n)$
\begin{equation}
\Theta(g(n))=\Omega(g(n))\cap O(g(n))
\end{equation}
This works because it is possible for
\begin{equation}
c_1 g(n)\le f(n)\le c_2g(n)
\end{equation}
for different $c_1$ and $c_2$. It would be very unusual for this to
apply to an algorithm, it would mean that $T(n)$ has the same behavior
for large $n$ no matter whether it is the best case or the worst case
scenario. There is a na\"ive largest element function in
Table~\ref{c_largest_linear_search} which is $\Theta(n)$. It searches
for the largest value in an unsorted array by looking at each element
in turn. In fact, for a completely unsorted array this is the best
algorithm, but, in practice, if finding the largest element in a set
is an important and frequent procedure, a special data structure,
called a heap, is used to keep track of which element is largest.

\begin{table}
\begin{lstlisting}[numbers=left]
int search(int a[],int n)
{

  int i;
  int best_val=a[0];

  for(i=1;i<n;i++)
    {
      if(a[i]>best_val)
	best_val= a[i];
    }

  return best_val;
}
\end{lstlisting}
\caption{Search for the largest element in an unsorted list. This
  function searches all the elements to see which is the largest, the
  inner loop always runs $n-1$ times since it doesn't know until it
  has looked at every element which is going to be the largest. This
  program is implemented as {\tt
    find\_largest.c}.\label{c_largest_linear_search}.}
\end{table}

Finally, little oh notation is a stricter version of big Oh notation
that is used in some more mathematical context, basically $f(n)\in
o(g(n))$ is $f(n)$ is less than $cg(n)$ for any choice of $c$, if $n$
is large enough:
\begin{equation}
\o(g(n))=\{f(n)| \exists n_0>0\in {\bf N}\mbox{ so that }|f(n)|\ge
c|g(n)|\,\forall n\ge n_0 \mbox{ and }\forall\,c>0\in {\bf R}\}
\end{equation}

\end{document}
